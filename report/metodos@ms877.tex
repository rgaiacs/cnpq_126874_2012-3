\section{Implementação do Método e Testes Computacionais}
O Método Cuthill-McKee Reverso foi implementado, na linguagem C, pelo autor
deste trabalho como um patch para o PCx
(\url{http://pages.cs.wisc.edu/~swright/PCx/}), e a implementação desenvolvida
encontra-se disponível em \url{https://github.com/r-gaia-cs/PCx.git}.

O PCx é um solver de Programação Linear que utilizar a variante de Mehrotra do
Método Preditor Corretor com a estratégia de correção de Gondzio e o Método do
Mínimo Grau Múltiplo de Liu\cite{George:1981:ComputerSolutionPD} (MMD) para o
reordenamento da matriz $A D^{-1} A^T$.

Testou-se a implementação (tendo como último commit aquele cuja identificação
inicia com 259cf79) com todos os problemas das bibliotecas:
\begin{itemize}
  \item ``Netlib LP'' (\url{http://www.netlib.org/lp/}),
  \item ``Kennington'' (\url{http://www.netlib.org/lp/data/kennington/}),
  \item ``Meszaros'' (\url{http://www.sztaki.hu/~meszaros/public_ftp/lptestset/misc/}),
  \item ``PDS'' (\url{http://plato.asu.edu/ftp/lptestset/pds/}),
  \item ``Rail'' (\url{http://plato.asu.edu/ftp/lptestset/rail/}).
\end{itemize}

Juntando os problemas de todas a bibliotecas, foram utilizados mais de 200
problemas dos quais apenas 4 o RCM apresentou um memor número de elementos não
zeros (3 problemas da Netlib e 1 da Meszaros). Em relação a problemas não
resolvidos, destaca-se os problemas da PDS que o MMD resolveu 5 e o RCM 1
dentre 9 problemas.

Informações adicionas sobre o resultado obtido para cada um dos problemas testados
encontram-se nas Tabelas~\ref{tab:resulken}-\ref{tab:resulrail} sendo que
``R'', ``NNZ'', ``IT'' e ``T''
significam, respectivamente, ``código de retorno'', ``número de elementos não
nulos'', ``número de iterações'' e ``tempo computacional'' (em segundos).
%\begin{table}
%  \centering
%  \caption{Resultados experimentais da biblioteca FCTP.}
%  \label{tab:resulfctp}
%  \begin{tabular}{|l|r|r|r|r|r|r|r|r|}
\hline
\multicolumn{1}{|c|}{Problema} & \multicolumn{4}{|c|}{MMD} &         \multicolumn{4}{|c|}{RCM} \\ \hline
\multicolumn{1}{|c|}{Nome} & \multicolumn{1}{|c|}{R} &
        \multicolumn{1}{|c|}{NNZ} & \multicolumn{1}{|c|}{IT} &
        \multicolumn{1}{|c|}{T} & \multicolumn{1}{|c|}{R} &
        \multicolumn{1}{|c|}{NNZ} & \multicolumn{1}{|c|}{IT} &
        \multicolumn{1}{|c|}{T} \\ \hline
bal8x12 & 3 & 0,0000E+00 & 0 & 0,0000E+00 & 3 & 0,0000E+00 & 0 & 0,0000E+00 \\ \hline
bk4x3 & 3 & 0,0000E+00 & 0 & 0,0000E+00 & 3 & 0,0000E+00 & 0 & 0,0000E+00 \\ \hline
gr4x6 & 3 & 0,0000E+00 & 0 & 0,0000E+00 & 3 & 0,0000E+00 & 0 & 0,0000E+00 \\ \hline
n3700 & 3 & 0,0000E+00 & 0 & 0,0000E+00 & 3 & 0,0000E+00 & 0 & 0,0000E+00 \\ \hline
n3701 & 3 & 0,0000E+00 & 0 & 0,0000E+00 & 3 & 0,0000E+00 & 0 & 0,0000E+00 \\ \hline
n3702 & 3 & 0,0000E+00 & 0 & 0,0000E+00 & 3 & 0,0000E+00 & 0 & 0,0000E+00 \\ \hline
n3703 & 3 & 0,0000E+00 & 0 & 0,0000E+00 & 3 & 0,0000E+00 & 0 & 0,0000E+00 \\ \hline
n3704 & 3 & 0,0000E+00 & 0 & 0,0000E+00 & 3 & 0,0000E+00 & 0 & 0,0000E+00 \\ \hline
n3705 & 3 & 0,0000E+00 & 0 & 0,0000E+00 & 3 & 0,0000E+00 & 0 & 0,0000E+00 \\ \hline
n3706 & 3 & 0,0000E+00 & 0 & 0,0000E+00 & 3 & 0,0000E+00 & 0 & 0,0000E+00 \\ \hline
n3707 & 3 & 0,0000E+00 & 0 & 0,0000E+00 & 3 & 0,0000E+00 & 0 & 0,0000E+00 \\ \hline
n3708 & 3 & 0,0000E+00 & 0 & 0,0000E+00 & 3 & 0,0000E+00 & 0 & 0,0000E+00 \\ \hline
n3709 & 3 & 0,0000E+00 & 0 & 0,0000E+00 & 3 & 0,0000E+00 & 0 & 0,0000E+00 \\ \hline
n370a & 3 & 0,0000E+00 & 0 & 0,0000E+00 & 3 & 0,0000E+00 & 0 & 0,0000E+00 \\ \hline
n370b & 3 & 0,0000E+00 & 0 & 0,0000E+00 & 3 & 0,0000E+00 & 0 & 0,0000E+00 \\ \hline
n370c & 3 & 0,0000E+00 & 0 & 0,0000E+00 & 3 & 0,0000E+00 & 0 & 0,0000E+00 \\ \hline
n370d & 3 & 0,0000E+00 & 0 & 0,0000E+00 & 3 & 0,0000E+00 & 0 & 0,0000E+00 \\ \hline
n370e & 3 & 0,0000E+00 & 0 & 0,0000E+00 & 3 & 0,0000E+00 & 0 & 0,0000E+00 \\ \hline
ran10x10a & 3 & 0,0000E+00 & 0 & 0,0000E+00 & 3 & 0,0000E+00 & 0 & 0,0000E+00 \\ \hline
ran10x10b & 3 & 0,0000E+00 & 0 & 0,0000E+00 & 3 & 0,0000E+00 & 0 & 0,0000E+00 \\ \hline
ran10x10c & 3 & 0,0000E+00 & 0 & 0,0000E+00 & 3 & 0,0000E+00 & 0 & 0,0000E+00 \\ \hline
ran10x12 & 3 & 0,0000E+00 & 0 & 0,0000E+00 & 3 & 0,0000E+00 & 0 & 0,0000E+00 \\ \hline
ran10x26 & 3 & 0,0000E+00 & 0 & 0,0000E+00 & 3 & 0,0000E+00 & 0 & 0,0000E+00 \\ \hline
ran12x12 & 3 & 0,0000E+00 & 0 & 0,0000E+00 & 3 & 0,0000E+00 & 0 & 0,0000E+00 \\ \hline
ran12x21 & 3 & 0,0000E+00 & 0 & 0,0000E+00 & 3 & 0,0000E+00 & 0 & 0,0000E+00 \\ \hline
ran13x13 & 3 & 0,0000E+00 & 0 & 0,0000E+00 & 3 & 0,0000E+00 & 0 & 0,0000E+00 \\ \hline
ran14x18 & 3 & 0,0000E+00 & 0 & 0,0000E+00 & 3 & 0,0000E+00 & 0 & 0,0000E+00 \\ \hline
ran16x16 & 3 & 0,0000E+00 & 0 & 0,0000E+00 & 3 & 0,0000E+00 & 0 & 0,0000E+00 \\ \hline
ran17x17 & 3 & 0,0000E+00 & 0 & 0,0000E+00 & 3 & 0,0000E+00 & 0 & 0,0000E+00 \\ \hline
ran4x64 & 3 & 0,0000E+00 & 0 & 0,0000E+00 & 3 & 0,0000E+00 & 0 & 0,0000E+00 \\ \hline
ran6x43 & 3 & 0,0000E+00 & 0 & 0,0000E+00 & 3 & 0,0000E+00 & 0 & 0,0000E+00 \\ \hline
ran8x32 & 3 & 0,0000E+00 & 0 & 0,0000E+00 & 3 & 0,0000E+00 & 0 & 0,0000E+00 \\ \hline
\end{tabular}

%\end{table}
\begin{table}
  \centering
  \caption{Resultados experimentais da biblioteca Kennington.}
  \label{tab:resulken}
  \begin{tabular}{|l|r|r|r|r|r|r|r|r|}
\hline
\multicolumn{1}{|c|}{Problema} & \multicolumn{4}{|c|}{MMD} &         \multicolumn{4}{|c|}{RCM} \\ \hline
\multicolumn{1}{|c|}{Nome} & \multicolumn{1}{|c|}{R} &
        \multicolumn{1}{|c|}{NNZ} & \multicolumn{1}{|c|}{IT} &
        \multicolumn{1}{|c|}{T} & \multicolumn{1}{|c|}{R} &
        \multicolumn{1}{|c|}{NNZ} & \multicolumn{1}{|c|}{IT} &
        \multicolumn{1}{|c|}{T} \\ \hline
cre-a & 0 & 3,3212E+04 & 23 & 1,8000E-01 & 0 & 2,6381E+06 & 18 & 2,3900E+01 \\ \hline
cre-b & 0 & 2,4863E+05 & 41 & 2,0900E+00 & 0 & 5,3999E+06 & 30 & 1,0192E+02 \\ \hline
cre-c & 0 & 2,8528E+04 & 25 & 1,5000E-01 & 0 & 3,4284E+05 & 23 & 1,9000E+00 \\ \hline
cre-d & 0 & 2,1209E+05 & 41 & 1,7300E+00 & 0 & 5,0801E+06 & 30 & 1,0852E+02 \\ \hline
ken-07 & 0 & 1,0034E+04 & 13 & 4,0000E-02 & 0 & 9,6033E+04 & 13 & 2,8000E-01 \\ \hline
ken-11 & 0 & 1,0291E+05 & 20 & 4,5000E-01 & 0 & 1,0638E+07 & 15 & 2,9063E+02 \\ \hline
ken-13 & 0 & 2,9842E+05 & 23 & 1,2400E+00 & 0 & 2,6536E+07 & 20 & 1,4141E+03 \\ \hline
ken-18 & 0 & 1,9289E+06 & 29 & 8,5100E+00 & -1 & 0,0000E+00 & 0 & 0,0000E+00 \\ \hline
osa-07 & 0 & 2,8276E+04 & 22 & 5,2000E-01 & 0 & 1,8898E+05 & 22 & 1,0100E+00 \\ \hline
osa-14 & 0 & 6,0795E+04 & 25 & 1,3900E+00 & 0 & 8,5696E+05 & 24 & 6,7400E+00 \\ \hline
osa-30 & 0 & 1,1508E+05 & 24 & 2,9800E+00 & 0 & 3,1262E+06 & 19 & 3,5540E+01 \\ \hline
osa-60 & 0 & 2,6591E+05 & 33 & 1,0890E+01 & 0 & 1,5942E+07 & 20 & 4,9131E+02 \\ \hline
pds-02 & 0 & 4,4301E+04 & 24 & 1,9000E-01 & 0 & 7,6833E+05 & 20 & 5,1300E+00 \\ \hline
pds-06 & 0 & 5,8934E+05 & 31 & 4,3900E+00 & 0 & 7,7159E+06 & 27 & 2,4528E+02 \\ \hline
pds-10 & 0 & 1,6877E+06 & 33 & 2,0660E+01 & 0 & 1,5180E+07 & 31 & 6,3432E+02 \\ \hline
pds-20 & 0 & 7,0896E+06 & 43 & 1,8843E+02 & 0 & 4,1662E+07 & 41 & 2,8887E+03 \\ \hline
\end{tabular}

\end{table}
\begin{table}
  \centering
  \caption{Resultados experimentais da biblioteca Meszaros (1/3).}
  \label{tab:resulmes0}
  \begin{tabular}{|l|r|r|r|r|r|r|r|r|}
\hline
\multicolumn{1}{|c|}{Problema} & \multicolumn{4}{|c|}{MMD} &         \multicolumn{4}{|c|}{RCM} \\ \hline
\multicolumn{1}{|c|}{Nome} & \multicolumn{1}{|c|}{R} &
        \multicolumn{1}{|c|}{NNZ} & \multicolumn{1}{|c|}{IT} &
        \multicolumn{1}{|c|}{T} & \multicolumn{1}{|c|}{R} &
        \multicolumn{1}{|c|}{NNZ} & \multicolumn{1}{|c|}{IT} &
        \multicolumn{1}{|c|}{T} \\ \hline
3-mix & 0 & 5,4100E+02 & 16 & 1,0000E-02 & 0 & 6,1300E+02 & 16 & 1,0000E-02 \\ \hline
aa01,mps & 3 & 0,0000E+00 & 0 & 0,0000E+00 & 3 & 0,0000E+00 & 0 & 0,0000E+00 \\ \hline
aa03,mps & 3 & 0,0000E+00 & 0 & 0,0000E+00 & 3 & 0,0000E+00 & 0 & 0,0000E+00 \\ \hline
air02 & 0 & 1,1800E+03 & 18 & 2,7000E-01 & 0 & 1,1900E+03 & 18 & 3,0000E-01 \\ \hline
air03 & 0 & 5,3030E+03 & 34 & 7,1000E-01 & 0 & 5,8610E+03 & 34 & 7,2000E-01 \\ \hline
air04 & 0 & 2,0654E+05 & 76 & 3,7700E+00 & 0 & 2,5434E+05 & 96 & 7,0100E+00 \\ \hline
air05 & 0 & 6,2864E+04 & 33 & 6,1000E-01 & 0 & 7,1846E+04 & 25 & 5,5000E-01 \\ \hline
air06 & 15 & 1,8586E+05 & 53 & 2,3200E+00 & 0 & 2,4812E+05 & 28 & 2,1600E+00 \\ \hline
aircraft & 14 & 3,7540E+03 & 10 & 1,2000E-01 & 14 & 3,7540E+03 & 10 & 1,8000E-01 \\ \hline
aramco & 0 & 9,5000E+01 & 9 & 0,0000E+00 & 0 & 1,1800E+02 & 9 & 0,0000E+00 \\ \hline
bas1lp & 0 & 2,2168E+06 & 9 & 1,8540E+01 & 0 & 3,2825E+06 & 9 & 5,1880E+01 \\ \hline
baxter-mat & 0 & 5,8934E+06 & 30 & 3,9960E+01 & 0 & 2,9029E+07 & 27 & 1,0537E+03 \\ \hline
bpmpd & 6 & 0,0000E+00 & 0 & 0,0000E+00 & 6 & 0,0000E+00 & 0 & 0,0000E+00 \\ \hline
complex1 & 14 & 4,7140E+05 & 16 & 4,0300E+00 & 15 & 4,9191E+05 & 37 & 1,0140E+01 \\ \hline
cr42 & 14 & 1,8040E+03 & 15 & 4,0000E-02 & 14 & 1,8040E+03 & 15 & 4,0000E-02 \\ \hline
crew & 0 & 8,5880E+03 & 12 & 1,6000E-01 & 0 & 8,8230E+03 & 12 & 1,6000E-01 \\ \hline
dano3mip & 12 & 1,0902E+06 & 34 & 1,3600E+01 & 12 & 3,4466E+06 & 32 & 7,8680E+01 \\ \hline
dbah00 & 0 & 2,8099E+05 & 43 & 1,2400E+00 & 0 & 4,8499E+06 & 32 & 1,0051E+02 \\ \hline
dbic1 & 0 & 1,8631E+06 & 45 & 2,8700E+01 & -1 & 0,0000E+00 & 0 & 0,0000E+00 \\ \hline
dbir1 & 0 & 2,5235E+06 & 24 & 6,8810E+01 & 0 & 5,7974E+06 & 23 & 2,8589E+02 \\ \hline
dbir2 & 0 & 2,8185E+06 & 26 & 8,4010E+01 & 0 & 6,7499E+06 & 26 & 3,9022E+02 \\ \hline
delfland & 0 & 3,4610E+03 & 46 & 4,0000E-02 & 0 & 1,2234E+04 & 46 & 8,0000E-02 \\ \hline
disp3,mps & 3 & 0,0000E+00 & 0 & 0,0000E+00 & 3 & 0,0000E+00 & 0 & 0,0000E+00 \\ \hline
ds90 & 0 & 9,9700E+04 & 28 & 3,7000E-01 & 0 & 1,1216E+06 & 21 & 9,9200E+00 \\ \hline
dsbmip & 0 & 1,6789E+04 & 48 & 1,6000E-01 & 0 & 5,9151E+04 & 48 & 4,3000E-01 \\ \hline
emsdz & 0 & 2,7497E+05 & 37 & 1,5000E+00 & 0 & 2,7832E+06 & 29 & 5,1400E+01 \\ \hline
f2177 & 0 & 1,2932E+07 & 2 & 2,8520E+01 & 0 & 2,5060E+07 & 2 & 1,0095E+02 \\ \hline
from-lp-file & -1 & 0,0000E+00 & 0 & 0,0000E+00 & -1 & 0,0000E+00 & 0 & 0,0000E+00 \\ \hline
gamsmod & 0 & 1,6610E+03 & 3 & 1,0000E-02 & 0 & 5,8360E+03 & 3 & 0,0000E+00 \\ \hline
iiasa & 0 & 4,5250E+03 & 15 & 3,0000E-02 & 0 & 4,6540E+03 & 15 & 4,0000E-02 \\ \hline
indata & 0 & 1,0008E+06 & 14 & 5,4600E+00 & 0 & 8,3833E+05 & 14 & 1,0150E+01 \\ \hline
jendrec1 & 6 & 0,0000E+00 & 0 & 0,0000E+00 & 6 & 0,0000E+00 & 0 & 0,0000E+00 \\ \hline
kl02,mps & 3 & 0,0000E+00 & 0 & 0,0000E+00 & 3 & 0,0000E+00 & 0 & 0,0000E+00 \\ \hline
kleemin3 & 0 & 6,0000E+00 & 5 & 0,0000E+00 & 0 & 6,0000E+00 & 5 & 0,0000E+00 \\ \hline
kleemin4 & 0 & 1,0000E+01 & 6 & 0,0000E+00 & 0 & 1,0000E+01 & 6 & 0,0000E+00 \\ \hline
kleemin5 & 0 & 1,5000E+01 & 8 & 0,0000E+00 & 0 & 1,5000E+01 & 8 & 0,0000E+00 \\ \hline
kleemin6 & 0 & 2,1000E+01 & 8 & 0,0000E+00 & 0 & 2,1000E+01 & 8 & 0,0000E+00 \\ \hline
kleemin7 & 0 & 2,8000E+01 & 8 & 0,0000E+00 & 0 & 2,8000E+01 & 8 & 0,0000E+00 \\ \hline
kleemin8 & 0 & 3,6000E+01 & 9 & 0,0000E+00 & 0 & 3,6000E+01 & 9 & 0,0000E+00 \\ \hline
l09a13l1d & 14 & 6,8450E+03 & 12 & 2,0000E-02 & 14 & 7,0910E+03 & 13 & 2,0000E-02 \\ \hline
leader & 0 & 3,9554E+05 & 49 & 2,8600E+00 & 14 & 4,7594E+06 & 48 & 5,1883E+02 \\ \hline
lindo & 0 & 2,0000E+01 & 9 & 0,0000E+00 & 0 & 2,0000E+01 & 9 & 0,0000E+00 \\ \hline
lp22 & 0 & 9,4273E+05 & 56 & 1,5690E+01 & 0 & 2,0149E+06 & 37 & 3,5400E+01 \\ \hline
lpl1 & 0 & 9,6124E+05 & 68 & 1,1760E+01 & 0 & 5,1342E+07 & 52 & 6,7786E+03 \\ \hline
lpl2 & 0 & 4,2869E+04 & 21 & 1,8000E-01 & 0 & 5,0591E+05 & 20 & 3,0700E+00 \\ \hline
lpl3 & 0 & 1,4890E+05 & 36 & 1,1200E+00 & 0 & 4,4995E+06 & 31 & 1,2124E+02 \\ \hline
model10 & 0 & 3,2144E+05 & 45 & 2,0800E+00 & 0 & 1,1938E+06 & 37 & 1,4640E+01 \\ \hline
model11 & 0 & 1,4007E+05 & 25 & 6,4000E-01 & 0 & 7,7321E+05 & 21 & 5,2500E+00 \\ \hline
model1 & 0 & 6,8610E+03 & 8 & 1,0000E-02 & 0 & 1,5093E+04 & 8 & 2,0000E-02 \\ \hline
model2 & 0 & 1,2167E+04 & 23 & 5,0000E-02 & 0 & 1,5749E+04 & 23 & 8,0000E-02 \\ \hline
model3 & 5 & 0,0000E+00 & 0 & 0,0000E+00 & 5 & 0,0000E+00 & 0 & 0,0000E+00 \\ \hline
model4 & 0 & 8,8743E+04 & 35 & 5,0000E-01 & 0 & 1,8571E+05 & 35 & 1,3200E+00 \\ \hline
model5 & 0 & 1,4052E+05 & 44 & 1,2300E+00 & 0 & 2,8807E+05 & 44 & 3,2300E+00 \\ \hline
model6 & 0 & 1,0634E+05 & 33 & 4,3000E-01 & 0 & 3,8909E+05 & 33 & 2,1400E+00 \\ \hline
model7 & 0 & 1,1000E+05 & 39 & 7,1000E-01 & 0 & 2,5601E+05 & 39 & 2,2000E+00 \\ \hline
model8 & 0 & 2,2376E+05 & 17 & 4,2000E-01 & 0 & 2,7580E+05 & 17 & 1,1400E+00 \\ \hline
model9 & 0 & 7,0839E+04 & 41 & 6,0000E-01 & 0 & 1,0504E+05 & 42 & 1,4400E+00 \\ \hline
nemsafm & 0 & 1,0160E+03 & 16 & 1,0000E-02 & 0 & 1,3240E+03 & 16 & 2,0000E-02 \\ \hline
nemscem & 0 & 2,5350E+03 & 18 & 2,0000E-02 & 0 & 1,4749E+04 & 18 & 3,0000E-02 \\ \hline
nemsemm1 & 0 & 2,4298E+05 & 57 & 1,5710E+01 & 0 & 3,9867E+05 & 57 & 1,9620E+01 \\ \hline
nemsemm2 & 0 & 6,2496E+04 & 35 & 8,7000E-01 & 0 & 3,8140E+05 & 35 & 3,3700E+00 \\ \hline
nemspmm1 & 0 & 1,1464E+05 & 39 & 7,4000E-01 & 0 & 8,6560E+05 & 31 & 9,0400E+00 \\ \hline
nemspmm2,in & 0 & 1,1890E+05 & 44 & 1,0100E+00 & 0 & 7,1490E+05 & 34 & 7,4900E+00 \\ \hline
nemswrld & 0 & 7,8312E+05 & 42 & 7,8000E+00 & 0 & 4,4720E+06 & 38 & 9,3030E+01 \\ \hline
nsct1 & 0 & 5,4180E+06 & 21 & 1,7080E+02 & 0 & 8,5935E+06 & 22 & 6,1142E+02 \\ \hline
nsct2 & 0 & 5,6456E+06 & 28 & 2,3305E+02 & 0 & 8,3502E+06 & 30 & 7,0599E+02 \\ \hline
nsic1 & 0 & 5,0420E+03 & 10 & 2,0000E-02 & 0 & 7,2400E+03 & 10 & 3,0000E-02 \\ \hline
nsic2 & 0 & 5,5160E+03 & 14 & 2,0000E-02 & 0 & 8,0110E+03 & 14 & 4,0000E-02 \\ \hline
nsir1 & 0 & 3,0346E+05 & 21 & 3,6600E+00 & 0 & 6,3039E+05 & 17 & 1,0390E+01 \\ \hline
nsir2 & 0 & 3,2795E+05 & 30 & 5,9400E+00 & 0 & 6,5743E+05 & 26 & 1,4420E+01 \\ \hline
nw14,mps & 3 & 0,0000E+00 & 0 & 0,0000E+00 & 3 & 0,0000E+00 & 0 & 0,0000E+00 \\ \hline
olivier & 14 & 4,6137E+05 & 50 & 4,5700E+00 & 14 & 5,6194E+06 & 43 & 4,4814E+02 \\ \hline
or4,mps & 3 & 0,0000E+00 & 0 & 0,0000E+00 & 3 & 0,0000E+00 & 0 & 0,0000E+00 \\ \hline
orna1,mps & 3 & 0,0000E+00 & 0 & 0,0000E+00 & 3 & 0,0000E+00 & 0 & 0,0000E+00 \\ \hline
orna2,mps & 3 & 0,0000E+00 & 0 & 0,0000E+00 & 3 & 0,0000E+00 & 0 & 0,0000E+00 \\ \hline
orna3,mps & 3 & 0,0000E+00 & 0 & 0,0000E+00 & 3 & 0,0000E+00 & 0 & 0,0000E+00 \\ \hline
orna7,mps & 3 & 0,0000E+00 & 0 & 0,0000E+00 & 3 & 0,0000E+00 & 0 & 0,0000E+00 \\ \hline
orswq2,mps & 3 & 0,0000E+00 & 0 & 0,0000E+00 & 3 & 0,0000E+00 & 0 & 0,0000E+00 \\ \hline
p0033 & 0 & 5,8000E+01 & 8 & 0,0000E+00 & 0 & 6,0000E+01 & 8 & 0,0000E+00 \\ \hline
p0040 & 0 & 9,6000E+01 & 9 & 0,0000E+00 & 0 & 9,6000E+01 & 9 & 0,0000E+00 \\ \hline
p0201 & 0 & 2,9220E+03 & 5 & 1,0000E-02 & 0 & 7,8160E+03 & 5 & 1,0000E-02 \\ \hline
p0282 & 0 & 9,5870E+03 & 13 & 2,0000E-02 & 0 & 1,3065E+04 & 13 & 4,0000E-02 \\ \hline
p0291 & 0 & 4,8890E+03 & 12 & 1,0000E-02 & 0 & 8,7050E+03 & 12 & 2,0000E-02 \\ \hline
p0548 & 0 & 1,0530E+03 & 24 & 2,0000E-02 & 0 & 1,1890E+03 & 24 & 2,0000E-02 \\ \hline
p05 & 0 & 2,3647E+05 & 23 & 7,1000E-01 & 0 & 2,2887E+06 & 19 & 2,3310E+01 \\ \hline
p10 & 0 & 4,6962E+05 & 27 & 1,6900E+00 & 0 & 7,7808E+06 & 29 & 2,5367E+02 \\ \hline
p12345 & 0 & 6,4360E+03 & 14 & 8,0000E-02 & 0 & 6,4360E+03 & 14 & 8,0000E-02 \\ \hline
p19328 & 0 & 1,6932E+04 & 15 & 2,7000E-01 & 0 & 1,6932E+04 & 15 & 3,1000E-01 \\ \hline
p19 & 0 & 2,9710E+03 & 19 & 2,0000E-02 & 0 & 9,8150E+03 & 19 & 3,0000E-02 \\ \hline
p2756 & 0 & 6,5240E+03 & 15 & 5,0000E-02 & 0 & 6,7656E+04 & 15 & 2,0000E-01 \\ \hline
pcb1000 & 0 & 2,9753E+04 & 23 & 1,8000E-01 & 0 & 3,9440E+05 & 23 & 1,8200E+00 \\ \hline
pcb3000 & 0 & 1,0109E+05 & 25 & 5,6000E-01 & 0 & 4,3551E+06 & 19 & 4,9320E+01 \\ \hline
primagaz & 0 & 1,2404E+04 & 13 & 1,1000E-01 & 0 & 1,2404E+04 & 13 & 1,4000E-01 \\ \hline
progas & 0 & 3,0946E+04 & 15 & 6,0000E-02 & 0 & 4,8144E+04 & 15 & 2,3000E-01 \\ \hline
ps & 15 & 1,1054E+07 & 40 & 5,1600E+02 & 15 & 1,2683E+07 & 48 & 6,2349E+02 \\ \hline
qiu & 0 & 4,6492E+04 & 5 & 2,0000E-02 & 0 & 1,0207E+05 & 5 & 8,0000E-02 \\ \hline
r05 & 0 & 4,3308E+05 & 23 & 1,7800E+00 & 0 & 2,1250E+06 & 18 & 2,6230E+01 \\ \hline
radio & -1 & 0,0000E+00 & 0 & 0,0000E+00 & -11 & 0,0000E+00 & 0 & 0,0000E+00 \\ \hline
rlf,pre,dual & 0 & 8,4078E+05 & 13 & 2,3400E+00 & 0 & 6,0951E+06 & 13 & 4,8841E+02 \\ \hline
routing & 0 & 3,1143E+06 & 20 & 6,0600E+00 & 0 & 1,2819E+07 & 17 & 8,7570E+01 \\ \hline
sc205 & 12 & 5,0422E+04 & 15 & 5,3000E-01 & 12 & 5,3625E+04 & 15 & 6,0000E-01 \\ \hline
seymour & 0 & 4,5141E+06 & 15 & 3,6330E+01 & 0 & 8,2165E+06 & 15 & 1,3423E+02 \\ \hline
slp-tsk & 0 & 4,0878E+06 & 19 & 6,4860E+01 & 0 & 4,0918E+06 & 19 & 1,2514E+02 \\ \hline
southern1 & 0 & 5,6710E+06 & 15 & 6,5000E+01 & 0 & 7,7658E+06 & 15 & 6,6628E+02 \\ \hline
sturing & 0 & 1,8861E+04 & 36 & 1,5000E-01 & 0 & 3,6899E+04 & 36 & 3,0000E-01 \\ \hline
sws & 0 & 5,8235E+04 & 10 & 2,1000E-01 & 0 & 4,6605E+05 & 10 & 5,9000E-01 \\ \hline
t0331-4l & 0 & 2,0167E+05 & 37 & 6,5800E+00 & 0 & 2,1556E+05 & 37 & 7,0600E+00 \\ \hline
test & 0 & 6,5540E+03 & 46 & 2,7000E-01 & 0 & 6,5540E+03 & 46 & 2,9000E-01 \\ \hline
testps,mod & 0 & 3,1000E+01 & 13 & 0,0000E+00 & 0 & 3,1000E+01 & 13 & 0,0000E+00 \\ \hline
unilever2 & 15 & 4,2276E+05 & 36 & 3,4700E+00 & 0 & 5,8050E+06 & 73 & 2,4664E+02 \\ \hline
us04,mps & 3 & 0,0000E+00 & 0 & 0,0000E+00 & 3 & 0,0000E+00 & 0 & 0,0000E+00 \\ \hline
world & 14 & 1,1112E+06 & 60 & 9,2800E+00 & -1 & 0,0000E+00 & 0 & 0,0000E+00 \\ \hline
zed & 0 & 3,3800E+03 & 9 & 2,0000E-02 & 0 & 3,9520E+03 & 9 & 2,0000E-02 \\ \hline
zz & 15 & 2,8570E+05 & 43 & 3,7000E+00 & 15 & 9,9260E+05 & 43 & 1,0700E+01 \\ \hline
\end{tabular}

\end{table}
\begin{table}
  \centering
  \caption{Resultados experimentais da biblioteca Meszaros (2/3).}
  \label{tab:resulmes1}
  \begin{tabular}{|l|r|r|r|r|r|r|r|r|}
\hline
\multicolumn{1}{|c|}{Problema} & \multicolumn{4}{|c|}{MMD} &         \multicolumn{4}{|c|}{RCM} \\ \hline
\multicolumn{1}{|c|}{Nome} & \multicolumn{1}{|c|}{R} &
        \multicolumn{1}{|c|}{NNZ} & \multicolumn{1}{|c|}{IT} &
        \multicolumn{1}{|c|}{T} & \multicolumn{1}{|c|}{R} &
        \multicolumn{1}{|c|}{NNZ} & \multicolumn{1}{|c|}{IT} &
        \multicolumn{1}{|c|}{T} \\ \hline
3-mix & 0 & 5,4100E+02 & 16 & 1,0000E-02 & 0 & 6,1300E+02 & 16 & 1,0000E-02 \\ \hline
aa01,mps & 3 & 0,0000E+00 & 0 & 0,0000E+00 & 3 & 0,0000E+00 & 0 & 0,0000E+00 \\ \hline
aa03,mps & 3 & 0,0000E+00 & 0 & 0,0000E+00 & 3 & 0,0000E+00 & 0 & 0,0000E+00 \\ \hline
air02 & 0 & 1,1800E+03 & 18 & 2,7000E-01 & 0 & 1,1900E+03 & 18 & 3,0000E-01 \\ \hline
air03 & 0 & 5,3030E+03 & 34 & 7,1000E-01 & 0 & 5,8610E+03 & 34 & 7,2000E-01 \\ \hline
air04 & 0 & 2,0654E+05 & 76 & 3,7700E+00 & 0 & 2,5434E+05 & 96 & 7,0100E+00 \\ \hline
air05 & 0 & 6,2864E+04 & 33 & 6,1000E-01 & 0 & 7,1846E+04 & 25 & 5,5000E-01 \\ \hline
air06 & 15 & 1,8586E+05 & 53 & 2,3200E+00 & 0 & 2,4812E+05 & 28 & 2,1600E+00 \\ \hline
aircraft & 14 & 3,7540E+03 & 10 & 1,2000E-01 & 14 & 3,7540E+03 & 10 & 1,8000E-01 \\ \hline
aramco & 0 & 9,5000E+01 & 9 & 0,0000E+00 & 0 & 1,1800E+02 & 9 & 0,0000E+00 \\ \hline
bas1lp & 0 & 2,2168E+06 & 9 & 1,8540E+01 & 0 & 3,2825E+06 & 9 & 5,1880E+01 \\ \hline
baxter-mat & 0 & 5,8934E+06 & 30 & 3,9960E+01 & 0 & 2,9029E+07 & 27 & 1,0537E+03 \\ \hline
bpmpd & 6 & 0,0000E+00 & 0 & 0,0000E+00 & 6 & 0,0000E+00 & 0 & 0,0000E+00 \\ \hline
complex1 & 14 & 4,7140E+05 & 16 & 4,0300E+00 & 15 & 4,9191E+05 & 37 & 1,0140E+01 \\ \hline
cr42 & 14 & 1,8040E+03 & 15 & 4,0000E-02 & 14 & 1,8040E+03 & 15 & 4,0000E-02 \\ \hline
crew & 0 & 8,5880E+03 & 12 & 1,6000E-01 & 0 & 8,8230E+03 & 12 & 1,6000E-01 \\ \hline
dano3mip & 12 & 1,0902E+06 & 34 & 1,3600E+01 & 12 & 3,4466E+06 & 32 & 7,8680E+01 \\ \hline
dbah00 & 0 & 2,8099E+05 & 43 & 1,2400E+00 & 0 & 4,8499E+06 & 32 & 1,0051E+02 \\ \hline
dbic1 & 0 & 1,8631E+06 & 45 & 2,8700E+01 & -1 & 0,0000E+00 & 0 & 0,0000E+00 \\ \hline
dbir1 & 0 & 2,5235E+06 & 24 & 6,8810E+01 & 0 & 5,7974E+06 & 23 & 2,8589E+02 \\ \hline
dbir2 & 0 & 2,8185E+06 & 26 & 8,4010E+01 & 0 & 6,7499E+06 & 26 & 3,9022E+02 \\ \hline
delfland & 0 & 3,4610E+03 & 46 & 4,0000E-02 & 0 & 1,2234E+04 & 46 & 8,0000E-02 \\ \hline
disp3,mps & 3 & 0,0000E+00 & 0 & 0,0000E+00 & 3 & 0,0000E+00 & 0 & 0,0000E+00 \\ \hline
ds90 & 0 & 9,9700E+04 & 28 & 3,7000E-01 & 0 & 1,1216E+06 & 21 & 9,9200E+00 \\ \hline
dsbmip & 0 & 1,6789E+04 & 48 & 1,6000E-01 & 0 & 5,9151E+04 & 48 & 4,3000E-01 \\ \hline
emsdz & 0 & 2,7497E+05 & 37 & 1,5000E+00 & 0 & 2,7832E+06 & 29 & 5,1400E+01 \\ \hline
f2177 & 0 & 1,2932E+07 & 2 & 2,8520E+01 & 0 & 2,5060E+07 & 2 & 1,0095E+02 \\ \hline
from-lp-file & -1 & 0,0000E+00 & 0 & 0,0000E+00 & -1 & 0,0000E+00 & 0 & 0,0000E+00 \\ \hline
gamsmod & 0 & 1,6610E+03 & 3 & 1,0000E-02 & 0 & 5,8360E+03 & 3 & 0,0000E+00 \\ \hline
iiasa & 0 & 4,5250E+03 & 15 & 3,0000E-02 & 0 & 4,6540E+03 & 15 & 4,0000E-02 \\ \hline
indata & 0 & 1,0008E+06 & 14 & 5,4600E+00 & 0 & 8,3833E+05 & 14 & 1,0150E+01 \\ \hline
jendrec1 & 6 & 0,0000E+00 & 0 & 0,0000E+00 & 6 & 0,0000E+00 & 0 & 0,0000E+00 \\ \hline
kl02,mps & 3 & 0,0000E+00 & 0 & 0,0000E+00 & 3 & 0,0000E+00 & 0 & 0,0000E+00 \\ \hline
kleemin3 & 0 & 6,0000E+00 & 5 & 0,0000E+00 & 0 & 6,0000E+00 & 5 & 0,0000E+00 \\ \hline
kleemin4 & 0 & 1,0000E+01 & 6 & 0,0000E+00 & 0 & 1,0000E+01 & 6 & 0,0000E+00 \\ \hline
kleemin5 & 0 & 1,5000E+01 & 8 & 0,0000E+00 & 0 & 1,5000E+01 & 8 & 0,0000E+00 \\ \hline
kleemin6 & 0 & 2,1000E+01 & 8 & 0,0000E+00 & 0 & 2,1000E+01 & 8 & 0,0000E+00 \\ \hline
kleemin7 & 0 & 2,8000E+01 & 8 & 0,0000E+00 & 0 & 2,8000E+01 & 8 & 0,0000E+00 \\ \hline
kleemin8 & 0 & 3,6000E+01 & 9 & 0,0000E+00 & 0 & 3,6000E+01 & 9 & 0,0000E+00 \\ \hline
l09a13l1d & 14 & 6,8450E+03 & 12 & 2,0000E-02 & 14 & 7,0910E+03 & 13 & 2,0000E-02 \\ \hline
leader & 0 & 3,9554E+05 & 49 & 2,8600E+00 & 14 & 4,7594E+06 & 48 & 5,1883E+02 \\ \hline
lindo & 0 & 2,0000E+01 & 9 & 0,0000E+00 & 0 & 2,0000E+01 & 9 & 0,0000E+00 \\ \hline
lp22 & 0 & 9,4273E+05 & 56 & 1,5690E+01 & 0 & 2,0149E+06 & 37 & 3,5400E+01 \\ \hline
lpl1 & 0 & 9,6124E+05 & 68 & 1,1760E+01 & 0 & 5,1342E+07 & 52 & 6,7786E+03 \\ \hline
lpl2 & 0 & 4,2869E+04 & 21 & 1,8000E-01 & 0 & 5,0591E+05 & 20 & 3,0700E+00 \\ \hline
lpl3 & 0 & 1,4890E+05 & 36 & 1,1200E+00 & 0 & 4,4995E+06 & 31 & 1,2124E+02 \\ \hline
model10 & 0 & 3,2144E+05 & 45 & 2,0800E+00 & 0 & 1,1938E+06 & 37 & 1,4640E+01 \\ \hline
model11 & 0 & 1,4007E+05 & 25 & 6,4000E-01 & 0 & 7,7321E+05 & 21 & 5,2500E+00 \\ \hline
model1 & 0 & 6,8610E+03 & 8 & 1,0000E-02 & 0 & 1,5093E+04 & 8 & 2,0000E-02 \\ \hline
model2 & 0 & 1,2167E+04 & 23 & 5,0000E-02 & 0 & 1,5749E+04 & 23 & 8,0000E-02 \\ \hline
model3 & 5 & 0,0000E+00 & 0 & 0,0000E+00 & 5 & 0,0000E+00 & 0 & 0,0000E+00 \\ \hline
model4 & 0 & 8,8743E+04 & 35 & 5,0000E-01 & 0 & 1,8571E+05 & 35 & 1,3200E+00 \\ \hline
model5 & 0 & 1,4052E+05 & 44 & 1,2300E+00 & 0 & 2,8807E+05 & 44 & 3,2300E+00 \\ \hline
model6 & 0 & 1,0634E+05 & 33 & 4,3000E-01 & 0 & 3,8909E+05 & 33 & 2,1400E+00 \\ \hline
model7 & 0 & 1,1000E+05 & 39 & 7,1000E-01 & 0 & 2,5601E+05 & 39 & 2,2000E+00 \\ \hline
model8 & 0 & 2,2376E+05 & 17 & 4,2000E-01 & 0 & 2,7580E+05 & 17 & 1,1400E+00 \\ \hline
model9 & 0 & 7,0839E+04 & 41 & 6,0000E-01 & 0 & 1,0504E+05 & 42 & 1,4400E+00 \\ \hline
nemsafm & 0 & 1,0160E+03 & 16 & 1,0000E-02 & 0 & 1,3240E+03 & 16 & 2,0000E-02 \\ \hline
nemscem & 0 & 2,5350E+03 & 18 & 2,0000E-02 & 0 & 1,4749E+04 & 18 & 3,0000E-02 \\ \hline
nemsemm1 & 0 & 2,4298E+05 & 57 & 1,5710E+01 & 0 & 3,9867E+05 & 57 & 1,9620E+01 \\ \hline
nemsemm2 & 0 & 6,2496E+04 & 35 & 8,7000E-01 & 0 & 3,8140E+05 & 35 & 3,3700E+00 \\ \hline
nemspmm1 & 0 & 1,1464E+05 & 39 & 7,4000E-01 & 0 & 8,6560E+05 & 31 & 9,0400E+00 \\ \hline
nemspmm2,in & 0 & 1,1890E+05 & 44 & 1,0100E+00 & 0 & 7,1490E+05 & 34 & 7,4900E+00 \\ \hline
nemswrld & 0 & 7,8312E+05 & 42 & 7,8000E+00 & 0 & 4,4720E+06 & 38 & 9,3030E+01 \\ \hline
nsct1 & 0 & 5,4180E+06 & 21 & 1,7080E+02 & 0 & 8,5935E+06 & 22 & 6,1142E+02 \\ \hline
nsct2 & 0 & 5,6456E+06 & 28 & 2,3305E+02 & 0 & 8,3502E+06 & 30 & 7,0599E+02 \\ \hline
nsic1 & 0 & 5,0420E+03 & 10 & 2,0000E-02 & 0 & 7,2400E+03 & 10 & 3,0000E-02 \\ \hline
nsic2 & 0 & 5,5160E+03 & 14 & 2,0000E-02 & 0 & 8,0110E+03 & 14 & 4,0000E-02 \\ \hline
nsir1 & 0 & 3,0346E+05 & 21 & 3,6600E+00 & 0 & 6,3039E+05 & 17 & 1,0390E+01 \\ \hline
nsir2 & 0 & 3,2795E+05 & 30 & 5,9400E+00 & 0 & 6,5743E+05 & 26 & 1,4420E+01 \\ \hline
nw14,mps & 3 & 0,0000E+00 & 0 & 0,0000E+00 & 3 & 0,0000E+00 & 0 & 0,0000E+00 \\ \hline
olivier & 14 & 4,6137E+05 & 50 & 4,5700E+00 & 14 & 5,6194E+06 & 43 & 4,4814E+02 \\ \hline
or4,mps & 3 & 0,0000E+00 & 0 & 0,0000E+00 & 3 & 0,0000E+00 & 0 & 0,0000E+00 \\ \hline
orna1,mps & 3 & 0,0000E+00 & 0 & 0,0000E+00 & 3 & 0,0000E+00 & 0 & 0,0000E+00 \\ \hline
orna2,mps & 3 & 0,0000E+00 & 0 & 0,0000E+00 & 3 & 0,0000E+00 & 0 & 0,0000E+00 \\ \hline
orna3,mps & 3 & 0,0000E+00 & 0 & 0,0000E+00 & 3 & 0,0000E+00 & 0 & 0,0000E+00 \\ \hline
orna7,mps & 3 & 0,0000E+00 & 0 & 0,0000E+00 & 3 & 0,0000E+00 & 0 & 0,0000E+00 \\ \hline
orswq2,mps & 3 & 0,0000E+00 & 0 & 0,0000E+00 & 3 & 0,0000E+00 & 0 & 0,0000E+00 \\ \hline
p0033 & 0 & 5,8000E+01 & 8 & 0,0000E+00 & 0 & 6,0000E+01 & 8 & 0,0000E+00 \\ \hline
p0040 & 0 & 9,6000E+01 & 9 & 0,0000E+00 & 0 & 9,6000E+01 & 9 & 0,0000E+00 \\ \hline
p0201 & 0 & 2,9220E+03 & 5 & 1,0000E-02 & 0 & 7,8160E+03 & 5 & 1,0000E-02 \\ \hline
p0282 & 0 & 9,5870E+03 & 13 & 2,0000E-02 & 0 & 1,3065E+04 & 13 & 4,0000E-02 \\ \hline
p0291 & 0 & 4,8890E+03 & 12 & 1,0000E-02 & 0 & 8,7050E+03 & 12 & 2,0000E-02 \\ \hline
p0548 & 0 & 1,0530E+03 & 24 & 2,0000E-02 & 0 & 1,1890E+03 & 24 & 2,0000E-02 \\ \hline
p05 & 0 & 2,3647E+05 & 23 & 7,1000E-01 & 0 & 2,2887E+06 & 19 & 2,3310E+01 \\ \hline
p10 & 0 & 4,6962E+05 & 27 & 1,6900E+00 & 0 & 7,7808E+06 & 29 & 2,5367E+02 \\ \hline
p12345 & 0 & 6,4360E+03 & 14 & 8,0000E-02 & 0 & 6,4360E+03 & 14 & 8,0000E-02 \\ \hline
p19328 & 0 & 1,6932E+04 & 15 & 2,7000E-01 & 0 & 1,6932E+04 & 15 & 3,1000E-01 \\ \hline
p19 & 0 & 2,9710E+03 & 19 & 2,0000E-02 & 0 & 9,8150E+03 & 19 & 3,0000E-02 \\ \hline
p2756 & 0 & 6,5240E+03 & 15 & 5,0000E-02 & 0 & 6,7656E+04 & 15 & 2,0000E-01 \\ \hline
pcb1000 & 0 & 2,9753E+04 & 23 & 1,8000E-01 & 0 & 3,9440E+05 & 23 & 1,8200E+00 \\ \hline
pcb3000 & 0 & 1,0109E+05 & 25 & 5,6000E-01 & 0 & 4,3551E+06 & 19 & 4,9320E+01 \\ \hline
primagaz & 0 & 1,2404E+04 & 13 & 1,1000E-01 & 0 & 1,2404E+04 & 13 & 1,4000E-01 \\ \hline
progas & 0 & 3,0946E+04 & 15 & 6,0000E-02 & 0 & 4,8144E+04 & 15 & 2,3000E-01 \\ \hline
ps & 15 & 1,1054E+07 & 40 & 5,1600E+02 & 15 & 1,2683E+07 & 48 & 6,2349E+02 \\ \hline
qiu & 0 & 4,6492E+04 & 5 & 2,0000E-02 & 0 & 1,0207E+05 & 5 & 8,0000E-02 \\ \hline
r05 & 0 & 4,3308E+05 & 23 & 1,7800E+00 & 0 & 2,1250E+06 & 18 & 2,6230E+01 \\ \hline
radio & -1 & 0,0000E+00 & 0 & 0,0000E+00 & -11 & 0,0000E+00 & 0 & 0,0000E+00 \\ \hline
rlf,pre,dual & 0 & 8,4078E+05 & 13 & 2,3400E+00 & 0 & 6,0951E+06 & 13 & 4,8841E+02 \\ \hline
routing & 0 & 3,1143E+06 & 20 & 6,0600E+00 & 0 & 1,2819E+07 & 17 & 8,7570E+01 \\ \hline
sc205 & 12 & 5,0422E+04 & 15 & 5,3000E-01 & 12 & 5,3625E+04 & 15 & 6,0000E-01 \\ \hline
seymour & 0 & 4,5141E+06 & 15 & 3,6330E+01 & 0 & 8,2165E+06 & 15 & 1,3423E+02 \\ \hline
slp-tsk & 0 & 4,0878E+06 & 19 & 6,4860E+01 & 0 & 4,0918E+06 & 19 & 1,2514E+02 \\ \hline
southern1 & 0 & 5,6710E+06 & 15 & 6,5000E+01 & 0 & 7,7658E+06 & 15 & 6,6628E+02 \\ \hline
sturing & 0 & 1,8861E+04 & 36 & 1,5000E-01 & 0 & 3,6899E+04 & 36 & 3,0000E-01 \\ \hline
sws & 0 & 5,8235E+04 & 10 & 2,1000E-01 & 0 & 4,6605E+05 & 10 & 5,9000E-01 \\ \hline
t0331-4l & 0 & 2,0167E+05 & 37 & 6,5800E+00 & 0 & 2,1556E+05 & 37 & 7,0600E+00 \\ \hline
test & 0 & 6,5540E+03 & 46 & 2,7000E-01 & 0 & 6,5540E+03 & 46 & 2,9000E-01 \\ \hline
testps,mod & 0 & 3,1000E+01 & 13 & 0,0000E+00 & 0 & 3,1000E+01 & 13 & 0,0000E+00 \\ \hline
unilever2 & 15 & 4,2276E+05 & 36 & 3,4700E+00 & 0 & 5,8050E+06 & 73 & 2,4664E+02 \\ \hline
us04,mps & 3 & 0,0000E+00 & 0 & 0,0000E+00 & 3 & 0,0000E+00 & 0 & 0,0000E+00 \\ \hline
world & 14 & 1,1112E+06 & 60 & 9,2800E+00 & -1 & 0,0000E+00 & 0 & 0,0000E+00 \\ \hline
zed & 0 & 3,3800E+03 & 9 & 2,0000E-02 & 0 & 3,9520E+03 & 9 & 2,0000E-02 \\ \hline
zz & 15 & 2,8570E+05 & 43 & 3,7000E+00 & 15 & 9,9260E+05 & 43 & 1,0700E+01 \\ \hline
\end{tabular}

\end{table}
\begin{table}
  \centering
  \caption{Resultados experimentais da biblioteca Meszaros (3/3).}
  \label{tab:resulmes2}
  \begin{tabular}{|l|r|r|r|r|r|r|r|r|}
\hline
\multicolumn{1}{|c|}{Problema} & \multicolumn{4}{|c|}{MMD} &         \multicolumn{4}{|c|}{RCM} \\ \hline
\multicolumn{1}{|c|}{Nome} & \multicolumn{1}{|c|}{R} &
        \multicolumn{1}{|c|}{NNZ} & \multicolumn{1}{|c|}{IT} &
        \multicolumn{1}{|c|}{T} & \multicolumn{1}{|c|}{R} &
        \multicolumn{1}{|c|}{NNZ} & \multicolumn{1}{|c|}{IT} &
        \multicolumn{1}{|c|}{T} \\ \hline
3-mix & 0 & 5,4100E+02 & 16 & 1,0000E-02 & 0 & 6,1300E+02 & 16 & 1,0000E-02 \\ \hline
aa01,mps & 3 & 0,0000E+00 & 0 & 0,0000E+00 & 3 & 0,0000E+00 & 0 & 0,0000E+00 \\ \hline
aa03,mps & 3 & 0,0000E+00 & 0 & 0,0000E+00 & 3 & 0,0000E+00 & 0 & 0,0000E+00 \\ \hline
air02 & 0 & 1,1800E+03 & 18 & 2,7000E-01 & 0 & 1,1900E+03 & 18 & 3,0000E-01 \\ \hline
air03 & 0 & 5,3030E+03 & 34 & 7,1000E-01 & 0 & 5,8610E+03 & 34 & 7,2000E-01 \\ \hline
air04 & 0 & 2,0654E+05 & 76 & 3,7700E+00 & 0 & 2,5434E+05 & 96 & 7,0100E+00 \\ \hline
air05 & 0 & 6,2864E+04 & 33 & 6,1000E-01 & 0 & 7,1846E+04 & 25 & 5,5000E-01 \\ \hline
air06 & 15 & 1,8586E+05 & 53 & 2,3200E+00 & 0 & 2,4812E+05 & 28 & 2,1600E+00 \\ \hline
aircraft & 14 & 3,7540E+03 & 10 & 1,2000E-01 & 14 & 3,7540E+03 & 10 & 1,8000E-01 \\ \hline
aramco & 0 & 9,5000E+01 & 9 & 0,0000E+00 & 0 & 1,1800E+02 & 9 & 0,0000E+00 \\ \hline
bas1lp & 0 & 2,2168E+06 & 9 & 1,8540E+01 & 0 & 3,2825E+06 & 9 & 5,1880E+01 \\ \hline
baxter-mat & 0 & 5,8934E+06 & 30 & 3,9960E+01 & 0 & 2,9029E+07 & 27 & 1,0537E+03 \\ \hline
bpmpd & 6 & 0,0000E+00 & 0 & 0,0000E+00 & 6 & 0,0000E+00 & 0 & 0,0000E+00 \\ \hline
complex1 & 14 & 4,7140E+05 & 16 & 4,0300E+00 & 15 & 4,9191E+05 & 37 & 1,0140E+01 \\ \hline
cr42 & 14 & 1,8040E+03 & 15 & 4,0000E-02 & 14 & 1,8040E+03 & 15 & 4,0000E-02 \\ \hline
crew & 0 & 8,5880E+03 & 12 & 1,6000E-01 & 0 & 8,8230E+03 & 12 & 1,6000E-01 \\ \hline
dano3mip & 12 & 1,0902E+06 & 34 & 1,3600E+01 & 12 & 3,4466E+06 & 32 & 7,8680E+01 \\ \hline
dbah00 & 0 & 2,8099E+05 & 43 & 1,2400E+00 & 0 & 4,8499E+06 & 32 & 1,0051E+02 \\ \hline
dbic1 & 0 & 1,8631E+06 & 45 & 2,8700E+01 & -1 & 0,0000E+00 & 0 & 0,0000E+00 \\ \hline
dbir1 & 0 & 2,5235E+06 & 24 & 6,8810E+01 & 0 & 5,7974E+06 & 23 & 2,8589E+02 \\ \hline
dbir2 & 0 & 2,8185E+06 & 26 & 8,4010E+01 & 0 & 6,7499E+06 & 26 & 3,9022E+02 \\ \hline
delfland & 0 & 3,4610E+03 & 46 & 4,0000E-02 & 0 & 1,2234E+04 & 46 & 8,0000E-02 \\ \hline
disp3,mps & 3 & 0,0000E+00 & 0 & 0,0000E+00 & 3 & 0,0000E+00 & 0 & 0,0000E+00 \\ \hline
ds90 & 0 & 9,9700E+04 & 28 & 3,7000E-01 & 0 & 1,1216E+06 & 21 & 9,9200E+00 \\ \hline
dsbmip & 0 & 1,6789E+04 & 48 & 1,6000E-01 & 0 & 5,9151E+04 & 48 & 4,3000E-01 \\ \hline
emsdz & 0 & 2,7497E+05 & 37 & 1,5000E+00 & 0 & 2,7832E+06 & 29 & 5,1400E+01 \\ \hline
f2177 & 0 & 1,2932E+07 & 2 & 2,8520E+01 & 0 & 2,5060E+07 & 2 & 1,0095E+02 \\ \hline
from-lp-file & -1 & 0,0000E+00 & 0 & 0,0000E+00 & -1 & 0,0000E+00 & 0 & 0,0000E+00 \\ \hline
gamsmod & 0 & 1,6610E+03 & 3 & 1,0000E-02 & 0 & 5,8360E+03 & 3 & 0,0000E+00 \\ \hline
iiasa & 0 & 4,5250E+03 & 15 & 3,0000E-02 & 0 & 4,6540E+03 & 15 & 4,0000E-02 \\ \hline
indata & 0 & 1,0008E+06 & 14 & 5,4600E+00 & 0 & 8,3833E+05 & 14 & 1,0150E+01 \\ \hline
jendrec1 & 6 & 0,0000E+00 & 0 & 0,0000E+00 & 6 & 0,0000E+00 & 0 & 0,0000E+00 \\ \hline
kl02,mps & 3 & 0,0000E+00 & 0 & 0,0000E+00 & 3 & 0,0000E+00 & 0 & 0,0000E+00 \\ \hline
kleemin3 & 0 & 6,0000E+00 & 5 & 0,0000E+00 & 0 & 6,0000E+00 & 5 & 0,0000E+00 \\ \hline
kleemin4 & 0 & 1,0000E+01 & 6 & 0,0000E+00 & 0 & 1,0000E+01 & 6 & 0,0000E+00 \\ \hline
kleemin5 & 0 & 1,5000E+01 & 8 & 0,0000E+00 & 0 & 1,5000E+01 & 8 & 0,0000E+00 \\ \hline
kleemin6 & 0 & 2,1000E+01 & 8 & 0,0000E+00 & 0 & 2,1000E+01 & 8 & 0,0000E+00 \\ \hline
kleemin7 & 0 & 2,8000E+01 & 8 & 0,0000E+00 & 0 & 2,8000E+01 & 8 & 0,0000E+00 \\ \hline
kleemin8 & 0 & 3,6000E+01 & 9 & 0,0000E+00 & 0 & 3,6000E+01 & 9 & 0,0000E+00 \\ \hline
l09a13l1d & 14 & 6,8450E+03 & 12 & 2,0000E-02 & 14 & 7,0910E+03 & 13 & 2,0000E-02 \\ \hline
leader & 0 & 3,9554E+05 & 49 & 2,8600E+00 & 14 & 4,7594E+06 & 48 & 5,1883E+02 \\ \hline
lindo & 0 & 2,0000E+01 & 9 & 0,0000E+00 & 0 & 2,0000E+01 & 9 & 0,0000E+00 \\ \hline
lp22 & 0 & 9,4273E+05 & 56 & 1,5690E+01 & 0 & 2,0149E+06 & 37 & 3,5400E+01 \\ \hline
lpl1 & 0 & 9,6124E+05 & 68 & 1,1760E+01 & 0 & 5,1342E+07 & 52 & 6,7786E+03 \\ \hline
lpl2 & 0 & 4,2869E+04 & 21 & 1,8000E-01 & 0 & 5,0591E+05 & 20 & 3,0700E+00 \\ \hline
lpl3 & 0 & 1,4890E+05 & 36 & 1,1200E+00 & 0 & 4,4995E+06 & 31 & 1,2124E+02 \\ \hline
model10 & 0 & 3,2144E+05 & 45 & 2,0800E+00 & 0 & 1,1938E+06 & 37 & 1,4640E+01 \\ \hline
model11 & 0 & 1,4007E+05 & 25 & 6,4000E-01 & 0 & 7,7321E+05 & 21 & 5,2500E+00 \\ \hline
model1 & 0 & 6,8610E+03 & 8 & 1,0000E-02 & 0 & 1,5093E+04 & 8 & 2,0000E-02 \\ \hline
model2 & 0 & 1,2167E+04 & 23 & 5,0000E-02 & 0 & 1,5749E+04 & 23 & 8,0000E-02 \\ \hline
model3 & 5 & 0,0000E+00 & 0 & 0,0000E+00 & 5 & 0,0000E+00 & 0 & 0,0000E+00 \\ \hline
model4 & 0 & 8,8743E+04 & 35 & 5,0000E-01 & 0 & 1,8571E+05 & 35 & 1,3200E+00 \\ \hline
model5 & 0 & 1,4052E+05 & 44 & 1,2300E+00 & 0 & 2,8807E+05 & 44 & 3,2300E+00 \\ \hline
model6 & 0 & 1,0634E+05 & 33 & 4,3000E-01 & 0 & 3,8909E+05 & 33 & 2,1400E+00 \\ \hline
model7 & 0 & 1,1000E+05 & 39 & 7,1000E-01 & 0 & 2,5601E+05 & 39 & 2,2000E+00 \\ \hline
model8 & 0 & 2,2376E+05 & 17 & 4,2000E-01 & 0 & 2,7580E+05 & 17 & 1,1400E+00 \\ \hline
model9 & 0 & 7,0839E+04 & 41 & 6,0000E-01 & 0 & 1,0504E+05 & 42 & 1,4400E+00 \\ \hline
nemsafm & 0 & 1,0160E+03 & 16 & 1,0000E-02 & 0 & 1,3240E+03 & 16 & 2,0000E-02 \\ \hline
nemscem & 0 & 2,5350E+03 & 18 & 2,0000E-02 & 0 & 1,4749E+04 & 18 & 3,0000E-02 \\ \hline
nemsemm1 & 0 & 2,4298E+05 & 57 & 1,5710E+01 & 0 & 3,9867E+05 & 57 & 1,9620E+01 \\ \hline
nemsemm2 & 0 & 6,2496E+04 & 35 & 8,7000E-01 & 0 & 3,8140E+05 & 35 & 3,3700E+00 \\ \hline
nemspmm1 & 0 & 1,1464E+05 & 39 & 7,4000E-01 & 0 & 8,6560E+05 & 31 & 9,0400E+00 \\ \hline
nemspmm2,in & 0 & 1,1890E+05 & 44 & 1,0100E+00 & 0 & 7,1490E+05 & 34 & 7,4900E+00 \\ \hline
nemswrld & 0 & 7,8312E+05 & 42 & 7,8000E+00 & 0 & 4,4720E+06 & 38 & 9,3030E+01 \\ \hline
nsct1 & 0 & 5,4180E+06 & 21 & 1,7080E+02 & 0 & 8,5935E+06 & 22 & 6,1142E+02 \\ \hline
nsct2 & 0 & 5,6456E+06 & 28 & 2,3305E+02 & 0 & 8,3502E+06 & 30 & 7,0599E+02 \\ \hline
nsic1 & 0 & 5,0420E+03 & 10 & 2,0000E-02 & 0 & 7,2400E+03 & 10 & 3,0000E-02 \\ \hline
nsic2 & 0 & 5,5160E+03 & 14 & 2,0000E-02 & 0 & 8,0110E+03 & 14 & 4,0000E-02 \\ \hline
nsir1 & 0 & 3,0346E+05 & 21 & 3,6600E+00 & 0 & 6,3039E+05 & 17 & 1,0390E+01 \\ \hline
nsir2 & 0 & 3,2795E+05 & 30 & 5,9400E+00 & 0 & 6,5743E+05 & 26 & 1,4420E+01 \\ \hline
nw14,mps & 3 & 0,0000E+00 & 0 & 0,0000E+00 & 3 & 0,0000E+00 & 0 & 0,0000E+00 \\ \hline
olivier & 14 & 4,6137E+05 & 50 & 4,5700E+00 & 14 & 5,6194E+06 & 43 & 4,4814E+02 \\ \hline
or4,mps & 3 & 0,0000E+00 & 0 & 0,0000E+00 & 3 & 0,0000E+00 & 0 & 0,0000E+00 \\ \hline
orna1,mps & 3 & 0,0000E+00 & 0 & 0,0000E+00 & 3 & 0,0000E+00 & 0 & 0,0000E+00 \\ \hline
orna2,mps & 3 & 0,0000E+00 & 0 & 0,0000E+00 & 3 & 0,0000E+00 & 0 & 0,0000E+00 \\ \hline
orna3,mps & 3 & 0,0000E+00 & 0 & 0,0000E+00 & 3 & 0,0000E+00 & 0 & 0,0000E+00 \\ \hline
orna7,mps & 3 & 0,0000E+00 & 0 & 0,0000E+00 & 3 & 0,0000E+00 & 0 & 0,0000E+00 \\ \hline
orswq2,mps & 3 & 0,0000E+00 & 0 & 0,0000E+00 & 3 & 0,0000E+00 & 0 & 0,0000E+00 \\ \hline
p0033 & 0 & 5,8000E+01 & 8 & 0,0000E+00 & 0 & 6,0000E+01 & 8 & 0,0000E+00 \\ \hline
p0040 & 0 & 9,6000E+01 & 9 & 0,0000E+00 & 0 & 9,6000E+01 & 9 & 0,0000E+00 \\ \hline
p0201 & 0 & 2,9220E+03 & 5 & 1,0000E-02 & 0 & 7,8160E+03 & 5 & 1,0000E-02 \\ \hline
p0282 & 0 & 9,5870E+03 & 13 & 2,0000E-02 & 0 & 1,3065E+04 & 13 & 4,0000E-02 \\ \hline
p0291 & 0 & 4,8890E+03 & 12 & 1,0000E-02 & 0 & 8,7050E+03 & 12 & 2,0000E-02 \\ \hline
p0548 & 0 & 1,0530E+03 & 24 & 2,0000E-02 & 0 & 1,1890E+03 & 24 & 2,0000E-02 \\ \hline
p05 & 0 & 2,3647E+05 & 23 & 7,1000E-01 & 0 & 2,2887E+06 & 19 & 2,3310E+01 \\ \hline
p10 & 0 & 4,6962E+05 & 27 & 1,6900E+00 & 0 & 7,7808E+06 & 29 & 2,5367E+02 \\ \hline
p12345 & 0 & 6,4360E+03 & 14 & 8,0000E-02 & 0 & 6,4360E+03 & 14 & 8,0000E-02 \\ \hline
p19328 & 0 & 1,6932E+04 & 15 & 2,7000E-01 & 0 & 1,6932E+04 & 15 & 3,1000E-01 \\ \hline
p19 & 0 & 2,9710E+03 & 19 & 2,0000E-02 & 0 & 9,8150E+03 & 19 & 3,0000E-02 \\ \hline
p2756 & 0 & 6,5240E+03 & 15 & 5,0000E-02 & 0 & 6,7656E+04 & 15 & 2,0000E-01 \\ \hline
pcb1000 & 0 & 2,9753E+04 & 23 & 1,8000E-01 & 0 & 3,9440E+05 & 23 & 1,8200E+00 \\ \hline
pcb3000 & 0 & 1,0109E+05 & 25 & 5,6000E-01 & 0 & 4,3551E+06 & 19 & 4,9320E+01 \\ \hline
primagaz & 0 & 1,2404E+04 & 13 & 1,1000E-01 & 0 & 1,2404E+04 & 13 & 1,4000E-01 \\ \hline
progas & 0 & 3,0946E+04 & 15 & 6,0000E-02 & 0 & 4,8144E+04 & 15 & 2,3000E-01 \\ \hline
ps & 15 & 1,1054E+07 & 40 & 5,1600E+02 & 15 & 1,2683E+07 & 48 & 6,2349E+02 \\ \hline
qiu & 0 & 4,6492E+04 & 5 & 2,0000E-02 & 0 & 1,0207E+05 & 5 & 8,0000E-02 \\ \hline
r05 & 0 & 4,3308E+05 & 23 & 1,7800E+00 & 0 & 2,1250E+06 & 18 & 2,6230E+01 \\ \hline
radio & -1 & 0,0000E+00 & 0 & 0,0000E+00 & -11 & 0,0000E+00 & 0 & 0,0000E+00 \\ \hline
rlf,pre,dual & 0 & 8,4078E+05 & 13 & 2,3400E+00 & 0 & 6,0951E+06 & 13 & 4,8841E+02 \\ \hline
routing & 0 & 3,1143E+06 & 20 & 6,0600E+00 & 0 & 1,2819E+07 & 17 & 8,7570E+01 \\ \hline
sc205 & 12 & 5,0422E+04 & 15 & 5,3000E-01 & 12 & 5,3625E+04 & 15 & 6,0000E-01 \\ \hline
seymour & 0 & 4,5141E+06 & 15 & 3,6330E+01 & 0 & 8,2165E+06 & 15 & 1,3423E+02 \\ \hline
slp-tsk & 0 & 4,0878E+06 & 19 & 6,4860E+01 & 0 & 4,0918E+06 & 19 & 1,2514E+02 \\ \hline
southern1 & 0 & 5,6710E+06 & 15 & 6,5000E+01 & 0 & 7,7658E+06 & 15 & 6,6628E+02 \\ \hline
sturing & 0 & 1,8861E+04 & 36 & 1,5000E-01 & 0 & 3,6899E+04 & 36 & 3,0000E-01 \\ \hline
sws & 0 & 5,8235E+04 & 10 & 2,1000E-01 & 0 & 4,6605E+05 & 10 & 5,9000E-01 \\ \hline
t0331-4l & 0 & 2,0167E+05 & 37 & 6,5800E+00 & 0 & 2,1556E+05 & 37 & 7,0600E+00 \\ \hline
test & 0 & 6,5540E+03 & 46 & 2,7000E-01 & 0 & 6,5540E+03 & 46 & 2,9000E-01 \\ \hline
testps,mod & 0 & 3,1000E+01 & 13 & 0,0000E+00 & 0 & 3,1000E+01 & 13 & 0,0000E+00 \\ \hline
unilever2 & 15 & 4,2276E+05 & 36 & 3,4700E+00 & 0 & 5,8050E+06 & 73 & 2,4664E+02 \\ \hline
us04,mps & 3 & 0,0000E+00 & 0 & 0,0000E+00 & 3 & 0,0000E+00 & 0 & 0,0000E+00 \\ \hline
world & 14 & 1,1112E+06 & 60 & 9,2800E+00 & -1 & 0,0000E+00 & 0 & 0,0000E+00 \\ \hline
zed & 0 & 3,3800E+03 & 9 & 2,0000E-02 & 0 & 3,9520E+03 & 9 & 2,0000E-02 \\ \hline
zz & 15 & 2,8570E+05 & 43 & 3,7000E+00 & 15 & 9,9260E+05 & 43 & 1,0700E+01 \\ \hline
\end{tabular}

\end{table}
\begin{table}
  \centering
  \caption{Resultados experimentais da biblioteca Netlib (1/3).}
  \label{tab:resulnet0}
  \begin{tabular}{|l|r|r|r|r|r|r|r|r|}
\hline
\multicolumn{1}{|c|}{Problema} & \multicolumn{4}{|c|}{MMD} &         \multicolumn{4}{|c|}{RCM} \\ \hline
Nome & R & NNZ & IT & T & R & NNZ & IT & T \\ \hline
25fv47 & 0 & 33809 & 25 & 1,100000e-01 & 0 & 70740 & 25 & 2,800000e-01 \\ \hline
80bau3b & 0 & 41367 & 36 & 3,100000e-01 & 0 & 230301 & 36 & 1,440000e+00 \\ \hline
adlittle & 0 & 404 & 11 & 0,000000e+00 & 0 & 503 & 11 & 0,000000e+00 \\ \hline
afiro & 0 & 107 & 7 & 0,000000e+00 & 0 & 166 & 7 & 0,000000e+00 \\ \hline
agg2 & 0 & 21482 & 21 & 6,000000e-02 & 0 & 59242 & 21 & 1,700000e-01 \\ \hline
agg3 & 0 & 21482 & 19 & 6,000000e-02 & 0 & 59242 & 19 & 1,500000e-01 \\ \hline
agg & 0 & 12297 & 18 & 3,000000e-02 & 0 & 26647 & 18 & 6,000000e-02 \\ \hline
bandm & 0 & 3936 & 16 & 2,000000e-02 & 0 & 7083 & 16 & 2,000000e-02 \\ \hline
beaconfd & 0 & 820 & 10 & 1,000000e-02 & 0 & 1029 & 10 & 0,000000e+00 \\ \hline
blend & 0 & 913 & 9 & 0,000000e+00 & 0 & 1813 & 9 & 1,000000e-02 \\ \hline
bnl1 & 0 & 12089 & 33 & 6,000000e-02 & 0 & 31463 & 33 & 1,400000e-01 \\ \hline
bnl2 & 0 & 81275 & 33 & 3,100000e-01 & 0 & 233939 & 33 & 1,340000e+00 \\ \hline
boeing1 & 0 & 5725 & 18 & 2,000000e-02 & 0 & 39025 & 19 & 9,000000e-02 \\ \hline
boeing2 & 0 & 2029 & 13 & 1,000000e-02 & 0 & 3093 & 13 & 2,000000e-02 \\ \hline
bore3d & 0 & 1034 & 15 & 1,000000e-02 & 0 & 1408 & 15 & 1,000000e-02 \\ \hline
brandy & 14 & 2755 & 17 & 2,000000e-02 & 14 & 3708 & 17 & 2,000000e-02 \\ \hline
capri & 0 & 3962 & 17 & 2,000000e-02 & 0 & 5711 & 17 & 2,000000e-02 \\ \hline
cycle & 0 & 56102 & 23 & 1,600000e-01 & 0 & 178165 & 23 & 8,000000e-01 \\ \hline
czprob & 0 & 3520 & 26 & 4,000000e-02 & 0 & 91320 & 26 & 2,200000e-01 \\ \hline
d2q06c & 0 & 137349 & 27 & 5,400000e-01 & 0 & 455430 & 24 & 3,010000e+00 \\ \hline
d6cube & 0 & 54840 & 17 & 2,200000e-01 & 0 & 65467 & 17 & 2,800000e-01 \\ \hline
degen2 & 0 & 16319 & 11 & 3,000000e-02 & 0 & 40896 & 11 & 8,000000e-02 \\ \hline
degen3 & 0 & 120906 & 16 & 5,000000e-01 & 0 & 531478 & 13 & 2,530000e+00 \\ \hline
dfl001 & 0 & 1638085 & 45 & 2,443000e+01 & 15 & 5720111 & 60 & 3,470800e+02 \\ \hline
e226 & 0 & 3229 & 18 & 2,000000e-02 & 0 & 7715 & 18 & 3,000000e-02 \\ \hline
etamacro & 0 & 10843 & 26 & 3,000000e-02 & 0 & 23260 & 26 & 7,000000e-02 \\ \hline
fffff800 & 0 & 9573 & 29 & 7,000000e-02 & 0 & 33386 & 29 & 1,300000e-01 \\ \hline
finnis & 0 & 4984 & 22 & 3,000000e-02 & 0 & 10466 & 22 & 4,000000e-02 \\ \hline
fit1d & 0 & 296 & 17 & 6,000000e-02 & 0 & 298 & 17 & 6,000000e-02 \\ \hline
fit1p & 0 & 627 & 16 & 7,000000e-02 & 0 & 627 & 16 & 8,000000e-02 \\ \hline
fit2d & 0 & 324 & 22 & 7,100000e-01 & 0 & 325 & 22 & 7,100000e-01 \\ \hline
fit2p & 0 & 3000 & 20 & 5,400000e-01 & 0 & 3000 & 20 & 5,800000e-01 \\ \hline
forplan & 0 & 3304 & 21 & 3,000000e-02 & 0 & 4352 & 21 & 3,000000e-02 \\ \hline
ganges & 0 & 29677 & 16 & 6,000000e-02 & 0 & 48834 & 16 & 1,600000e-01 \\ \hline
gfrd-pnc & 0 & 2112 & 16 & 2,000000e-02 & 0 & 3316 & 16 & 2,000000e-02 \\ \hline
greenbea & 14 & 49055 & 50 & 4,400000e-01 & 14 & 150868 & 49 & 1,520000e+00 \\ \hline
greenbeb & 14 & 47783 & 40 & 3,500000e-01 & 14 & 167691 & 39 & 1,180000e+00 \\ \hline
grow15 & 0 & 6090 & 19 & 4,000000e-02 & 0 & 6090 & 19 & 5,000000e-02 \\ \hline
grow22 & 0 & 9058 & 21 & 6,000000e-02 & 0 & 9030 & 21 & 8,000000e-02 \\ \hline
grow7 & 0 & 2730 & 16 & 2,000000e-02 & 0 & 2730 & 16 & 2,000000e-02 \\ \hline
israel & 0 & 11488 & 19 & 4,000000e-02 & 0 & 11506 & 19 & 6,000000e-02 \\ \hline
kb2 & 0 & 503 & 11 & 0,000000e+00 & 0 & 747 & 11 & 0,000000e+00 \\ \hline
lotfi & 0 & 1369 & 13 & 1,000000e-02 & 0 & 2694 & 13 & 1,000000e-02 \\ \hline
maros & 0 & 13454 & 18 & 4,000000e-02 & 0 & 33456 & 18 & 1,000000e-01 \\ \hline
maros-r7 & 0 & 534188 & 14 & 1,610000e+00 & 0 & 439842 & 14 & 4,100000e+00 \\ \hline
modszk1 & 0 & 10550 & 20 & 3,000000e-02 & 0 & 82365 & 20 & 1,700000e-01 \\ \hline
nesm & 0 & 21776 & 25 & 1,200000e-01 & 0 & 29048 & 25 & 1,600000e-01 \\ \hline
perold & 0 & 21782 & 33 & 9,000000e-02 & 0 & 44505 & 33 & 2,100000e-01 \\ \hline
pilot4 & 0 & 14279 & 46 & 1,400000e-01 & 0 & 26482 & 46 & 2,100000e-01 \\ \hline
pilot87 & 0 & 425654 & 29 & 3,350000e+00 & 0 & 771344 & 25 & 7,710000e+00 \\ \hline
pilot,ja & 0 & 47924 & 29 & 2,000000e-01 & 0 & 95733 & 29 & 4,800000e-01 \\ \hline
pilot & 0 & 200812 & 34 & 1,300000e+00 & 0 & 435889 & 30 & 4,180000e+00 \\ \hline
pilotnov & 0 & 46353 & 16 & 1,100000e-01 & 0 & 95971 & 16 & 2,600000e-01 \\ \hline
pilot,we & 0 & 15605 & 45 & 1,300000e-01 & 0 & 32919 & 45 & 2,100000e-01 \\ \hline
recipe & 0 & 277 & 8 & 0,000000e+00 & 0 & 278 & 8 & 1,000000e-02 \\ \hline
sc105 & 0 & 569 & 9 & 0,000000e+00 & 0 & 825 & 9 & 0,000000e+00 \\ \hline
sc205 & 0 & 1156 & 10 & 1,000000e-02 & 0 & 1716 & 10 & 1,000000e-02 \\ \hline
sc50a & 0 & 242 & 7 & 0,000000e+00 & 0 & 326 & 7 & 0,000000e+00 \\ \hline
sc50b & 0 & 231 & 6 & 0,000000e+00 & 0 & 361 & 6 & 0,000000e+00 \\ \hline
scagr25 & 0 & 2944 & 17 & 2,000000e-02 & 0 & 4258 & 17 & 3,000000e-02 \\ \hline
scagr7 & 0 & 730 & 13 & 1,000000e-02 & 0 & 1036 & 13 & 1,000000e-02 \\ \hline
scfxm1 & 0 & 4430 & 17 & 2,000000e-02 & 0 & 7419 & 17 & 3,000000e-02 \\ \hline
scfxm2 & 14 & 9284 & 19 & 4,000000e-02 & 14 & 15458 & 19 & 7,000000e-02 \\ \hline
scfxm3 & 0 & 14138 & 19 & 6,000000e-02 & 0 & 23442 & 19 & 1,300000e-01 \\ \hline
scorpion & 0 & 2275 & 11 & 1,000000e-02 & 0 & 4926 & 11 & 2,000000e-02 \\ \hline
scrs8 & 0 & 4477 & 21 & 2,000000e-02 & 0 & 8304 & 21 & 3,000000e-02 \\ \hline
scsd1 & 0 & 1392 & 8 & 1,000000e-02 & 0 & 1472 & 8 & 0,000000e+00 \\ \hline
scsd6 & 0 & 2545 & 11 & 2,000000e-02 & 0 & 2740 & 11 & 2,000000e-02 \\ \hline
scsd8 & 0 & 5879 & 10 & 3,000000e-02 & 0 & 5942 & 10 & 3,000000e-02 \\ \hline
sctap1 & 0 & 2435 & 14 & 1,000000e-02 & 0 & 5367 & 14 & 2,000000e-02 \\ \hline
sctap2 & 0 & 11736 & 12 & 3,000000e-02 & 0 & 55394 & 12 & 1,100000e-01 \\ \hline
sctap3 & 0 & 16100 & 14 & 6,000000e-02 & 0 & 102574 & 14 & 2,600000e-01 \\ \hline
seba & 0 & 53711 & 13 & 1,700000e-01 & 0 & 55513 & 13 & 4,000000e-01 \\ \hline
share1b & 0 & 1417 & 18 & 1,000000e-02 & 0 & 1762 & 18 & 2,000000e-02 \\ \hline
share2b & 0 & 1041 & 17 & 1,000000e-02 & 0 & 1542 & 17 & 1,000000e-02 \\ \hline
shell & 0 & 3983 & 20 & 2,000000e-02 & 0 & 9539 & 20 & 4,000000e-02 \\ \hline
ship04l & 0 & 2641 & 12 & 2,000000e-02 & 0 & 7944 & 12 & 2,000000e-02 \\ \hline
ship04s & 0 & 1778 & 12 & 1,000000e-02 & 0 & 4091 & 12 & 2,000000e-02 \\ \hline
ship08l & 0 & 4442 & 14 & 3,000000e-02 & 0 & 15947 & 14 & 4,000000e-02 \\ \hline
ship08s & 0 & 2295 & 11 & 2,000000e-02 & 0 & 4290 & 11 & 2,000000e-02 \\ \hline
ship12l & 0 & 5506 & 15 & 4,000000e-02 & 0 & 16980 & 15 & 5,000000e-02 \\ \hline
ship12s & 0 & 2507 & 12 & 2,000000e-02 & 0 & 3801 & 12 & 2,000000e-02 \\ \hline
sierra & 0 & 12862 & 18 & 6,000000e-02 & 0 & 43552 & 18 & 1,400000e-01 \\ \hline
stair & 0 & 14682 & 13 & 3,000000e-02 & 0 & 15047 & 13 & 5,000000e-02 \\ \hline
standata & 0 & 2395 & 13 & 1,000000e-02 & 0 & 3953 & 13 & 2,000000e-02 \\ \hline
standgub & 0 & 2395 & 13 & 1,000000e-02 & 0 & 3953 & 13 & 1,000000e-02 \\ \hline
standmps & 0 & 3957 & 23 & 2,000000e-02 & 0 & 9136 & 23 & 3,000000e-02 \\ \hline
stocfor1 & 0 & 805 & 11 & 0,000000e+00 & 0 & 1388 & 11 & 1,000000e-02 \\ \hline
stocfor2 & 0 & 22841 & 19 & 7,000000e-02 & 0 & 32610 & 19 & 2,400000e-01 \\ \hline
tuff & 0 & 7051 & 18 & 2,000000e-02 & 0 & 9820 & 18 & 4,000000e-02 \\ \hline
vtp,base & 0 & 505 & 10 & 0,000000e+00 & 0 & 504 & 10 & 0,000000e+00 \\ \hline
wood1p & 0 & 11645 & 22 & 3,400000e-01 & 0 & 14537 & 22 & 3,600000e-01 \\ \hline
woodw & 0 & 30027 & 30 & 1,800000e-01 & 0 & 126724 & 30 & 5,800000e-01 \\ \hline
\end{tabular}

\end{table}
\begin{table}
  \centering
  \caption{Resultados experimentais da biblioteca Netlib (2/3).}
  \label{tab:resulnet1}
  \begin{tabular}{|l|r|r|r|r|r|r|r|r|}
\hline
\multicolumn{1}{|c|}{Problema} & \multicolumn{4}{|c|}{MMD} &         \multicolumn{4}{|c|}{RCM} \\ \hline
Nome & R & NNZ & IT & T & R & NNZ & IT & T \\ \hline
25fv47 & 0 & 33809 & 25 & 1,100000e-01 & 0 & 70740 & 25 & 2,800000e-01 \\ \hline
80bau3b & 0 & 41367 & 36 & 3,100000e-01 & 0 & 230301 & 36 & 1,440000e+00 \\ \hline
adlittle & 0 & 404 & 11 & 0,000000e+00 & 0 & 503 & 11 & 0,000000e+00 \\ \hline
afiro & 0 & 107 & 7 & 0,000000e+00 & 0 & 166 & 7 & 0,000000e+00 \\ \hline
agg2 & 0 & 21482 & 21 & 6,000000e-02 & 0 & 59242 & 21 & 1,700000e-01 \\ \hline
agg3 & 0 & 21482 & 19 & 6,000000e-02 & 0 & 59242 & 19 & 1,500000e-01 \\ \hline
agg & 0 & 12297 & 18 & 3,000000e-02 & 0 & 26647 & 18 & 6,000000e-02 \\ \hline
bandm & 0 & 3936 & 16 & 2,000000e-02 & 0 & 7083 & 16 & 2,000000e-02 \\ \hline
beaconfd & 0 & 820 & 10 & 1,000000e-02 & 0 & 1029 & 10 & 0,000000e+00 \\ \hline
blend & 0 & 913 & 9 & 0,000000e+00 & 0 & 1813 & 9 & 1,000000e-02 \\ \hline
bnl1 & 0 & 12089 & 33 & 6,000000e-02 & 0 & 31463 & 33 & 1,400000e-01 \\ \hline
bnl2 & 0 & 81275 & 33 & 3,100000e-01 & 0 & 233939 & 33 & 1,340000e+00 \\ \hline
boeing1 & 0 & 5725 & 18 & 2,000000e-02 & 0 & 39025 & 19 & 9,000000e-02 \\ \hline
boeing2 & 0 & 2029 & 13 & 1,000000e-02 & 0 & 3093 & 13 & 2,000000e-02 \\ \hline
bore3d & 0 & 1034 & 15 & 1,000000e-02 & 0 & 1408 & 15 & 1,000000e-02 \\ \hline
brandy & 14 & 2755 & 17 & 2,000000e-02 & 14 & 3708 & 17 & 2,000000e-02 \\ \hline
capri & 0 & 3962 & 17 & 2,000000e-02 & 0 & 5711 & 17 & 2,000000e-02 \\ \hline
cycle & 0 & 56102 & 23 & 1,600000e-01 & 0 & 178165 & 23 & 8,000000e-01 \\ \hline
czprob & 0 & 3520 & 26 & 4,000000e-02 & 0 & 91320 & 26 & 2,200000e-01 \\ \hline
d2q06c & 0 & 137349 & 27 & 5,400000e-01 & 0 & 455430 & 24 & 3,010000e+00 \\ \hline
d6cube & 0 & 54840 & 17 & 2,200000e-01 & 0 & 65467 & 17 & 2,800000e-01 \\ \hline
degen2 & 0 & 16319 & 11 & 3,000000e-02 & 0 & 40896 & 11 & 8,000000e-02 \\ \hline
degen3 & 0 & 120906 & 16 & 5,000000e-01 & 0 & 531478 & 13 & 2,530000e+00 \\ \hline
dfl001 & 0 & 1638085 & 45 & 2,443000e+01 & 15 & 5720111 & 60 & 3,470800e+02 \\ \hline
e226 & 0 & 3229 & 18 & 2,000000e-02 & 0 & 7715 & 18 & 3,000000e-02 \\ \hline
etamacro & 0 & 10843 & 26 & 3,000000e-02 & 0 & 23260 & 26 & 7,000000e-02 \\ \hline
fffff800 & 0 & 9573 & 29 & 7,000000e-02 & 0 & 33386 & 29 & 1,300000e-01 \\ \hline
finnis & 0 & 4984 & 22 & 3,000000e-02 & 0 & 10466 & 22 & 4,000000e-02 \\ \hline
fit1d & 0 & 296 & 17 & 6,000000e-02 & 0 & 298 & 17 & 6,000000e-02 \\ \hline
fit1p & 0 & 627 & 16 & 7,000000e-02 & 0 & 627 & 16 & 8,000000e-02 \\ \hline
fit2d & 0 & 324 & 22 & 7,100000e-01 & 0 & 325 & 22 & 7,100000e-01 \\ \hline
fit2p & 0 & 3000 & 20 & 5,400000e-01 & 0 & 3000 & 20 & 5,800000e-01 \\ \hline
forplan & 0 & 3304 & 21 & 3,000000e-02 & 0 & 4352 & 21 & 3,000000e-02 \\ \hline
ganges & 0 & 29677 & 16 & 6,000000e-02 & 0 & 48834 & 16 & 1,600000e-01 \\ \hline
gfrd-pnc & 0 & 2112 & 16 & 2,000000e-02 & 0 & 3316 & 16 & 2,000000e-02 \\ \hline
greenbea & 14 & 49055 & 50 & 4,400000e-01 & 14 & 150868 & 49 & 1,520000e+00 \\ \hline
greenbeb & 14 & 47783 & 40 & 3,500000e-01 & 14 & 167691 & 39 & 1,180000e+00 \\ \hline
grow15 & 0 & 6090 & 19 & 4,000000e-02 & 0 & 6090 & 19 & 5,000000e-02 \\ \hline
grow22 & 0 & 9058 & 21 & 6,000000e-02 & 0 & 9030 & 21 & 8,000000e-02 \\ \hline
grow7 & 0 & 2730 & 16 & 2,000000e-02 & 0 & 2730 & 16 & 2,000000e-02 \\ \hline
israel & 0 & 11488 & 19 & 4,000000e-02 & 0 & 11506 & 19 & 6,000000e-02 \\ \hline
kb2 & 0 & 503 & 11 & 0,000000e+00 & 0 & 747 & 11 & 0,000000e+00 \\ \hline
lotfi & 0 & 1369 & 13 & 1,000000e-02 & 0 & 2694 & 13 & 1,000000e-02 \\ \hline
maros & 0 & 13454 & 18 & 4,000000e-02 & 0 & 33456 & 18 & 1,000000e-01 \\ \hline
maros-r7 & 0 & 534188 & 14 & 1,610000e+00 & 0 & 439842 & 14 & 4,100000e+00 \\ \hline
modszk1 & 0 & 10550 & 20 & 3,000000e-02 & 0 & 82365 & 20 & 1,700000e-01 \\ \hline
nesm & 0 & 21776 & 25 & 1,200000e-01 & 0 & 29048 & 25 & 1,600000e-01 \\ \hline
perold & 0 & 21782 & 33 & 9,000000e-02 & 0 & 44505 & 33 & 2,100000e-01 \\ \hline
pilot4 & 0 & 14279 & 46 & 1,400000e-01 & 0 & 26482 & 46 & 2,100000e-01 \\ \hline
pilot87 & 0 & 425654 & 29 & 3,350000e+00 & 0 & 771344 & 25 & 7,710000e+00 \\ \hline
pilot,ja & 0 & 47924 & 29 & 2,000000e-01 & 0 & 95733 & 29 & 4,800000e-01 \\ \hline
pilot & 0 & 200812 & 34 & 1,300000e+00 & 0 & 435889 & 30 & 4,180000e+00 \\ \hline
pilotnov & 0 & 46353 & 16 & 1,100000e-01 & 0 & 95971 & 16 & 2,600000e-01 \\ \hline
pilot,we & 0 & 15605 & 45 & 1,300000e-01 & 0 & 32919 & 45 & 2,100000e-01 \\ \hline
recipe & 0 & 277 & 8 & 0,000000e+00 & 0 & 278 & 8 & 1,000000e-02 \\ \hline
sc105 & 0 & 569 & 9 & 0,000000e+00 & 0 & 825 & 9 & 0,000000e+00 \\ \hline
sc205 & 0 & 1156 & 10 & 1,000000e-02 & 0 & 1716 & 10 & 1,000000e-02 \\ \hline
sc50a & 0 & 242 & 7 & 0,000000e+00 & 0 & 326 & 7 & 0,000000e+00 \\ \hline
sc50b & 0 & 231 & 6 & 0,000000e+00 & 0 & 361 & 6 & 0,000000e+00 \\ \hline
scagr25 & 0 & 2944 & 17 & 2,000000e-02 & 0 & 4258 & 17 & 3,000000e-02 \\ \hline
scagr7 & 0 & 730 & 13 & 1,000000e-02 & 0 & 1036 & 13 & 1,000000e-02 \\ \hline
scfxm1 & 0 & 4430 & 17 & 2,000000e-02 & 0 & 7419 & 17 & 3,000000e-02 \\ \hline
scfxm2 & 14 & 9284 & 19 & 4,000000e-02 & 14 & 15458 & 19 & 7,000000e-02 \\ \hline
scfxm3 & 0 & 14138 & 19 & 6,000000e-02 & 0 & 23442 & 19 & 1,300000e-01 \\ \hline
scorpion & 0 & 2275 & 11 & 1,000000e-02 & 0 & 4926 & 11 & 2,000000e-02 \\ \hline
scrs8 & 0 & 4477 & 21 & 2,000000e-02 & 0 & 8304 & 21 & 3,000000e-02 \\ \hline
scsd1 & 0 & 1392 & 8 & 1,000000e-02 & 0 & 1472 & 8 & 0,000000e+00 \\ \hline
scsd6 & 0 & 2545 & 11 & 2,000000e-02 & 0 & 2740 & 11 & 2,000000e-02 \\ \hline
scsd8 & 0 & 5879 & 10 & 3,000000e-02 & 0 & 5942 & 10 & 3,000000e-02 \\ \hline
sctap1 & 0 & 2435 & 14 & 1,000000e-02 & 0 & 5367 & 14 & 2,000000e-02 \\ \hline
sctap2 & 0 & 11736 & 12 & 3,000000e-02 & 0 & 55394 & 12 & 1,100000e-01 \\ \hline
sctap3 & 0 & 16100 & 14 & 6,000000e-02 & 0 & 102574 & 14 & 2,600000e-01 \\ \hline
seba & 0 & 53711 & 13 & 1,700000e-01 & 0 & 55513 & 13 & 4,000000e-01 \\ \hline
share1b & 0 & 1417 & 18 & 1,000000e-02 & 0 & 1762 & 18 & 2,000000e-02 \\ \hline
share2b & 0 & 1041 & 17 & 1,000000e-02 & 0 & 1542 & 17 & 1,000000e-02 \\ \hline
shell & 0 & 3983 & 20 & 2,000000e-02 & 0 & 9539 & 20 & 4,000000e-02 \\ \hline
ship04l & 0 & 2641 & 12 & 2,000000e-02 & 0 & 7944 & 12 & 2,000000e-02 \\ \hline
ship04s & 0 & 1778 & 12 & 1,000000e-02 & 0 & 4091 & 12 & 2,000000e-02 \\ \hline
ship08l & 0 & 4442 & 14 & 3,000000e-02 & 0 & 15947 & 14 & 4,000000e-02 \\ \hline
ship08s & 0 & 2295 & 11 & 2,000000e-02 & 0 & 4290 & 11 & 2,000000e-02 \\ \hline
ship12l & 0 & 5506 & 15 & 4,000000e-02 & 0 & 16980 & 15 & 5,000000e-02 \\ \hline
ship12s & 0 & 2507 & 12 & 2,000000e-02 & 0 & 3801 & 12 & 2,000000e-02 \\ \hline
sierra & 0 & 12862 & 18 & 6,000000e-02 & 0 & 43552 & 18 & 1,400000e-01 \\ \hline
stair & 0 & 14682 & 13 & 3,000000e-02 & 0 & 15047 & 13 & 5,000000e-02 \\ \hline
standata & 0 & 2395 & 13 & 1,000000e-02 & 0 & 3953 & 13 & 2,000000e-02 \\ \hline
standgub & 0 & 2395 & 13 & 1,000000e-02 & 0 & 3953 & 13 & 1,000000e-02 \\ \hline
standmps & 0 & 3957 & 23 & 2,000000e-02 & 0 & 9136 & 23 & 3,000000e-02 \\ \hline
stocfor1 & 0 & 805 & 11 & 0,000000e+00 & 0 & 1388 & 11 & 1,000000e-02 \\ \hline
stocfor2 & 0 & 22841 & 19 & 7,000000e-02 & 0 & 32610 & 19 & 2,400000e-01 \\ \hline
tuff & 0 & 7051 & 18 & 2,000000e-02 & 0 & 9820 & 18 & 4,000000e-02 \\ \hline
vtp,base & 0 & 505 & 10 & 0,000000e+00 & 0 & 504 & 10 & 0,000000e+00 \\ \hline
wood1p & 0 & 11645 & 22 & 3,400000e-01 & 0 & 14537 & 22 & 3,600000e-01 \\ \hline
woodw & 0 & 30027 & 30 & 1,800000e-01 & 0 & 126724 & 30 & 5,800000e-01 \\ \hline
\end{tabular}

\end{table}
\begin{table}
  \centering
  \caption{Resultados experimentais da biblioteca Netlib (3/3).}
  \label{tab:resulnet2}
  \begin{tabular}{|l|r|r|r|r|r|r|r|r|}
\hline
\multicolumn{1}{|c|}{Problema} & \multicolumn{4}{|c|}{MMD} &         \multicolumn{4}{|c|}{RCM} \\ \hline
Nome & R & NNZ & IT & T & R & NNZ & IT & T \\ \hline
25fv47 & 0 & 33809 & 25 & 1,100000e-01 & 0 & 70740 & 25 & 2,800000e-01 \\ \hline
80bau3b & 0 & 41367 & 36 & 3,100000e-01 & 0 & 230301 & 36 & 1,440000e+00 \\ \hline
adlittle & 0 & 404 & 11 & 0,000000e+00 & 0 & 503 & 11 & 0,000000e+00 \\ \hline
afiro & 0 & 107 & 7 & 0,000000e+00 & 0 & 166 & 7 & 0,000000e+00 \\ \hline
agg2 & 0 & 21482 & 21 & 6,000000e-02 & 0 & 59242 & 21 & 1,700000e-01 \\ \hline
agg3 & 0 & 21482 & 19 & 6,000000e-02 & 0 & 59242 & 19 & 1,500000e-01 \\ \hline
agg & 0 & 12297 & 18 & 3,000000e-02 & 0 & 26647 & 18 & 6,000000e-02 \\ \hline
bandm & 0 & 3936 & 16 & 2,000000e-02 & 0 & 7083 & 16 & 2,000000e-02 \\ \hline
beaconfd & 0 & 820 & 10 & 1,000000e-02 & 0 & 1029 & 10 & 0,000000e+00 \\ \hline
blend & 0 & 913 & 9 & 0,000000e+00 & 0 & 1813 & 9 & 1,000000e-02 \\ \hline
bnl1 & 0 & 12089 & 33 & 6,000000e-02 & 0 & 31463 & 33 & 1,400000e-01 \\ \hline
bnl2 & 0 & 81275 & 33 & 3,100000e-01 & 0 & 233939 & 33 & 1,340000e+00 \\ \hline
boeing1 & 0 & 5725 & 18 & 2,000000e-02 & 0 & 39025 & 19 & 9,000000e-02 \\ \hline
boeing2 & 0 & 2029 & 13 & 1,000000e-02 & 0 & 3093 & 13 & 2,000000e-02 \\ \hline
bore3d & 0 & 1034 & 15 & 1,000000e-02 & 0 & 1408 & 15 & 1,000000e-02 \\ \hline
brandy & 14 & 2755 & 17 & 2,000000e-02 & 14 & 3708 & 17 & 2,000000e-02 \\ \hline
capri & 0 & 3962 & 17 & 2,000000e-02 & 0 & 5711 & 17 & 2,000000e-02 \\ \hline
cycle & 0 & 56102 & 23 & 1,600000e-01 & 0 & 178165 & 23 & 8,000000e-01 \\ \hline
czprob & 0 & 3520 & 26 & 4,000000e-02 & 0 & 91320 & 26 & 2,200000e-01 \\ \hline
d2q06c & 0 & 137349 & 27 & 5,400000e-01 & 0 & 455430 & 24 & 3,010000e+00 \\ \hline
d6cube & 0 & 54840 & 17 & 2,200000e-01 & 0 & 65467 & 17 & 2,800000e-01 \\ \hline
degen2 & 0 & 16319 & 11 & 3,000000e-02 & 0 & 40896 & 11 & 8,000000e-02 \\ \hline
degen3 & 0 & 120906 & 16 & 5,000000e-01 & 0 & 531478 & 13 & 2,530000e+00 \\ \hline
dfl001 & 0 & 1638085 & 45 & 2,443000e+01 & 15 & 5720111 & 60 & 3,470800e+02 \\ \hline
e226 & 0 & 3229 & 18 & 2,000000e-02 & 0 & 7715 & 18 & 3,000000e-02 \\ \hline
etamacro & 0 & 10843 & 26 & 3,000000e-02 & 0 & 23260 & 26 & 7,000000e-02 \\ \hline
fffff800 & 0 & 9573 & 29 & 7,000000e-02 & 0 & 33386 & 29 & 1,300000e-01 \\ \hline
finnis & 0 & 4984 & 22 & 3,000000e-02 & 0 & 10466 & 22 & 4,000000e-02 \\ \hline
fit1d & 0 & 296 & 17 & 6,000000e-02 & 0 & 298 & 17 & 6,000000e-02 \\ \hline
fit1p & 0 & 627 & 16 & 7,000000e-02 & 0 & 627 & 16 & 8,000000e-02 \\ \hline
fit2d & 0 & 324 & 22 & 7,100000e-01 & 0 & 325 & 22 & 7,100000e-01 \\ \hline
fit2p & 0 & 3000 & 20 & 5,400000e-01 & 0 & 3000 & 20 & 5,800000e-01 \\ \hline
forplan & 0 & 3304 & 21 & 3,000000e-02 & 0 & 4352 & 21 & 3,000000e-02 \\ \hline
ganges & 0 & 29677 & 16 & 6,000000e-02 & 0 & 48834 & 16 & 1,600000e-01 \\ \hline
gfrd-pnc & 0 & 2112 & 16 & 2,000000e-02 & 0 & 3316 & 16 & 2,000000e-02 \\ \hline
greenbea & 14 & 49055 & 50 & 4,400000e-01 & 14 & 150868 & 49 & 1,520000e+00 \\ \hline
greenbeb & 14 & 47783 & 40 & 3,500000e-01 & 14 & 167691 & 39 & 1,180000e+00 \\ \hline
grow15 & 0 & 6090 & 19 & 4,000000e-02 & 0 & 6090 & 19 & 5,000000e-02 \\ \hline
grow22 & 0 & 9058 & 21 & 6,000000e-02 & 0 & 9030 & 21 & 8,000000e-02 \\ \hline
grow7 & 0 & 2730 & 16 & 2,000000e-02 & 0 & 2730 & 16 & 2,000000e-02 \\ \hline
israel & 0 & 11488 & 19 & 4,000000e-02 & 0 & 11506 & 19 & 6,000000e-02 \\ \hline
kb2 & 0 & 503 & 11 & 0,000000e+00 & 0 & 747 & 11 & 0,000000e+00 \\ \hline
lotfi & 0 & 1369 & 13 & 1,000000e-02 & 0 & 2694 & 13 & 1,000000e-02 \\ \hline
maros & 0 & 13454 & 18 & 4,000000e-02 & 0 & 33456 & 18 & 1,000000e-01 \\ \hline
maros-r7 & 0 & 534188 & 14 & 1,610000e+00 & 0 & 439842 & 14 & 4,100000e+00 \\ \hline
modszk1 & 0 & 10550 & 20 & 3,000000e-02 & 0 & 82365 & 20 & 1,700000e-01 \\ \hline
nesm & 0 & 21776 & 25 & 1,200000e-01 & 0 & 29048 & 25 & 1,600000e-01 \\ \hline
perold & 0 & 21782 & 33 & 9,000000e-02 & 0 & 44505 & 33 & 2,100000e-01 \\ \hline
pilot4 & 0 & 14279 & 46 & 1,400000e-01 & 0 & 26482 & 46 & 2,100000e-01 \\ \hline
pilot87 & 0 & 425654 & 29 & 3,350000e+00 & 0 & 771344 & 25 & 7,710000e+00 \\ \hline
pilot,ja & 0 & 47924 & 29 & 2,000000e-01 & 0 & 95733 & 29 & 4,800000e-01 \\ \hline
pilot & 0 & 200812 & 34 & 1,300000e+00 & 0 & 435889 & 30 & 4,180000e+00 \\ \hline
pilotnov & 0 & 46353 & 16 & 1,100000e-01 & 0 & 95971 & 16 & 2,600000e-01 \\ \hline
pilot,we & 0 & 15605 & 45 & 1,300000e-01 & 0 & 32919 & 45 & 2,100000e-01 \\ \hline
recipe & 0 & 277 & 8 & 0,000000e+00 & 0 & 278 & 8 & 1,000000e-02 \\ \hline
sc105 & 0 & 569 & 9 & 0,000000e+00 & 0 & 825 & 9 & 0,000000e+00 \\ \hline
sc205 & 0 & 1156 & 10 & 1,000000e-02 & 0 & 1716 & 10 & 1,000000e-02 \\ \hline
sc50a & 0 & 242 & 7 & 0,000000e+00 & 0 & 326 & 7 & 0,000000e+00 \\ \hline
sc50b & 0 & 231 & 6 & 0,000000e+00 & 0 & 361 & 6 & 0,000000e+00 \\ \hline
scagr25 & 0 & 2944 & 17 & 2,000000e-02 & 0 & 4258 & 17 & 3,000000e-02 \\ \hline
scagr7 & 0 & 730 & 13 & 1,000000e-02 & 0 & 1036 & 13 & 1,000000e-02 \\ \hline
scfxm1 & 0 & 4430 & 17 & 2,000000e-02 & 0 & 7419 & 17 & 3,000000e-02 \\ \hline
scfxm2 & 14 & 9284 & 19 & 4,000000e-02 & 14 & 15458 & 19 & 7,000000e-02 \\ \hline
scfxm3 & 0 & 14138 & 19 & 6,000000e-02 & 0 & 23442 & 19 & 1,300000e-01 \\ \hline
scorpion & 0 & 2275 & 11 & 1,000000e-02 & 0 & 4926 & 11 & 2,000000e-02 \\ \hline
scrs8 & 0 & 4477 & 21 & 2,000000e-02 & 0 & 8304 & 21 & 3,000000e-02 \\ \hline
scsd1 & 0 & 1392 & 8 & 1,000000e-02 & 0 & 1472 & 8 & 0,000000e+00 \\ \hline
scsd6 & 0 & 2545 & 11 & 2,000000e-02 & 0 & 2740 & 11 & 2,000000e-02 \\ \hline
scsd8 & 0 & 5879 & 10 & 3,000000e-02 & 0 & 5942 & 10 & 3,000000e-02 \\ \hline
sctap1 & 0 & 2435 & 14 & 1,000000e-02 & 0 & 5367 & 14 & 2,000000e-02 \\ \hline
sctap2 & 0 & 11736 & 12 & 3,000000e-02 & 0 & 55394 & 12 & 1,100000e-01 \\ \hline
sctap3 & 0 & 16100 & 14 & 6,000000e-02 & 0 & 102574 & 14 & 2,600000e-01 \\ \hline
seba & 0 & 53711 & 13 & 1,700000e-01 & 0 & 55513 & 13 & 4,000000e-01 \\ \hline
share1b & 0 & 1417 & 18 & 1,000000e-02 & 0 & 1762 & 18 & 2,000000e-02 \\ \hline
share2b & 0 & 1041 & 17 & 1,000000e-02 & 0 & 1542 & 17 & 1,000000e-02 \\ \hline
shell & 0 & 3983 & 20 & 2,000000e-02 & 0 & 9539 & 20 & 4,000000e-02 \\ \hline
ship04l & 0 & 2641 & 12 & 2,000000e-02 & 0 & 7944 & 12 & 2,000000e-02 \\ \hline
ship04s & 0 & 1778 & 12 & 1,000000e-02 & 0 & 4091 & 12 & 2,000000e-02 \\ \hline
ship08l & 0 & 4442 & 14 & 3,000000e-02 & 0 & 15947 & 14 & 4,000000e-02 \\ \hline
ship08s & 0 & 2295 & 11 & 2,000000e-02 & 0 & 4290 & 11 & 2,000000e-02 \\ \hline
ship12l & 0 & 5506 & 15 & 4,000000e-02 & 0 & 16980 & 15 & 5,000000e-02 \\ \hline
ship12s & 0 & 2507 & 12 & 2,000000e-02 & 0 & 3801 & 12 & 2,000000e-02 \\ \hline
sierra & 0 & 12862 & 18 & 6,000000e-02 & 0 & 43552 & 18 & 1,400000e-01 \\ \hline
stair & 0 & 14682 & 13 & 3,000000e-02 & 0 & 15047 & 13 & 5,000000e-02 \\ \hline
standata & 0 & 2395 & 13 & 1,000000e-02 & 0 & 3953 & 13 & 2,000000e-02 \\ \hline
standgub & 0 & 2395 & 13 & 1,000000e-02 & 0 & 3953 & 13 & 1,000000e-02 \\ \hline
standmps & 0 & 3957 & 23 & 2,000000e-02 & 0 & 9136 & 23 & 3,000000e-02 \\ \hline
stocfor1 & 0 & 805 & 11 & 0,000000e+00 & 0 & 1388 & 11 & 1,000000e-02 \\ \hline
stocfor2 & 0 & 22841 & 19 & 7,000000e-02 & 0 & 32610 & 19 & 2,400000e-01 \\ \hline
tuff & 0 & 7051 & 18 & 2,000000e-02 & 0 & 9820 & 18 & 4,000000e-02 \\ \hline
vtp,base & 0 & 505 & 10 & 0,000000e+00 & 0 & 504 & 10 & 0,000000e+00 \\ \hline
wood1p & 0 & 11645 & 22 & 3,400000e-01 & 0 & 14537 & 22 & 3,600000e-01 \\ \hline
woodw & 0 & 30027 & 30 & 1,800000e-01 & 0 & 126724 & 30 & 5,800000e-01 \\ \hline
\end{tabular}

\end{table}
\begin{table}
  \centering
  \caption{Resultados experimentais da biblioteca Netlib-QAP.}
  \label{tab:resulnetqap}
  \begin{tabular}{|l|r|r|r|r|r|r|r|r|}
\hline
\multicolumn{1}{|c|}{Problema} & \multicolumn{4}{|c|}{MMD} &         \multicolumn{4}{|c|}{RCM} \\ \hline
Nome & R & NNZ & IT & T & R & NNZ & IT & T \\ \hline
qap12 & 15 & 2138580 & 41 & 3,427000e+01 & 15 & 3439794 & 42 & 8,677000e+01 \\ \hline
qap15 & 15 & 8197968 & 46 & 3,353400e+02 & 15 & 13674441 & 54 & 8,185200e+02 \\ \hline
qap8 & 0 & 193944 & 8 & 2,400000e-01 & 0 & 273509 & 32 & 1,720000e+00 \\ \hline
\end{tabular}

\end{table}
\begin{table}
  \centering
  \caption{Resultados experimentais da biblioteca PDS.}
  \label{tab:resulpds}
  \begin{tabular}{|l|r|r|r|r|r|r|r|r|}
\hline
\multicolumn{1}{|c|}{Problema} & \multicolumn{4}{|c|}{MMD} &         \multicolumn{4}{|c|}{RCM} \\ \hline
\multicolumn{1}{|c|}{Nome} & \multicolumn{1}{|c|}{R} &
        \multicolumn{1}{|c|}{NNZ} & \multicolumn{1}{|c|}{IT} &
        \multicolumn{1}{|c|}{T} & \multicolumn{1}{|c|}{R} &
        \multicolumn{1}{|c|}{NNZ} & \multicolumn{1}{|c|}{IT} &
        \multicolumn{1}{|c|}{T} \\ \hline
pds-20 & 0 & 4,1662E+07 & 39 & 2,8013E+03 & 0 & 4,1662E+07 & 39 & 2,8013E+03 \\ \hline
pds-30 & 0 & 1,4644E+07 & 48 & 5,4419E+02 & 0 & 1,4644E+07 & 48 & 5,4419E+02 \\ \hline
pds-40 & 0 & 2,8195E+07 & 50 & 1,6533E+03 & 0 & 2,8195E+07 & 50 & 1,6533E+03 \\ \hline
pds-50 & 0 & 4,1767E+07 & 54 & 3,0229E+03 & 0 & 4,1767E+07 & 54 & 3,0229E+03 \\ \hline
pds-60 & 0 & 5,8119E+07 & 53 & 5,1108E+03 & 0 & 5,8119E+07 & 53 & 5,1108E+03 \\ \hline
pds-70 & -1 & 0,0000E+00 & 0 & 0,0000E+00 & -1 & 0,0000E+00 & 0 & 0,0000E+00 \\ \hline
pds-80 & -1 & 0,0000E+00 & 0 & 0,0000E+00 & -1 & 0,0000E+00 & 0 & 0,0000E+00 \\ \hline
pds-90 & -1 & 0,0000E+00 & 0 & 0,0000E+00 & -1 & 0,0000E+00 & 0 & 0,0000E+00 \\ \hline
pds-100 & -1 & 0,0000E+00 & 0 & 0,0000E+00 & -1 & 0,0000E+00 & 0 & 0,0000E+00 \\ \hline
\end{tabular}

\end{table}
\begin{table}
  \centering
  \caption{Resultados experimentais da biblioteca Rail.}
  \label{tab:resulrail}
  \begin{tabular}{|l|r|r|r|r|r|r|r|r|}
\hline
\multicolumn{1}{|c|}{Problema} & \multicolumn{4}{|c|}{MMD} &         \multicolumn{4}{|c|}{RCM} \\ \hline
\multicolumn{1}{|c|}{Nome} & \multicolumn{1}{|c|}{R} &
        \multicolumn{1}{|c|}{NNZ} & \multicolumn{1}{|c|}{IT} &
        \multicolumn{1}{|c|}{T} & \multicolumn{1}{|c|}{R} &
        \multicolumn{1}{|c|}{NNZ} & \multicolumn{1}{|c|}{IT} &
        \multicolumn{1}{|c|}{T} \\ \hline
rail2586 & 0 & 1,2368E+06 & 90 & 2,3844E+02 & 0 & 1,4425E+06 & 90 & 2,6242E+02 \\ \hline
rail4284 & 0 & 5,8048E+06 & 71 & 6,3984E+02 & 0 & 7,0453E+06 & 71 & 8,4889E+02 \\ \hline
rail507 & 0 & 6,8847E+04 & 36 & 3,3400E+00 & 0 & 7,7942E+04 & 30 & 3,6300E+00 \\ \hline
rail516 & 0 & 5,7220E+04 & 29 & 2,1600E+00 & 0 & 7,3268E+04 & 29 & 2,3700E+00 \\ \hline
rail582 & 0 & 8,6529E+04 & 35 & 4,0400E+00 & 0 & 1,1498E+05 & 35 & 4,3800E+00 \\ \hline
\end{tabular}

\end{table}
