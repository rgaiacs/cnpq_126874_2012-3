% Copyright (C) 2012 Raniere Silva
% 
% This file is part of 'CNPq 126874/2012-3'.
% 
% 'CNPq 126874/2012-3' is licensed under the Creative Commons
% Attribution 3.0 Unported License. To view a copy of this license,
% visit http://creativecommons.org/licenses/by/3.0/.
% 
% 'CNPq 126874/2012-3' is distributed in the hope that it will be
% useful, but WITHOUT ANY WARRANTY; without even the implied warranty of
% MERCHANTABILITY or FITNESS FOR A PARTICULAR PURPOSE.

\section{Implementações}
A seguir enumeramos algumas das implementações disponíveis.

\subsection{Fortran90}
\subsubsection{Independente, John Burkardt}
John Burkardt, do Departamento de Computação Científica da The Florida State
University, implementou o RCM.

Disponível em \url{http://people.sc.fsu.edu/~jburkardt/f_src/rcm/rcm.html} e
sob licença GNU LGPL.

\subsection{C}
\subsubsection{Independente, David Fritzsche}
David Fritzsche implementou o RCM para sua tese de mestrado (``Graph
Theoretical Methods for Preconditioners'', defendida em 2004 no
Departamento de Matemática da Bergische Universitat Wuppertal. 

Disponível em \url{http://math.temple.edu/~daffi/software/rcm/} e sob licença ao
estilo BSD.

\subsubsection{Independente, John Burkardt}
John Burkardt, do Departamento de Computação Científica da The Florida State
University, implementou o RCM.

Disponível em \url{http://people.sc.fsu.edu/~jburkardt/cpp_src/rcm/rcm.html} e
sob licença GNU LGPL.

\subsection{C++}
\subsubsection{The Boost Graph Library (BGL)}
É uma biblioteca para trabalhar com grafos.

Encontra-se disponível em
\url{http://www.boost.org/doc/libs/1_51_0/libs/graph/doc/index.html} e sob
\textit{Boost license}.

\subsection{Python}
\subsubsection{PyOrder}
É uma biblioteca para reordenamento de matrizes esparsas.

Encontra-se disponível em \url{https://github.com/dpo/pyorder} sob licença GNU
LGPL.

\subsubsection{NetworkX}
É uma biblioteca para trabalhar com grafos.

Encontra-se disponível em \url{http://networkx.lanl.gov/index.html} e a
implementação do RCM em
\url{http://networkx.lanl.gov/examples/algorithms/rcm.html?highlight=cuthill}.
Encontra-se sob a licença BSD.

\subsection{GNU Octave/Matlab}
\subsubsection{Independente, John Burkardt}
John Burkardt, do Departamento de Computação Científica da The Florida State
University, implementou o RCM.

Disponível em \url{http://people.sc.fsu.edu/~jburkardt/cpp_src/rcm/rcm.html} e
sob licença GNU LGPL.
