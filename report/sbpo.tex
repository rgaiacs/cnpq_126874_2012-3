% Copyright (C) 2012 Raniere Silva
% 
% This file is part of 'CNPq 126874/2012-3'.
% 
% 'CNPq 126874/2012-3' is licensed under the Creative Commons
% Attribution 3.0 Unported License. To view a copy of this license,
% visit http://creativecommons.org/licenses/by/3.0/.
% 
% 'CNPq 126874/2012-3' is distributed in the hope that it will be
% useful, but WITHOUT ANY WARRANTY; without even the implied warranty of
% MERCHANTABILITY or FITNESS FOR A PARTICULAR PURPOSE.

\documentclass[11pt,a4paper]{article}
% Limite de 8 páginas.
\usepackage[utf8]{inputenc} %Ser\'{a} utilizado a codifica\c{c}ao UTF8
\usepackage[top=3.3cm,left=2.9cm,right=2.9cm,bottom=2.5cm]{geometry} %Borda das p\'{a}gina
\usepackage[T1]{fontenc}
\usepackage{mathptmx}
%\usepackage{fontspec}
%\setmainfont{Times New Roman}
\usepackage[english,brazil]{babel} %Ser\'{a} utilizado o idioma portugu\^{e}s
\usepackage{indentfirst} %Identa\c{c}\~{a}o de linha
\usepackage{amsmath, amsfonts, amssymb, amsthm} % Pacote matem\'{a}tico
\DeclareMathOperator*{\diag}{diag}
\usepackage{graphicx} %Pacote para incluir figuras
\usepackage{rotating}
\usepackage{color} %Pacote para cores
\usepackage{tikz} %Pacote para desenho de figuras
\usetikzlibrary{matrix}
\usepackage{pgfplots} %Pacote para desenho de gr\'{a}ficos
% \usepackage{subfigure} % Pacote para subfiguras
\usepackage{url} % Pacote para url
\usepackage{hyperref} % Pacote para hyperlink
\usepackage{csvsimple}
\usepackage{algorithmic} % Pacote para algoritmos
\usepackage{algorithm} % Pacote para algoritmos
\usepackage{listings} % Pacote para c\'{o}digos
\usepackage{lscape}

%Personalização
%\pgfplotsset{
%height=0.35\textheight,
%width=0.8\textwidth,
%legend style={
%at={(0.3,0.97)},
%anchor=north,
%font=\footnotesize
%},
%xlabel={},
%ylabel={},
%xtick={},
%grid=major,
%}

\hypersetup{
%colorlinks = true,
}

\floatname{algorithm}{Algoritmo}

\algsetup{linenosize=\small}
\renewcommand{\algorithmicrequire}{\textbf{Entrada:}}
\renewcommand{\algorithmicensure}{\textbf{Saída:}}
\renewcommand{\algorithmicend}{\textbf{fim}}
\renewcommand{\algorithmicif}{\textbf{se}}
\renewcommand{\algorithmicthen}{\textbf{ent\~{a}o}}
\renewcommand{\algorithmicelse}{\textbf{caso contr\'{a}rio}}
\renewcommand{\algorithmicendif}{\algorithmicend}
\renewcommand{\algorithmicfor}{\textbf{para}}
\renewcommand{\algorithmicforall}{\textbf{para todo}}
\renewcommand{\algorithmicdo}{\textbf{fa\c{c}a}}
\renewcommand{\algorithmicendfor}{\algorithmicend}
\renewcommand{\algorithmicwhile}{\textbf{enquanto}}
\renewcommand{\algorithmicendwhile}{\algorithmicend}
\renewcommand{\algorithmicrepeat}{\textbf{repita}}
\renewcommand{\algorithmicuntil}{\textbf{at\'{e}}}
\renewcommand{\algorithmicreturn}{\textbf{retorne}}
\renewcommand{\algorithmiccomment}[1]{\hspace{2em}/* #1 */}

\renewcommand{\lstlistingname}{C\'{o}digo}
\lstset{ %
% language=Octave,                % the language of the code
basicstyle=\ttfamily\small,       % the size of the fonts that are used for the code
numbers=left,                   % where to put the line-numbers
numberstyle=\footnotesize,      % the size of the fonts that are used for the line-numbers
stepnumber=5,                   % the step between two line-numbers. If it's 1, each line 
% will be numbered
numbersep=5pt,                  % how far the line-numbers are from the code
% backgroundcolor=\color{white},  % choose the background color. You must add \usepackage{color}
showspaces=false,               % show spaces adding particular underscores
showstringspaces=false,         % underline spaces within strings
showtabs=false,                 % show tabs within strings adding particular underscores
% frame=single,                   % adds a frame around the code
tabsize=4,                      % sets default tabsize to 2 spaces
captionpos=t,                   % sets the caption-position to bottom
breaklines=true,                % sets automatic line breaking
breakatwhitespace=false,        % sets if automatic breaks should only happen at whitespace
caption={\texttt{\lstname}},                 % show the filename of files included with \lstinputlisting;
% also try caption instead of title
% escapeinside={\%*}{*)},         % if you want to add a comment within your code
% morekeywords={#}            % if you want to add more keywords to the set
}

\newtheorem{defi}{Definição}
\newtheorem{prop}{Proposição}
\newtheorem{exem}{Exemplo}

\pagestyle{empty}
\begin{document}
% Identificação
%     Projeto
%     Bolsista / RA
%     Orientador
%     Local de execução
%     Vigência
\title{Implementa\c{c}\~{a}o eficiente da heur\'{i}stica de reordenamento de Cuthill-McKee Reversa}
\author{Raniere Gaia Costa da Silva \\
Instituto de Matemática, Estatística e Computação Científica \\
Universidade Estadual de Campinas \\
Rua Sérgio Buarque de Holanda, 651 – Cidade Universitária ``Zeferino Vaz'' \\
Distr. Barão Geraldo – Campinas – São Paulo – Brasil CEP 13083-859 \\
\url{ra092767@ime.unicamp.br}
\and Aurelio Ribeiro Leite de Oliveira \\
Instituto de Matemática, Estatística e Computação Científica \\
Universidade Estadual de Campinas \\
Rua Sérgio Buarque de Holanda, 651 – Cidade Universitária ``Zeferino Vaz'' \\
Distr. Barão Geraldo – Campinas – São Paulo – Brasil CEP 13083-859 \\
\url{aurelio@ime.unicamp.br}}
\date{}
\maketitle\thispagestyle{empty}

% Máximo de 150 palavras.
% 3 palavras chaves
\begin{abstract}
Neste trabalho revisitou-se a heurística de reordenamento Cuthill-McKee
Reversa, proposta em 1969, que busca um reordenamento para matrizes esparsas
que reduza a largura de banda destas. O autor deste trabalho implementou a heurística
revisitada, testou com parte da biblioteca
``Netlib LP'' para os quais obteve bons resultados visto que, aproximadamente,
75\% das matrizes testadas tiveram o envelope reduzido com o
reordenamento obtido pela heurística e essa redução foi em média de 35\%.

Palavras-chaves: Pontos Interiores, Preprocessamento.
\end{abstract}

\begin{abstract}
Neste trabalho revisitou-se a heurística de reordenamento Cuthill-McKee
Reversa, proposta em 1969, que busca um reordenamento para matrizes esparsas
que reduza a largura de banda destas. O autor deste trabalho implementou a heurística
revisitada, testou com parte da biblioteca
``Netlib LP'' para os quais obteve bons resultados visto que, aproximadamente,
75\% das matrizes testadas tiveram o envelope reduzido com o
reordenamento obtido pela heurística e essa redução foi em média de 35\%.

Palavras-chaves: Pontos Interiores, Preprocessamento.
\end{abstract}

\newpage

% Introdução
%     Introdução ao assunto: deve ser bastante geral
%     Informações da literatura: tornam a introdução mais específica ao assunto
%     Colocação da questão estudada: especificar os objetivos do trabalho
%     Atividades desenvolvidas: dar uma idéia geral de como foi desenvolvido o
%     trabalho
% Copyright (C) 2012 Raniere Silva
% 
% This file is part of 'CNPq 126874/2012-3'.
% 
% 'CNPq 126874/2012-3' is licensed under the Creative Commons
% Attribution 3.0 Unported License. To view a copy of this license,
% visit http://creativecommons.org/licenses/by/3.0/.
% 
% 'CNPq 126874/2012-3' is distributed in the hope that it will be
% useful, but WITHOUT ANY WARRANTY; without even the implied warranty of
% MERCHANTABILITY or FITNESS FOR A PARTICULAR PURPOSE.

\section{Introdução}
Neste trabalho revisitou-se a heurística de reordenamento Cuthill-McKee
Reversa, proposta em 1969, que busca um reordenamento para matrizes esparsas
que reduza a largura de banda destas. O autor deste relatório implementou a heurística
revisitada, testou com parte da biblioteca
``Netlib LP'' para os quais obteve bons resultados visto que, aproximadamente,
75\% das matrizes testadas tiveram o envelope reduzido com o
reordenamento obtido pela heurística e essa redução foi em média de 35\%.

\section{Métodos de Pontos Interiores Primal-Dual}
Consideremos o Problema de Programação Linear na Forma Padrão (PPLFP) e o
problema dual (PPLFD) associado:
\begin{align*}
    \text{minimizar } & c^T x
    &&& \text{maximizar } & b^T y \\
    \text{sujeito a } & A x = b
    &&\text{e}& \text{sujeito a } & A^T y + z = c \\
    & x \geq 0
    &&& & z \geq 0,
\end{align*}
onde $A \in \mathbb{R}^{m \times n}$ é uma matriz de posto completo $m$ e $c$,
$b$, $x$, $y$ e $z$ são vetores colunas de dimensão apropriada.  O \textit{gap}
dual é dado por $\gamma = c^T x - b^T y$ que se reduz a $\gamma = x^T z$ para
pontos primais e duais factíveis.

Os Métodos de Pontos Interiores Primais-Duais para resolver PPLFP consistem em, a
partir de uma tripla inicial $(x^0, y^0, z^0)$, construir uma
sequência de triplas, $(x^i, y^i, z^i)$, dada por $x^{i + 1} = x^i + \alpha^i
\Delta x^i$, $y^{i + 1} = y^i + \alpha^i \Delta y^i$, $z^{i + 1} = z^i +
\alpha^i \Delta z^i$, onde $i \geq 0$ e $\alpha^i \in (0, 1]$, que convirja para a
tripla $(x^*, y^*, z^*)$ que é solução de PPLFP e PPLFD. A constante $\alpha^i$
deve ser escolhida tal que $x^i, z^i > 0$ e, frequentemente, $\alpha^i << 1$.

A direção afim nos Métodos de Pontos Interiores Primais-Duais, $(\Delta x^i,
\Delta y^i, \Delta z^i)$,  é dada por:
\begin{align}
    \begin{bmatrix}
         0 & A^T & I \\
         A & 0 & 0 \\
         Z & 0 & X
     \end{bmatrix} \begin{bmatrix}
         \Delta x^i \\
         \Delta y^i \\
         \Delta z^i
     \end{bmatrix} &= \begin{bmatrix}
         r_d \\
         r_p \\
         r_a
     \end{bmatrix},
     \label{eq:primal_dual_intp:lin_system}
\end{align}
onde $X = \diag(x^i)$, $Z = \diag(z^i)$, $r_p = b - A x^i$, $r_d = c - A^T y^i
- z^i$, $r_a = - X Z e$ e $e$ representa o vetor de uns.

Por meio da eliminação de variáveis é possível reduzir
\eqref{eq:primal_dual_intp:lin_system} para
\begin{align}
    A D^{-1} A^T \Delta y^i &= A D^{-1} r_1 - r_2,
    \label{eq:primal_dual_intp:lin_norm}
\end{align}
tal que $\Delta x^i = D^{-1} \left( r_1 - A^T \Delta y^i \right)$ e $\Delta z^i
= X^{-1} \left( r_a - Z \Delta x^i \right)$.

Para resolver \eqref{eq:primal_dual_intp:lin_norm} costuma-se utilizar a
fatoração de Cholesky pois $A D^{-1} A^T$ é uma matriz simétrica definida
positiva. Durante a construção da sequência de triplas $(x^i, y^i, z^i)$ apenas
$D$ se altera na matriz e por esse motivo a estrutura esparsa de $A D^{-1} A^T$ é mantida
durante todas as iterações do Método de Pontos Interiores Primal-Dual.

\section{Matrizes Esparsas}
Como visto na seção anterior, um dos passos dos Métodos de Pontos Interiores
Primal-Dual consiste em resolver um sistema linear simétrico definido positivo,
cuja estrutura esparsa é mantida inalterada durante todo o método, utilizando a fatoração de
Cholesky. A fatoração de Cholesky possui algumas propriedades interessantes
relacionadas com a banda e o envelope de matrizes.

Mudando a notação, considere $A \in \mathbb{R}^{n \times n}$ uma matriz
simétrica definida positiva genérica com entradas $A_{ij}$. Para cada linha $i$
de $A$, $i = 1, \ldots, n$, seja $f_i(A) = \min\left\{ j \mid A_{ij} \neq 0
\right\}$, isso é, $f_i(A)$ corresponde a primeira coluna cujo elemento na linha
$i$ é diferente de zero.

Seguindo Cuthill e McKee \cite{Cuthill:1969:ReducingBandwidth}, definimos
$\beta_i(A) = i - f_i(A), 1 \leq i \leq n$ e $\beta(A) = \max\left\{ \beta_i(A)
\mid 1 \leq i \leq n \right\}$ em que $\beta_i(A)$ é a largura de banda da
$i$-ésima linha de $A$ e $\beta(A)$ é a largura de banda da matriz $A$. A banda
da matriz $A$ é definida como $\text{Band}(A) = \left\{ \left\{ i, j \right\}
\mid 0 < i - j \leq \beta(A) \right\}$.

O envelope da matriz $A$ é definido como $\text{Env}(A) = \left\{ \left\{ i, j
\right\} \mid 0 < i - j \leq \beta_i(A) \right\}$.

A fatoração de Cholesky preserva a banda da matriz e o envelope da matriz.

\section{Método Cuthill-McKee Reverso}
Considere a matriz $A$ e seu fator de Cholesky $G$ dados por
\begin{align*}
    A &= \begin{bmatrix}
        1 &  1 &  1 &  1 \\
        1 & 10 &  0 &  0 \\
        1 &  0 & 10 &  0 \\
        1 &  0 &  0 & 10
    \end{bmatrix}, & G &= \begin{bmatrix}
        1,0 &  0,0 &  0,0 &  0,0 \\
        1,0 &  3,0 &  0,0 &  0,0 \\
        1,0 & -0,3 &  2,9 &  0,0 \\
        1,0 & -0,3 & -0,3 &  2,9 \\
    \end{bmatrix}.
\end{align*}
Observa-se que vários elementos nulos na parte triangular inferior de $A$ são
não nulos em $G$. A perda de elementos nulos ao realizar a fatoração de
Cholesky é denominada de preenchimento e deve ser evitada sempre que possível.

Na seção anterior verificou-se que a fatoração de Cholesky preserva a banda e
envelope da matriz. Por esse motivo, ao utilizar matrizes de banda e
envelope pequenos o preenchimento da matriz também será pequeno. Será que dada
uma matriz $A \in \mathbb{R}^{n \times n}$ simétrica definida positiva existe
uma matriz $A' = P A P^T$, onde $P$ é uma matriz de permutação, tal que a banda
e/ou envelope de $A'$ é menor que o de $A$?

\begin{exem}
    Considere a matriz $A$ e $G$ dada por
    \begin{align*}
        A &= \begin{bmatrix}
            1 & 1 & 1 & 1 & 1 & 0 & 0 \\
            1 & 2 & 1 & 1 & 1 & 1 & 1 \\
            1 & 1 & 2 & 1 & 1 & 1 & 1 \\
            1 & 1 & 1 & 2 & 1 & 0 & 0 \\
            1 & 1 & 1 & 1 & 2 & 0 & 0 \\
            0 & 1 & 1 & 0 & 0 & 3 & 2 \\
            0 & 1 & 1 & 0 & 0 & 2 & 3
        \end{bmatrix},
        & G &= \begin{bmatrix}
            1 & 0 & 0 & 0 & 0 & 0 & 0 \\
            1 & 1 & 0 & 0 & 0 & 0 & 0 \\
            1 & 0 & 1 & 0 & 0 & 0 & 0 \\
            1 & 0 & 0 & 1 & 0 & 0 & 0 \\
            1 & 0 & 0 & 0 & 1 & 0 & 0 \\
            0 & 1 & 1 & 0 & 0 & 1 & 0 \\
            0 & 1 & 1 & 0 & 0 & 0 & 1
        \end{bmatrix}.
    \end{align*}
    Verifica-se que a largura da banda de $A$ e $G$ é $5$ e o envelope é $19$.

    Agora, considere a matriz $A' = P A P^T = G' G'^T$, em que $P$ é uma matriz de
    permutação, dada por
    % O vetor de permutação é
    % [4, 3, 0, 2, 1, 6, 5]
    \begin{align*}
        A' &= \begin{bmatrix}
            2 & 1 & 1 & 1 & 1 & 0 & 0 \\
            1 & 2 & 1 & 1 & 1 & 0 & 0 \\
            1 & 1 & 1 & 1 & 1 & 0 & 0 \\
            1 & 1 & 1 & 2 & 1 & 1 & 1 \\
            1 & 1 & 1 & 1 & 2 & 1 & 1 \\
            0 & 0 & 0 & 1 & 1 & 3 & 2 \\
            0 & 0 & 0 & 1 & 1 & 2 & 3
        \end{bmatrix},
        & G' &= \begin{bmatrix}
             1,4 & 0 & 0 & 0 & 0 & 0 & 0 \\
            0,70 & 1,2 & 0 & 0 & 0 & 0 & 0 \\
            0,70 & 0,40 & 0,57 & 0 & 0 & 0 & 0 \\
            0,70 & 0,40 & 0,57 & 1 & 0 & 0 & 0 \\
            0,70 & 0,40 & 0,57 & 0 & 1 & 0 & 0 \\
            0 & 0 & 0 & 1 & 1 & 1 & 0 \\
            0 & 0 & 0 & 1 & 1 & 0 & 1
        \end{bmatrix}.
    \end{align*}
    Verifica-se que a largura da banda de $A$ e $G$ é $4$ e o envelope é $15$.
\end{exem}

De acordo com o exemplo anterior, verifica-se a possibilidade de preprocessar a matriz
de forma a reduzir a banda e envelope.

Na próxima subseção apresentamos algumas definições da Teoria de Grafos que
serão utilizadas no algoritmo estudado para obter uma permutação que busca
reduzir a largura de banda e envelope.

\subsection{Grafos e matrizes esparsas}
Um grafo é, fundamentalmente, um modo de representar uma relação binária entre
objetos. Para o propósito deste trabalho, considere um grafo $G = (V, E)$ como
um conjunto de vértices $V = \{v_1, v_2, \ldots \}$ e um conjunto de
arestas $E = \{e_1, e_2, \ldots \}$, que são representadas por pares
não ordenados, por exemplo, $e_1 = \{v_1 , v_2\}$.

Assim como um grafo, uma matriz também descreve uma relação binária entre objetos
através de seus elementos não nulos. Uma matriz simétrica $A \in \mathbb{R}^{n
\times n}$ induz um grafo $G(A)$, onde os vértices do grafo correspondem as
dimensões da matriz e a aresta $e = \{i, j\}$ existe se e somente se $A_{ij} \neq 0$.
Na Figura~\ref{fig:exem_matrix2graph} é ilustrado a relação entre uma matriz e um grafo.
\begin{figure}[hbt]
    \centering
    \begin{tikzpicture}
        \matrix (A) [matrix of math nodes,%
        left delimiter  = \lbrack,%
        right delimiter = \rbrack] at (0,0)
        {%
            1 & 1 & 1 & 1 & 1 & 0 & 0 \\
            1 & 2 & 1 & 1 & 1 & 1 & 1 \\
            1 & 1 & 2 & 1 & 1 & 1 & 1 \\
            1 & 1 & 1 & 2 & 1 & 0 & 0 \\
            1 & 1 & 1 & 1 & 2 & 0 & 0 \\
            0 & 1 & 1 & 0 & 0 & 3 & 2 \\
            0 & 1 & 1 & 0 & 0 & 2 & 3 \\
        };
        %Graph
        \node[draw, circle] (1) at (6,0) {1};
        \node[draw, circle] (2) at (8,1) {2};
        \node[draw, circle] (3) at (8,-1) {3};
        \node[draw, circle] (4) at (4,1) {4};
        \node[draw, circle] (5) at (4,-1) {5};
        \node[draw, circle] (6) at (10,1) {6};
        \node[draw, circle] (7) at (10,-1) {7};
        \draw (1) -- (2);
        \draw (1) -- (3);
        \draw (1) -- (4);
        \draw (1) -- (5);
        \draw (2) -- (3);
        \draw (2) -- (4);
        \draw (2) to[out=190, in=60] (5);
        \draw (2) -- (6);
        \draw (2) -- (7);
        \draw (3) to[out=120, in=-10] (4);
        \draw (3) -- (5);
        \draw (3) -- (6);
        \draw (3) -- (7);
        \draw (4) -- (5);
        \draw (6) -- (7);
    \end{tikzpicture}
    \caption{Ilustração do grafo (a direita) correspondente a uma matriz (a
    esquerda).}
    \label{fig:exem_matrix2graph}
\end{figure}

É importante destacar que permutar, simetricamente, linhas e colunas de uma
matriz corresponde a renumerar os vértices do grafo. Na
Figura~\ref{fig:exem_matrix2graph_perm} é ilustrado uma permutação da matriz
presente na Figura~\ref{fig:exem_matrix2graph} e a renumeração dos vértices do
grafo.
\begin{figure}[!hbt]
    \centering
    \begin{tikzpicture}
        \matrix (A) [matrix of math nodes,%
        left delimiter  = \lbrack,%
        right delimiter = \rbrack] at (0,0)
        {%
            2 & 1 & 1 & 1 & 1 & 0 & 0 \\
            1 & 2 & 1 & 1 & 1 & 0 & 0 \\
            1 & 1 & 1 & 1 & 1 & 0 & 0 \\
            1 & 1 & 1 & 2 & 1 & 1 & 1 \\
            1 & 1 & 1 & 1 & 2 & 1 & 1 \\
            0 & 0 & 0 & 1 & 1 & 3 & 2 \\
            0 & 0 & 0 & 1 & 1 & 2 & 3 \\
        };
        %Graph
        \node[draw, circle] (1) at (6,0) {3};
        \node[draw, circle] (2) at (8,1) {4};
        \node[draw, circle] (3) at (8,-1) {5};
        \node[draw, circle] (4) at (4,1) {2};
        \node[draw, circle] (5) at (4,-1) {3};
        \node[draw, circle] (6) at (10,1) {6};
        \node[draw, circle] (7) at (10,-1) {7};
        \draw (1) -- (2);
        \draw (1) -- (3);
        \draw (1) -- (4);
        \draw (1) -- (5);
        \draw (2) -- (3);
        \draw (2) -- (4);
        \draw (2) to[out=190, in=60] (5);
        \draw (2) -- (6);
        \draw (2) -- (7);
        \draw (3) to[out=120, in=-10] (4);
        \draw (3) -- (5);
        \draw (3) -- (6);
        \draw (3) -- (7);
        \draw (4) -- (5);
        \draw (6) -- (7);
    \end{tikzpicture}
    \caption{Ilustração da permutação simétrica de uma matriz e a renumeração
    dos vértices do grafo correspondente.}
    \label{fig:exem_matrix2graph_perm}
\end{figure}

Além dos conceitos básicos da Teoria de Grafos, temos:
\begin{description}
    \item[Distância entre vértices] ou $d(v_1, v_2)$, número de arestas que
        formam o menor caminho ligando os vértices $v_1$ e $v_2$.
    \item[Excentricidade] ou $l(v)$, maior distância do vértice $v$ a qualquer outro.
    \item[Diâmetro] maior excentricidade dentre os vértices de um grafo.
    \item[Pseudo-diâmetro] corresponde a uma alta excentricidade, porém não
        necessariamente a maior de todas.
    \item[Vértices periféricos] são vértices cuja excentricidade é igual ao
        diâmetro do grafo.
    \item[Vértices pseudo-periféricos] são vértices que apresentam altas
        excentricidades, mas não necessariamente a maior.
    \item[Estrutura de nível] com raiz no vértice $v$ é uma partição do conjunto
      $V$ em níveis $L_0, L_2, \ldots, L_{l(v)}$ tal que
        \begin{enumerate}
            \item $L_0 = \left\{ v \right\}$,
            \item para $i > 0$, $L_i$ é o conjunto de vértices adjacentes aos
                vértices presentes no nível $L_{i - 1}$ e ainda não pertencentes
                a nenhum nível.
        \end{enumerate}

        Na Figura~\ref{fig:graph2level_struct} é ilustrado a estrutura de
        nível, dentre as várias possíveis, para um grafo.
\end{description}
\begin{figure}[htb]
    \centering
    \begin{tikzpicture}[level distance=1cm]
        % graph
        \node[draw, circle] (1) at (2,0) {1};
        \node[draw, circle] (2) at (4,1) {2};
        \node[draw, circle] (3) at (4,-1) {3};
        \node[draw, circle] (4) at (0,1) {4};
        \node[draw, circle] (5) at (0,-1) {5};
        \node[draw, circle] (6) at (6,1) {6};
        \node[draw, circle] (7) at (6,-1) {7};
        \draw (1) -- (2);
        \draw (1) -- (3);
        \draw (1) -- (4);
        \draw (1) -- (5);
        \draw (2) -- (3);
        \draw (2) -- (4);
        \draw (2) to[out=190, in=60] (5);
        \draw (2) -- (6);
        \draw (2) -- (7);
        \draw (3) to[out=120, in=-10] (4);
        \draw (3) -- (5);
        \draw (3) -- (6);
        \draw (3) -- (7);
        \draw (4) -- (5);
        \draw (6) -- (7);
        % level structure
        \node[draw, circle] at (10,1) {6}
            child { node[draw, circle] {2}
                child { node[draw, circle] {1} }
                child { node[draw, circle] {4} }
                child { node[draw, circle] {5} } } 
            child { node[draw, circle] {3} }
            child { node[draw, circle] {7} };
            \path[white, text=black] node at (13,1) {$L_1$}
            child { node {$L_2$}
                child { node {$L_3$} } };
    \end{tikzpicture}
    \caption{Grafo (esquerda) e sua estrutura de nível com raiz no vértice 6
    (direita).}
    \label{fig:graph2level_struct}
\end{figure}
\subsection{Método Cuthill-McKee Reverso}
Cuthill e McKee \cite{Cuthill:1969:ReducingBandwidth} propuseram uma heurística de
reordenação, ver algoritmo abaixo, cujo objetivo principal é reduzir a largura
de banda de uma matriz simétrica $A \in \mathbb{R}^{n \times n}$ cujo grafo
é conexo.
\begin{algorithm}[hbt]
    \caption{Pseudo-código de Cuthill-McKee}
    \label{alg:rcm}
    \begin{algorithmic}[1]
        \REQUIRE Grafo $G(A)$ e um vértice inicial $v$.
        \ENSURE $o$, novo ordenamento dos vértices de $G(A)$.
        \STATE Marca todos os vértices como não visitados.
        \STATE $o \longleftarrow \text{ vetor de zeros}$
        \STATE $i \longleftarrow 1$
        \STATE $f \longleftarrow \text{ fila vazia}$
        \STATE Adicionar $v$ na fila $f$.
        \STATE Marca $v$ como visitado.
        \WHILE{$f$ não for vazia}
            \STATE Desenfileira $f$ em $v$.
            \STATE $o_i \longleftarrow v$
            \STATE $i \longleftarrow i + 1$
            \FORALL{vértice $w$ adjacente a $v$, em ordem crescente de grau,}
                \IF{$w$ ainda não foi visitado}
                    \STATE Adicionar $v$ na fila $f$.
                    \STATE Marca $w$ como visitado.
                \ENDIF
            \ENDFOR
        \ENDWHILE
    \end{algorithmic}
\end{algorithm}

Para o caso de uma matriz $A$ cujo grafo não é conexo, Cuthill e McKee propõe
aplicar a mesma heurística para cada uma das componentes conexas.

George \cite{George:1971:ComputerImplementation} verificou
experimentalmente que ao reverter o ordenamento obtido pelo Algoritmo
Cuthill-McKee, \textit{i.e.}, trocando $o_i$ por $o_{n - i + 1}$ para $i = 1, 2,
\ldots, n$, o novo ordenamento mantém a mesma banda mas diminui o
envelope da matriz (essa modificação nunca aumenta o envelope da matriz). Essa
versão do algoritmo é conhecida como Método Cuthill-McKee Reverso (RCM).

Uma dos parâmetros de entrada do Método Cuthill-McKee (Reverso) é o vértice inicial sendo
que experimentos computacionais \cite{Cuthill:1969:ReducingBandwidth} sugerem que
vértices pseudo-periféricos são bons candidatos.

Uma heurística para encontrar vértices pseudo-periféricos foi proposto por Alan
George e Joseph W. H. Liu \cite{George:1979:NodeFinder}, ver algoritmo a seguir, e
é baseado na observação de que $y \in \mathcal{L}_{l(x)}(x) \Longrightarrow l(x)
\leq l(y)$, \textit{i.e.}, se $y$ pertence ao nível mais elevado da estrutura de
níveis com raiz em $x$ então a excentricidade de $x$ é menor ou igual a $x$.
Essa observação é verificada com maior facilidade em grafos que são uma árvore,
ver figura abaixo.
\begin{figure}[!hbt]
    \centering
    \begin{tikzpicture}[level distance=.8cm]
        \node[draw, circle] at (0,0) {1}
            child { node[draw, circle] {2} }
            child { node[draw, circle] {3}
                child { node[draw, circle] {4} }
                child { node[draw, circle] {5} } };
        \node[draw, circle] at (6,0) {5}
            child { node[draw, circle] {3}
                child { node[draw, circle] {4} }
                child { node[draw, circle] {1} 
                    child { node[draw, circle] {2} } } };
    \end{tikzpicture}
    \caption{Grafo e sua estrutura de níveis com raiz em 1 (a esquerda) e sua
    estrutura de níveis com raiz em 5 (a direita).}
    \label{fig:ilus_obser}
\end{figure}
\begin{algorithm}[H]
    \caption{Pseudo-código para encontrar vértice pseudo-periférico, $x$.}
    \label{alg:ppn}
    \begin{algorithmic}[1]
        \REQUIRE Grafo $G(A)$.
        \ENSURE $x$.
        \STATE $r \longleftarrow \text{Nó arbitrário em }G(A)$
        \STATE Construir estrutura de nível a partir de $r$, $\mathcal{L}(r)$.
        \label{alg:ppn:brle}
        \STATE Escolher um vértice $x$ pertencente ao último nível de
        $\mathcal{L}(r)$.
        \STATE Construir estrutura de nível a partir de $x$, $\mathcal{L}(x)$.
        \IF{$l(x) > l(r)$}
            \STATE $r \longleftarrow x$
            \STATE Retorna para a linha~\ref{alg:ppn:brle}
        \ENDIF
    \end{algorithmic}
\end{algorithm}


% Materiais e Métodos
%     Materiais: citar os equipamentos, reagentes e outros ítens utilizados,
%     informando fabricante ou fornecedor
%     Métodos: descrever os procedimentos detalhados, que possam ser
%     reproduzidos com os materiais e equipamentos descritos
\section{Implementação do Método e Testes Computacionais}
O Método Cuthill-McKee Reverso foi implementado, na linguagem C, pelo autor
deste trabalho como um patch para o PCx
(\url{http://pages.cs.wisc.edu/~swright/PCx/}), e a implementação desenvolvida
encontra-se disponível em \url{https://github.com/r-gaia-cs/PCx.git}.

O PCx é um solver de Programação Linear que utilizar a variante de Mehrotra do
Método Preditor Corretor com a estratégia de correção de Gondzio e o Método do
Mínimo Grau Múltiplo de Liu\cite{George:1981:ComputerSolutionPD} (MMD) para o
reordenamento da matriz $A D^{-1} A^T$.

Testou-se a implementação (tendo como último commit aquele cuja identificação
inicia com c8e1e1d) com 94 problemas da biblioteca ``Netlib LP''
(\url{http://www.netlib.org/lp/}), cujo acesso é livre.

Dos 94 problemas testados, ao utilizar reordenamento obtido com o MMD 87
problemas foram resolvidos enquanto que com o reordenamento do RCM apenas 85
problemas foram resolvidos. Dentre estes 85 problemas resolvidos por ambos, os
dados referentes aos testes computacionais dos 43 maiores problemas encontram-se
na Tabela~\ref{tab:resul2} sendo que ``LIN'', ``COL'', ``ENV'', ``IT'' e ``T''
significam, respectivamente, ``número de linhas originais'', ``número de colunas
originais'', ``tamanho do envelope'', ``número de iterações'' e ``tempo
computacional'' (em segundos).
\begin{table}
  \caption{Resultados experimentais.}
  \label{tab:resul2}
  \input{bench2@sbpo.csv}
\end{table}

O envelope decorrente do uso do RCM foi menor que o do MMD em apenas 4 problemas
e empatou em outros 5. Para os problemas em que o envelope decorrente do RCM foi
maior que o do MMD, esse aumento foi em média de 55\%. Em relação ao número de
iterações do PCx, o RCM foi melhor em 2 problemas e empatou nos outros 81. Já em
relação ao tempo, o RCM foi melhor em 6 problemas e empatou em outros 17.


% Resultados
%     Descrição dos resultados: deve ser clara e objetiva, resumindo os achados
%     principais que serão detalhados em tabelas e figuras.
%     Ilustrações dos resultados: tabelas e figuras são muito importantes; seu
%     número deve ser o menor possível, e elas devem ser construídas com cuidado
%     para incluir todas as informações necessárias com clareza.
%     Tabelas: devem ser numeradas sequencialmente (Tabela 1, Tabela 2, etc).
%     Seu título deve ser informativo, colocado acima e justificado à esquerda.
%     Notas de rodapé (a, b, c...) podem ser colocados diretamente abaixo da
%     mesma.
%     Figuras (fotos, esquemas, gráficos): devem ser numeradas sequencialmente
%     (Figura 1, Figura 2, etc). Seu título deve ser informativo, colocado
%     abaixo e justificado à esquerda, descrevendo o que é mostrado.
\section{Conclusões}
Concluímos, pelo menos para os problemas testados, que a heurística de mínimo
grau múltiplo é superior à heurística de Cuthill e McKee por gerar menos
elementos não nulos na decomposição de Cholesky (levando em conta o tamanho do
envelope). Estudos futuros visarão identificar classes de problemas onde a
heurística Cuthill-McKee Reversa pode obter melhor desempenho.

\section*{Agradecimentos}
Este trabalho foi financiado pelo CNPq através do projeto PIBIC.


% Discussão / Conclusões
%     Descrição dos dados à luz da literatura
%     Descrição de possíveis fontes de erro e seu efeito sobre os dados
%     Se seus experimentos falharam, quais as sugestões para corrigir o
%     problema?
\input{conclusoes@relatorio_parcial.tex}

% Matéria encaminhada para publicação
%     Quando houver, referir resumos ou artigos científicos publicados ou
%     encaminhados para publicação
\input{publicacoes@relatorio_parcial.tex}

% Perspectivas de continuidade ou desdobramento do trabalho
%     O projeto foi concluído ou será continuado?
% \section{Trabalhos Futuros}
Como continuação deste trabalho, o Método Cuthill-McKee Reverso será implmentado
em C de forma integrada ao PCx para uma melhor comparação da redução do envelope.


% Outras atividades de interesse universitário
%     Descrever participações em congressos, cursos extra-curriculares, etc
\section{Outras atividades}
No período de bolsa o aluno participou dos seguintes treinamentos:
\begin{itemize}
  \item ``Introdução ao MPI'', de 23 a 26/04/2013 com a carga horário de 15
    horas,
  \item ``Introdução ao OpenMP'', de 14 a 16/05/2013 com carga horária de 09
    horas.
\end{itemize}

O aluno também teve o trabalho de iniciação pré-selecionado para o Prêmio de
Iniciação Científica do XLV Simpósio Brasileiro de Pesquisa Operacional como um
dos cinco melhores.


% TODO Para a versao final, comentar as linhas a seguir.
\appendix
% Apoio
%     Citar as agências que financiaram o projeto
% Filename: info.tex
% This code is part of 'CNPq 126874/2012-3'.
% 
% Description: Informa\c{c}\~{o}es.
% 
% Created: 20.08.12 07:27:58 PM
% Last Change: 20.08.12 07:27:58 PM
% 
% Author: Raniere Silva, <r.gaia.cs@gmail.com>
% 
% Copyright (c) 2012, Raniere Silva. All rights reserved.
% 
% This file is license under the terms of the GNU General Public License as published by the Free Software Foundation, either version 3 of the License, or (at your option) any later version. More details at <http://www.gnu.org/licenses/>
%
\section{Informa\c{c}\~{o}es}
Este trabalho foi financiado pelo Conselho Nacional de Desenvolvimento Cient\'{i}fico e Tecnol\'{o}gico pelo Processo 126874/2012-3.

Este trabalho \'{e} licenciado sob a Licen\c{c}a Creative Commons Atribui\c{c}\~{a}o 3.0 N\~{a}o Adaptada License. Para ver uma c\'{o}pia desta licen\c{c}a, visite \url{http://creativecommons.org/licenses/by/3.0/}.
\begin{center}
    \includegraphics{figures/cc-by.png}
\end{center}

%% Copyright (C) 2012 Raniere Silva
% 
% This file is part of 'CNPq 126874/2012-3'.
% 
% 'CNPq 126874/2012-3' is licensed under the Creative Commons
% Attribution 3.0 Unported License. To view a copy of this license,
% visit http://creativecommons.org/licenses/by/3.0/.
% 
% 'CNPq 126874/2012-3' is distributed in the hope that it will be
% useful, but WITHOUT ANY WARRANTY; without even the implied warranty of
% MERCHANTABILITY or FITNESS FOR A PARTICULAR PURPOSE.

\section{FAQ}
\begin{enumerate}
    \item \textbf{Justifique o problema de reduzir a largura de banda ser
        convertido em encontrar uma renumeração dos vértices do grafo tal que
        a diferença entre os índices seja mínima?}
        \cite{Fernanda:2005:ReordenacaoCCCG}

        O ato de permutar linhas e colunas de uma matrix corresponde a renumerar
        os nós de um grafo. \cite{Gibbs:1976:ReducingBandwidth}

    \item \textbf{Por que nós de alta excentricidade produzem bons
        resultados?} \cite{Fernanda:2005:ReordenacaoCCCG}

    \item \textbf{Por que um limitante inferior para a medida a largura de
        banda de $P A P^t$ para qualquer permutação $P$ é dado pelo menor
        inteiro maior ou igual a $D/2$, onde $D$ é o grau máximo de
        qualquer nó do grafo $G(A)$?} \cite{Cuthill:1969:ReducingBandwidth}

        Tomando o nó de grau máximo a maneira de obter a menor largura de
        banda para ele é posicionando-o na diagonal de modo que metade dos nós
        adjacentes esteja de um lado da banda e a outra metade do outro lado.

    \item \textbf{Por que iniciar com um nó de grau mínimo é uma boa escolha?}
        \cite{Cuthill:1969:ReducingBandwidth}

    \item \textbf{Por que numerar os nós em ordem crescente de grau?}
        \cite{Cuthill:1969:ReducingBandwidth}


\end{enumerate}

%% Copyright (C) 2012 Raniere Silva
% 
% This file is part of 'CNPq 126874/2012-3'.
% 
% 'CNPq 126874/2012-3' is licensed under the Creative Commons
% Attribution 3.0 Unported License. To view a copy of this license,
% visit http://creativecommons.org/licenses/by/3.0/.
% 
% 'CNPq 126874/2012-3' is distributed in the hope that it will be
% useful, but WITHOUT ANY WARRANTY; without even the implied warranty of
% MERCHANTABILITY or FITNESS FOR A PARTICULAR PURPOSE.

\section{Implementações}
A seguir enumeramos algumas das implementações disponíveis.

\subsection{C}
\subsubsection{Independente}
David Fritzsche, implementou o RCM para sua tese de mestrado (``Graph
Theoretical Methods for Preconditioners'', defendida em 2004 no
Departamento de Matemática da Bergische Universität Wuppertal. 

Disponível em \url{http://math.temple.edu/~daffi/software/rcm/}.

\subsection{C++}
\subsubsection{The Boost Graph Library (BGL)}
É uma biblioteca para trabalhar com grafos.

Encontra-se disponível em
\url{http://www.boost.org/doc/libs/1_51_0/libs/graph/doc/index.html}

\subsection{Python}
\subsubsection{PyOrder}
É uma biblioteca para reordenamento de matrizes esparsas.

Encontra-se disponível em \url{https://github.com/dpo/pyorder}.

\subsubsection{NetworkX}
É uma biblioteca para trabalhar com grafos.

Encontra-se disponível em \url{http://networkx.lanl.gov/index.html} e a
implementação do RCM em
\url{http://networkx.lanl.gov/examples/algorithms/rcm.html?highlight=cuthill}


% Bibliografia
%     Diversos formatos: definir qual o mais apropriado
%     IMPORTANTE
%     - Não liste se não citar.
%     - Não cite se não listar.
\bibliographystyle{siam}
\bibliography{../references}

% Tabela aqui devido ao limite do número de páginas.
\begin{landscape}
\begin{table}
  \caption{Resultados experimentais.}
  \label{tab:resul2}
  \begin{tabular}{|l|r|r|r|r|r|r|r|r|}
\hline
\multicolumn{9}{|c|}{Problema da biblioteca Netlib} \\ \hline
\multicolumn{1}{|c|}{Problema} & \multicolumn{4}{|c|}{MMD} &         \multicolumn{4}{|c|}{RCM} \\ \hline
\multicolumn{1}{|c|}{Nome} & \multicolumn{1}{|c|}{R} &
        \multicolumn{1}{|c|}{NNZ} & \multicolumn{1}{|c|}{IT} &
        \multicolumn{1}{|c|}{T} & \multicolumn{1}{|c|}{R} &
        \multicolumn{1}{|c|}{NNZ} & \multicolumn{1}{|c|}{IT} &
        \multicolumn{1}{|c|}{T} \\ \hline
dfl001 & 0 & 1,64E+06 & 45 & 2,44E+01 & 15 & 5,72E+06 & 60 & 3,47E+02 \\ \hline
fit2p & 0 & 3,00E+03 & 20 & 5,40E-01 & 0 & 3,00E+03 & 20 & 5,80E-01 \\ \hline
80bau3b & 0 & 4,14E+04 & 36 & 3,10E-01 & 0 & 2,30E+05 & 36 & 1,44E+00 \\ \hline
maros-r7 & 0 & 5,34E+05 & 14 & 1,61E+00 & 0 & 4,40E+05 & 14 & 4,10E+00 \\ \hline
pilot87 & 0 & 4,26E+05 & 29 & 3,35E+00 & 0 & 7,71E+05 & 25 & 7,71E+00 \\ \hline
d2q06c & 0 & 1,37E+05 & 27 & 5,40E-01 & 0 & 4,55E+05 & 24 & 3,01E+00 \\ \hline
greenbea & 14 & 4,91E+04 & 50 & 4,40E-01 & 14 & 1,51E+05 & 49 & 1,52E+00 \\ \hline
greenbeb & 14 & 4,78E+04 & 40 & 3,50E-01 & 14 & 1,68E+05 & 39 & 1,18E+00 \\ \hline
bnl2 & 0 & 8,13E+04 & 33 & 3,10E-01 & 0 & 2,34E+05 & 33 & 1,34E+00 \\ \hline
pilot & 0 & 2,01E+05 & 34 & 1,30E+00 & 0 & 4,36E+05 & 30 & 4,18E+00 \\ \hline
stocfor2 & 0 & 2,28E+04 & 19 & 7,00E-02 & 0 & 3,26E+04 & 19 & 2,40E-01 \\ \hline
sctap3 & 0 & 1,61E+04 & 14 & 6,00E-02 & 0 & 1,03E+05 & 14 & 2,60E-01 \\ \hline
cycle & 0 & 5,61E+04 & 23 & 1,60E-01 & 0 & 1,78E+05 & 23 & 8,00E-01 \\ \hline
degen3 & 0 & 1,21E+05 & 16 & 5,00E-01 & 0 & 5,31E+05 & 13 & 2,53E+00 \\ \hline
woodw & 0 & 3,00E+04 & 30 & 1,80E-01 & 0 & 1,27E+05 & 30 & 5,80E-01 \\ \hline
sierra & 0 & 1,29E+04 & 18 & 6,00E-02 & 0 & 4,36E+04 & 18 & 1,40E-01 \\ \hline
ship12l & 0 & 5,51E+03 & 15 & 4,00E-02 & 0 & 1,70E+04 & 15 & 5,00E-02 \\ \hline
sctap2 & 0 & 1,17E+04 & 12 & 3,00E-02 & 0 & 5,54E+04 & 12 & 1,10E-01 \\ \hline
d6cube & 0 & 5,48E+04 & 17 & 2,20E-01 & 0 & 6,55E+04 & 17 & 2,80E-01 \\ \hline
pilot.we & 0 & 1,56E+04 & 45 & 1,30E-01 & 0 & 3,29E+04 & 45 & 2,10E-01 \\ \hline
nesm & 0 & 2,18E+04 & 25 & 1,20E-01 & 0 & 2,90E+04 & 25 & 1,60E-01 \\ \hline
czprob & 0 & 3,52E+03 & 26 & 4,00E-02 & 0 & 9,13E+04 & 26 & 2,20E-01 \\ \hline
pilotnov & 0 & 4,64E+04 & 16 & 1,10E-01 & 0 & 9,60E+04 & 16 & 2,60E-01 \\ \hline
ganges & 0 & 2,97E+04 & 16 & 6,00E-02 & 0 & 4,88E+04 & 16 & 1,60E-01 \\ \hline
scfxm30 & 0 & 1,41E+04 & 19 & 6,00E-02 & 0 & 2,34E+04 & 19 & 1,30E-01 \\ \hline
\end{tabular}

  \begin{tabular}{|l|r|r|r|r|r|r|r|r|}
\hline
\multicolumn{9}{|c|}{Problema da biblioteca do Mészáros} \\ \hline
\multicolumn{1}{|c|}{Problema} & \multicolumn{4}{|c|}{MMD} &         \multicolumn{4}{|c|}{RCM} \\ \hline
\multicolumn{1}{|c|}{Nome} & \multicolumn{1}{|c|}{R} &
        \multicolumn{1}{|c|}{NNZ} & \multicolumn{1}{|c|}{IT} &
        \multicolumn{1}{|c|}{T} & \multicolumn{1}{|c|}{R} &
        \multicolumn{1}{|c|}{NNZ} & \multicolumn{1}{|c|}{IT} &
        \multicolumn{1}{|c|}{T} \\ \hline
dbic1 & 0 & 1,86E+06 & 45 & 2,87E+01 & -1 & INF & INF & INF \\ \hline
lpl1 & 0 & 9,61E+05 & 68 & 1,18E+01 & 0 & 5,13E+07 & 52 & 6,78E+03 \\ \hline
world & 14 & 1,11E+06 & 60 & 9,28E+00 & -1 & INF & INF & INF \\ \hline
routing & 0 & 3,11E+06 & 20 & 6,06E+00 & 0 & 1,28E+07 & 17 & 8,76E+01 \\ \hline
baxter\_mat & 0 & 5,89E+06 & 30 & 4,00E+01 & 0 & 2,90E+07 & 27 & 1,05E+03 \\ \hline
zz & 15 & 2,86E+05 & 43 & 3,70E+00 & 15 & 9,93E+05 & 43 & 1,07E+01 \\ \hline
southern1 & 0 & 5,67E+06 & 15 & 6,50E+01 & 0 & 7,77E+06 & 15 & 6,66E+02 \\ \hline
sc205 & 12 & 5,04E+04 & 15 & 5,30E-01 & 12 & 5,36E+04 & 15 & 6,00E-01 \\ \hline
dbir2 & 0 & 2,82E+06 & 26 & 8,40E+01 & 0 & 6,75E+06 & 26 & 3,90E+02 \\ \hline
dbir1 & 0 & 2,52E+06 & 24 & 6,88E+01 & 0 & 5,80E+06 & 23 & 2,86E+02 \\ \hline
nsct2 & 0 & 5,65E+06 & 28 & 2,33E+02 & 0 & 8,35E+06 & 30 & 7,06E+02 \\ \hline
nsct1 & 0 & 5,42E+06 & 21 & 1,71E+02 & 0 & 8,59E+06 & 22 & 6,11E+02 \\ \hline
unilever2 & 15 & 4,23E+05 & 36 & 3,47E+00 & 0 & 5,80E+06 & 73 & 2,47E+02 \\ \hline
olivier & 14 & 4,61E+05 & 50 & 4,57E+00 & 14 & 5,62E+06 & 43 & 4,48E+02 \\ \hline
rlf.pre.dual & 0 & 8,41E+05 & 13 & 2,34E+00 & 0 & 6,10E+06 & 13 & 4,88E+02 \\ \hline
lpl3 & 0 & 1,49E+05 & 36 & 1,12E+00 & 0 & 4,50E+06 & 31 & 1,21E+02 \\ \hline
p10 & 0 & 4,70E+05 & 27 & 1,69E+00 & 0 & 7,78E+06 & 29 & 2,54E+02 \\ \hline
leader & 0 & 3,96E+05 & 49 & 2,86E+00 & 14 & 4,76E+06 & 48 & 5,19E+02 \\ \hline
ps & 15 & 1,11E+07 & 40 & 5,16E+02 & 15 & 1,27E+07 & 48 & 6,23E+02 \\ \hline
dbah00 & 0 & 2,81E+05 & 43 & 1,24E+00 & 0 & 4,85E+06 & 32 & 1,01E+02 \\ \hline
nemswrld & 0 & 7,83E+05 & 42 & 7,80E+00 & 0 & 4,47E+06 & 38 & 9,30E+01 \\ \hline
nemsemm1 & 0 & 2,43E+05 & 57 & 1,57E+01 & 0 & 3,99E+05 & 57 & 1,96E+01 \\ \hline
model11 & 0 & 1,40E+05 & 25 & 6,40E-01 & 0 & 7,73E+05 & 21 & 5,25E+00 \\ \hline
nemsemm2 & 0 & 6,25E+04 & 35 & 8,70E-01 & 0 & 3,81E+05 & 35 & 3,37E+00 \\ \hline
emsdz & 0 & 2,75E+05 & 37 & 1,50E+00 & 0 & 2,78E+06 & 29 & 5,14E+01 \\ \hline
\end{tabular}

\end{table}
\end{landscape}
\end{document}
