% Filename: motivacao.tex
% This code is part of 'CNPq 126874/2012-3'.
% 
% Description: Relat\'{o}rio Parcial.
% 
% Created: 20.08.12 07:27:58 PM
% Last Change: 29.08.12 08:20:00 AM
% 
% Author: Raniere Silva, <r.gaia.cs@gmail.com>
% 
% Copyright (c) 2012, Raniere Silva. All rights reserved.
% 
% This file is license under the terms of the GNU General Public License as published by the Free Software Foundation, either version 3 of the License, or (at your option) any later version. More details at <http://www.gnu.org/licenses/>
%
\section{Método Preditor-Corretor}
% TODO Escrever

\section{Matrizes Esparsas}
Uma matriz é dita esparsa quando a maioria de seus elementos são iguais a zero, ou seja, ela possui relativamente poucos elementos não nulos. Sistemas lineares que surgem na solução de problemas reais possuem, em sua maioria, matrizes esparsas e de ordem elevada. Nestes casos, é proibitivo armazenar toda a matriz por questão de memória do computador e o que se faz na prática é guardar somente os elementos não nulos.
% TODO Incluir exemplos e outras informações

\subsection{Matrizes Simétricas}
% TODO Escrever definição
Uma matrix simétrica é

\begin{figure}[!htb]
    \centering
    \begin{tikzpicture}
        \matrix (A) [matrix of math nodes,%
        left delimiter  = (,%
        right delimiter = )] at (0,0)
        {%
        X & X & 0 & X & 0 \\
        X & X & X & 0 & X \\
        0 & X & X & 0 & 0 \\
        X & 0 & 0 & X & X \\
        0 & X & 0 & X & X \\
        };
        \node[above, shift={(0,.5)}] at (A-1-1) {$1$};
        \node[above, shift={(0,.5)}] at (A-1-2) {$2$};
        \node[above, shift={(0,.5)}] at (A-1-3) {$3$};
        \node[above, shift={(0,.5)}] at (A-1-4) {$4$};
        \node[above, shift={(0,.5)}] at (A-1-5) {$5$};
        \node[right, shift={(1,0)}] at (A-1-5) {$1$};
        \node[right, shift={(1,0)}] at (A-2-5) {$2$};
        \node[right, shift={(1,0)}] at (A-3-5) {$3$};
        \node[right, shift={(1,0)}] at (A-4-5) {$4$};
        \node[right, shift={(1,0)}] at (A-5-5) {$5$};
    \end{tikzpicture}
    \caption{Exemplo de matriz simétrica.}
    \label{fig:exem_matriz_simetrica}
\end{figure}
A largura de banda\footnote{Em inglês, \textit{bandwidth}.}, $\beta$, de uma matriz simétrica $A \in \mathbb{R}^{n \times n}$ é dada pela maior distância de um elemento não nulo à diagonal principal, ou seja:
\begin{align*}
    \beta(A) &= \max_{a_{ij} \neq 0} | i - j |.
\end{align*}

% TODO Escrever melhor
É possível ``generalizar'' a largura de banda para cada linha da matiz simética

O \textit{profile}, $\rho$, da matriz é dado por
\begin{align*}
    \rho(A) = \sum_{i = 1}^n \beta_i(A).
\end{align*}
