\section{Resultados}
Os dados referentes aos testes computacionais encontram-se na
Tabela~\ref{tab:resul}.
\begin{table}
    \centering
    \caption{Resultados experimentais}
    \label{tab:resul}
    \csvautotabular{bench1@relatorio_parcial.csv}
\end{table}
\begin{table}
    \centering
    \csvautotabular{bench2@relatorio_parcial.csv}
\end{table}

A partir dos dados das tabelas anteriores temos que o reordenamento obtido
através do Método Cuthill-McKee 
\begin{itemize}
    \item 71 matrizes (78,02\%) tiveram a largura de banda reduzida,
    \item 7 matrizes (7.69\%) não apresentaram alteração na largura de banda, e
    \item 13 matrizes (14.29\%) tiveram a largura de banda aumentada.
\end{itemize}

Dentre as 7 matrizes que não apresentaram alteração na largura de banda,
\begin{itemize}
    \item 1 matriz teve o envelope reduzido,
    \item 5 matrizes não apresentaram alteração no envelope, e
    \item 1 matriz teve o envelope aumentado.
\end{itemize}

Dentre as matrizes que tiveram a largura de banda aumentada, esse aumento foi
em média de 3\% e no pior dos casos de aproximadamente 10\%.
