\section{Resultados Obtidos e Conclusão}
Os dados referentes aos testes computacionais encontram-se na
Tabela~\ref{tab:resul} sendo que as colunas ``Lar. Banda'', ``Envelope''
referem-se a matriz $A D^{-1} A^T$ e ``Lar. Banda R.'' e ``Envelope R.''
referem-se a matriz $P A D^{-1} A^T P^T$ em que a matriz de permutação $P$ é
obtida pelo Método Cuthill-McKee Reverso.
\begin{table}
    \centering
    \caption{Resultados experimentais do Método Cuthill-McKee Reverso para
    a ``Netlib LP''.}
    \label{tab:resul}
    \csvautotabular{bench1@relatorio_parcial.csv}
\end{table}
\begin{table}
    \centering
    \csvautotabular{bench2@relatorio_parcial.csv}
\end{table}

A partir dos dados da Tabela~\ref{bench:resul} temos que o reordenamento obtido
através do Método Cuthill-McKee Reverso resultou na
\begin{itemize}
    \item redução da largura de banda para 71 matrizes (78,02\%),
    \item manutenção da largura de banda para 7 matrizes (7.69\%),
    \item aumenta da largura de banda para 13 matrizes (14.29\%).
\end{itemize}

Dentre as matrizes que tiveram a largura de banda reduzida, a nova largura de
banda corresponde, em média, a 60\% da largura de banda inicial. Já dentre as
matrizes que tiveram a largura de banda aumentada, a nova largura de banda foi,
em média, 24\% maior que a largura de banda inicial.

Dentre as 7 matrizes que não apresentaram alteração na largura de banda,
\begin{itemize}
    \item 1 matriz teve o envelope reduzido,
    \item 5 matrizes não apresentaram alteração no envelope, e
    \item 1 matriz teve o envelope aumentado.
\end{itemize}

Pelos resultados, concluímos ser vantajoso utilizar o reordenamento obtido pelo
Método Cuthill-McKee Reverso pois este reordenamento reduz a largura de banda
para a grande maioria dos casos, aproximadamente 80\% dos testes, e essa
redução costuma ser significativa, superior a 40\%.
