% Copyright (C) 2012 Raniere Silva
% 
% This file is part of 'CNPq 126874/2012-3'.
% 
% 'CNPq 126874/2012-3' is licensed under the Creative Commons
% Attribution 3.0 Unported License. To view a copy of this license,
% visit http://creativecommons.org/licenses/by/3.0/.
% 
% 'CNPq 126874/2012-3' is distributed in the hope that it will be
% useful, but WITHOUT ANY WARRANTY; without even the implied warranty of
% MERCHANTABILITY or FITNESS FOR A PARTICULAR PURPOSE.

\section{Método Preditor-Corretor}
Consideremos o problema de programação linear na forma padrão:
\begin{align*}
    \text{minimizar } & c^t x \\
    \text{sujeito a } & A x = b, \\
    & x \geq 0,
\end{align*}
onde $A \in \mathbb{R}^{m \times n}$ é uma matriz de posto completo $m$ e $c$,
$b$ e $x$ são vetores colunas de dimensão apropriada. Associado a este problema
temos o problema dual:
\begin{align*}
    \text{maximizar } & b^t y \\
    \text{sujeito a } & A^t y + z = c, \\
    & z \geq 0,
\end{align*}
onde $y$ é um vetor coluna de dimensão $m$ de variáveis livres e $z$ é o vetor
coluna de dimensão $n$ de variáveis de folga duais. O \textit{gap} dual é dado
por $\gamma = c^t x - b^t y$ que se reduz a $\gamma = x^t z$ para pontos primais
e duais factíveis.

A direção afim nos métodos de pontos interiores primais-duais é dada por:
\begin{align}
    \begin{bmatrix}
         0 & I & A^t \\
         Z & X & 0 \\
         A & 0 & 0
     \end{bmatrix} \begin{bmatrix}
         \tilde{x} \\
         \tilde{z} \\
         \tilde{y}
     \end{bmatrix} &= \begin{bmatrix}
         r_d \\
         r_a \\
         r_p
     \end{bmatrix},
     \label{eq:primal_dual_intp:lin_system}
\end{align}
onde $X = \diag(x)$, $Z = \diag(z)$ e os respiduos são dados por $r_p = b - A
x$, $r_d = c - A^t y - z$, $r_a = - X Z e$ e $e$ representa o vetor de uns.

Eliminando as variáveis $\tilde{z}$ de \eqref{eq:primal_dual_intp:lin_system}
obtemos o sistema aumentado:
\begin{align}
    \begin{bmatrix}
        -D & A^t \\
        A & 0
    \end{bmatrix} \begin{bmatrix}
        \tilde{x} \\
        \tilde{y}
    \end{bmatrix} &= \begin{bmatrix}
        r_1 \\
        r_2
    \end{bmatrix},
    \label{eq:primal_dual_intp:aug_lin_system}
\end{align}
onde $D = X^{-1} Z$.

A forma mais utilizada para resolver \eqref{eq:primal_dual_intp:aug_lin_system}
consiste em reduzir o sistema através de eliminção das variáveis $\tilde{x}$ a
um sistema simétrico definido positivo com a matriz de equações normais $A
D^{-1} A^t$ e aplicar então a fatoração de Cholesky.

\section{Matrizes Esparsas}
Uma matriz é dita esparsa quando a maioria de seus elementos são iguais a zero, ou seja, ela possui relativamente poucos elementos não nulos. Sistemas lineares que surgem na solução de problemas reais possuem, em sua maioria, matrizes esparsas e de ordem elevada. Nestes casos, é proibitivo armazenar toda a matriz por questão de memória do computador e o que se faz na prática é guardar somente os elementos não nulos.
% TODO Incluir exemplos e outras informações

\subsection{Matrizes Simétricas}
% TODO Escrever definição
Uma matrix simétrica é

\begin{figure}[!htb]
    \centering
    \begin{tikzpicture}
        \matrix (A) [matrix of math nodes,%
        left delimiter  = (,%
        right delimiter = )] at (0,0)
        {%
        X & X & 0 & X & 0 \\
        X & X & X & 0 & X \\
        0 & X & X & 0 & 0 \\
        X & 0 & 0 & X & X \\
        0 & X & 0 & X & X \\
        };
        \node[above, shift={(0,.5)}] at (A-1-1) {$1$};
        \node[above, shift={(0,.5)}] at (A-1-2) {$2$};
        \node[above, shift={(0,.5)}] at (A-1-3) {$3$};
        \node[above, shift={(0,.5)}] at (A-1-4) {$4$};
        \node[above, shift={(0,.5)}] at (A-1-5) {$5$};
        \node[right, shift={(1,0)}] at (A-1-5) {$1$};
        \node[right, shift={(1,0)}] at (A-2-5) {$2$};
        \node[right, shift={(1,0)}] at (A-3-5) {$3$};
        \node[right, shift={(1,0)}] at (A-4-5) {$4$};
        \node[right, shift={(1,0)}] at (A-5-5) {$5$};
    \end{tikzpicture}
    \caption{Exemplo de matriz simétrica.}
    \label{fig:exem_matriz_simetrica}
\end{figure}
A largura de banda\footnote{Em inglês, \textit{bandwidth}.}, $\beta$, de uma matriz simétrica $A \in \mathbb{R}^{n \times n}$ é dada pela maior distância de um elemento não nulo à diagonal principal, ou seja:
\begin{align*}
    \beta(A) &= \max_{a_{ij} \neq 0} | i - j |.
\end{align*}

% TODO Escrever melhor
É possível ``generalizar'' a largura de banda para cada linha da matiz simética

O \textit{profile}, $\rho$, da matriz é dado por
\begin{align*}
    \rho(A) = \sum_{i = 1}^n \beta_i(A).
\end{align*}
\subsection{Grafos e matrizes esparsas}
A teoria dos grafos foi identificada como uma poderosa ferramenta para a computação de matrizes esparsas quando Seymour Parter [Par61] usou grafos não direcionados para modelar a eliminação Gaussiana há mais de quarenta anos atrás. Grafos podem ser usados para modelar matrizes simítricas, não simétricas, fatorações etc. Além de tornar mais fácil a compreensão e análise de algoritmos para matrizes esparsas, eles também aumentam o escopo de manipulações destas matrizes usando algoritmos e técnicas já existentes.

Um grafo é, fundamentalmente, um modo de representar uma relação binária entre objetos. Para o propósito deste trabalho, considere um grafo $G = (V, E)$ como um conjunto de vértices $V = \{v_1, v_2, \ldots \}$ (ou nós) e um conjunto de arestas $E = \{e_1, e_2, \ldots \}$. Estas arestas são representadas por pares não ordenados, por exemplo, $e_1 = (v_1 , v_2)$.

Um conhecimento básico de teoria dos grafos é importante para o entendimento do trabalho. Conceitos mais específicos serão introduzidos quando necessários. Um resumo dos principais conceitos é dado por:
\begin{description}
    \item[Grau do vértice] número de arestas incidentes no vértice.
    \item[Vértices adjacentes]  dois vértices $v_1$ e $v_2$ são adjacentes quando possue uma aresta entre eles, ou seja, $e_1 = (v_1, v_2)$.
    \item[Caminho] sequência de arestas disjuntas $\left( (x_1, x_2), (x_2, x_3), \ldots (x_{k - 1}, x_k) \right)$.
    \item[Grafo conectado (conexo)] possui caminho entre qualquer par de vértices.
    \item[Subgrafo completo (clique)] cada vértice do subgrafo possue aresta incidente em todos os outros vértices do subgrafo.
    \item[Árvore] grafo conectado sem ciclos.
\end{description}

Assim como um grafo, uma matriz tambm descreve uma relação binária entre variáveis através de seus elementos não nulos. Uma matriz simétrica $A_{n \times n}$ pode ser modelada como um grafo $G(V, E)$, onde os $n$ vértices em $V$ representam as dimensões da matriz e existe uma aresta conectando os vértices $i$ e $j$ se $a_{ij} = 0$.
% TODO Escrever
