% Copyright (C) 2012 Raniere Silva
% 
% This file is part of 'CNPq 126874/2012-3'.
% 
% 'CNPq 126874/2012-3' is licensed under the Creative Commons
% Attribution 3.0 Unported License. To view a copy of this license,
% visit http://creativecommons.org/licenses/by/3.0/.
% 
% 'CNPq 126874/2012-3' is distributed in the hope that it will be
% useful, but WITHOUT ANY WARRANTY; without even the implied warranty of
% MERCHANTABILITY or FITNESS FOR A PARTICULAR PURPOSE.

\documentclass[12pt,a4paper]{article}
% Copyright (C) 2012 Raniere Silva
% 
% This file is part of 'CNPq 126874/2012-3'.
% 
% 'CNPq 126874/2012-3' is licensed under the Creative Commons
% Attribution 3.0 Unported License. To view a copy of this license,
% visit http://creativecommons.org/licenses/by/3.0/.
% 
% 'CNPq 126874/2012-3' is distributed in the hope that it will be
% useful, but WITHOUT ANY WARRANTY; without even the implied warranty of
% MERCHANTABILITY or FITNESS FOR A PARTICULAR PURPOSE.

%Pacotes utilizados
\usepackage[top=3cm,left=2cm,right=2cm,bottom=3cm]{geometry} %Borda das p\'{a}gina
\usepackage[utf8]{inputenc} %Ser\'{a} utilizado a codifica\c{c}ao UTF8
\usepackage[T1]{fontenc}
%\usepackage[latin1]{inputenc} % Ser\'{a} utilizado a codifica\c{c}ao latin1
\usepackage[brazil]{babel} %Ser\'{a} utilizado o idioma portugu\^{e}s
\usepackage{indentfirst} %Identa\c{c}\~{a}o de linha
\usepackage{setspace} %Espa\c{c}amento entre linas
\usepackage{amsmath, amsfonts, amssymb, amsthm} % Pacote matem\'{a}tico
\DeclareMathOperator*{\diag}{diag}
\usepackage{graphicx} %Pacote para incluir figuras
\usepackage{color} %Pacote para cores
\usepackage{tikz} %Pacote para desenho de figuras
\usetikzlibrary{matrix}
\usepackage{pgfplots} %Pacote para desenho de gr\'{a}ficos
% \usepackage{subfigure} % Pacote para subfiguras
\usepackage{url} % Pacote para url
\usepackage{hyperref} % Pacote para hyperlink
\usepackage{algorithmic} % Pacote para algoritmos
\usepackage{algorithm} % Pacote para algoritmos
\usepackage{listings} % Pacote para c\'{o}digos

%Personalização
%\pgfplotsset{
%height=0.35\textheight,
%width=0.8\textwidth,
%legend style={
%at={(0.3,0.97)},
%anchor=north,
%font=\footnotesize
%},
%xlabel={},
%ylabel={},
%xtick={},
%grid=major,
%}

\hypersetup{
%colorlinks = true,
}

\floatname{algorithm}{Algoritmo}

\algsetup{linenosize=\small}
\renewcommand{\algorithmicrequire}{\textbf{Entrada:}}
\renewcommand{\algorithmicensure}{\textbf{Saída:}}
\renewcommand{\algorithmicend}{\textbf{fim}}
\renewcommand{\algorithmicif}{\textbf{se}}
\renewcommand{\algorithmicthen}{\textbf{ent\~{a}o}}
\renewcommand{\algorithmicelse}{\textbf{caso contr\'{a}rio}}
\renewcommand{\algorithmicendif}{\algorithmicend}
\renewcommand{\algorithmicfor}{\textbf{para}}
\renewcommand{\algorithmicforall}{\textbf{para todo}}
\renewcommand{\algorithmicdo}{\textbf{fa\c{c}a}}
\renewcommand{\algorithmicendfor}{\algorithmicend}
\renewcommand{\algorithmicwhile}{\textbf{enquanto}}
\renewcommand{\algorithmicendwhile}{\algorithmicend}
\renewcommand{\algorithmicrepeat}{\textbf{repita}}
\renewcommand{\algorithmicuntil}{\textbf{at\'{e}}}
\renewcommand{\algorithmicreturn}{\textbf{retorne}}
\renewcommand{\algorithmiccomment}[1]{\hspace{2em}/* #1 */}

\renewcommand{\lstlistingname}{C\'{o}digo}
\lstset{ %
% language=Octave,                % the language of the code
basicstyle=\ttfamily\small,       % the size of the fonts that are used for the code
numbers=left,                   % where to put the line-numbers
numberstyle=\footnotesize,      % the size of the fonts that are used for the line-numbers
stepnumber=5,                   % the step between two line-numbers. If it's 1, each line 
% will be numbered
numbersep=5pt,                  % how far the line-numbers are from the code
% backgroundcolor=\color{white},  % choose the background color. You must add \usepackage{color}
showspaces=false,               % show spaces adding particular underscores
showstringspaces=false,         % underline spaces within strings
showtabs=false,                 % show tabs within strings adding particular underscores
% frame=single,                   % adds a frame around the code
tabsize=4,                      % sets default tabsize to 2 spaces
captionpos=t,                   % sets the caption-position to bottom
breaklines=true,                % sets automatic line breaking
breakatwhitespace=false,        % sets if automatic breaks should only happen at whitespace
caption={\texttt{\lstname}},                 % show the filename of files included with \lstinputlisting;
% also try caption instead of title
% escapeinside={\%*}{*)},         % if you want to add a comment within your code
% morekeywords={#}            % if you want to add more keywords to the set
}

\newtheorem{defi}{Definição}
\newtheorem{prop}{Proposição}

\begin{document}
% Identificação
%     Projeto
%     Bolsista / RA
%     Orientador
%     Local de execução
%     Vigência
\title{Implementa\c{c}\~{a}o eficiente da heur\'{i}stica de reordenamento de Cuthill-McKee Reversa}
\author{Raniere Silva\footnote{\url{ra092767@ime.unicamp.br}} \\ Orientando
\and Aurelio Oliveira\footnote{\url{aurelio@ime.unicamp.br}} \\ Orientador}
\maketitle

\tableofcontents

% Introdução
%     Introdução ao assunto: deve ser bastante geral
%     Informações da literatura: tornam a introdução mais específica ao assunto
%     Colocação da questão estudada: especificar os objetivos do trabalho
%     Atividades desenvolvidas: dar uma idéia geral de como foi desenvolvido o
%     trabalho
\input{introducao@ms877.tex}

% Materiais e Métodos
%     Materiais: citar os equipamentos, reagentes e outros ítens utilizados,
%     informando fabricante ou fornecedor
%     Métodos: descrever os procedimentos detalhados, que possam ser
%     reproduzidos com os materiais e equipamentos descritos
\section{Implementação do Método e Testes Computacionais}
O Método Cuthill-McKee Reverso foi implementado, na linguagem C, pelo autor
deste trabalho como um patch para o PCx
(\url{http://pages.cs.wisc.edu/~swright/PCx/}), e a implementação desenvolvida
encontra-se disponível em \url{https://github.com/r-gaia-cs/PCx.git}.

O PCx é um solver de Programação Linear que utilizar a variante de Mehrotra do
Método Preditor Corretor com a estratégia de correção de Gondzio e o Método do
Mínimo Grau Múltiplo de Liu\cite{George:1981:ComputerSolutionPD} (MMD) para o
reordenamento da matriz $A D^{-1} A^T$.

Testou-se a implementação (tendo como último commit aquele cuja identificação
inicia com 259cf79) com todos os problemas das bibliotecas:
\begin{itemize}
  \item ``Netlib LP'' (\url{http://www.netlib.org/lp/}),
  \item ``Kennington'' (\url{http://www.netlib.org/lp/data/kennington/}),
  \item ``Meszaros'' (\url{http://www.sztaki.hu/~meszaros/public_ftp/lptestset/misc/}),
  \item ``PDS'' (\url{http://plato.asu.edu/ftp/lptestset/pds/}),
  \item ``Rail'' (\url{http://plato.asu.edu/ftp/lptestset/rail/}).
\end{itemize}

Juntando os problemas de todas a bibliotecas, foram utilizados mais de 200
problemas dos quais apenas 4 o RCM apresentou um memor número de elementos não
zeros (3 problemas da Netlib e 1 da Meszaros). Em relação a problemas não
resolvidos, destaca-se os problemas da PDS que o MMD resolveu 5 e o RCM 1
dentre 9 problemas.

Informações adicionas sobre o resultado obtido para cada um dos problemas testados
encontram-se nas Tabelas~\ref{tab:resulken}-\ref{tab:resulrail} sendo que
``R'', ``NNZ'', ``IT'' e ``T''
significam, respectivamente, ``código de retorno''\footnote{Quando diferente de
zero o resultado encontrado não é confiável ou programa foi abortado}, ``número de elementos não
nulos'', ``número de iterações'' e ``tempo computacional'' (em segundos).
%\begin{table}
%  \centering
%  \caption{Resultados experimentais da biblioteca FCTP.}
%  \label{tab:resulfctp}
%  \begin{tabular}{|l|r|r|r|r|r|r|r|r|}
\hline
\multicolumn{1}{|c|}{Problema} & \multicolumn{4}{|c|}{MMD} &         \multicolumn{4}{|c|}{RCM} \\ \hline
\multicolumn{1}{|c|}{Nome} & \multicolumn{1}{|c|}{R} &
        \multicolumn{1}{|c|}{NNZ} & \multicolumn{1}{|c|}{IT} &
        \multicolumn{1}{|c|}{T} & \multicolumn{1}{|c|}{R} &
        \multicolumn{1}{|c|}{NNZ} & \multicolumn{1}{|c|}{IT} &
        \multicolumn{1}{|c|}{T} \\ \hline
bal8x12 & 3 & 0,0000E+00 & 0 & 0,0000E+00 & 3 & 0,0000E+00 & 0 & 0,0000E+00 \\ \hline
bk4x3 & 3 & 0,0000E+00 & 0 & 0,0000E+00 & 3 & 0,0000E+00 & 0 & 0,0000E+00 \\ \hline
gr4x6 & 3 & 0,0000E+00 & 0 & 0,0000E+00 & 3 & 0,0000E+00 & 0 & 0,0000E+00 \\ \hline
n3700 & 3 & 0,0000E+00 & 0 & 0,0000E+00 & 3 & 0,0000E+00 & 0 & 0,0000E+00 \\ \hline
n3701 & 3 & 0,0000E+00 & 0 & 0,0000E+00 & 3 & 0,0000E+00 & 0 & 0,0000E+00 \\ \hline
n3702 & 3 & 0,0000E+00 & 0 & 0,0000E+00 & 3 & 0,0000E+00 & 0 & 0,0000E+00 \\ \hline
n3703 & 3 & 0,0000E+00 & 0 & 0,0000E+00 & 3 & 0,0000E+00 & 0 & 0,0000E+00 \\ \hline
n3704 & 3 & 0,0000E+00 & 0 & 0,0000E+00 & 3 & 0,0000E+00 & 0 & 0,0000E+00 \\ \hline
n3705 & 3 & 0,0000E+00 & 0 & 0,0000E+00 & 3 & 0,0000E+00 & 0 & 0,0000E+00 \\ \hline
n3706 & 3 & 0,0000E+00 & 0 & 0,0000E+00 & 3 & 0,0000E+00 & 0 & 0,0000E+00 \\ \hline
n3707 & 3 & 0,0000E+00 & 0 & 0,0000E+00 & 3 & 0,0000E+00 & 0 & 0,0000E+00 \\ \hline
n3708 & 3 & 0,0000E+00 & 0 & 0,0000E+00 & 3 & 0,0000E+00 & 0 & 0,0000E+00 \\ \hline
n3709 & 3 & 0,0000E+00 & 0 & 0,0000E+00 & 3 & 0,0000E+00 & 0 & 0,0000E+00 \\ \hline
n370a & 3 & 0,0000E+00 & 0 & 0,0000E+00 & 3 & 0,0000E+00 & 0 & 0,0000E+00 \\ \hline
n370b & 3 & 0,0000E+00 & 0 & 0,0000E+00 & 3 & 0,0000E+00 & 0 & 0,0000E+00 \\ \hline
n370c & 3 & 0,0000E+00 & 0 & 0,0000E+00 & 3 & 0,0000E+00 & 0 & 0,0000E+00 \\ \hline
n370d & 3 & 0,0000E+00 & 0 & 0,0000E+00 & 3 & 0,0000E+00 & 0 & 0,0000E+00 \\ \hline
n370e & 3 & 0,0000E+00 & 0 & 0,0000E+00 & 3 & 0,0000E+00 & 0 & 0,0000E+00 \\ \hline
ran10x10a & 3 & 0,0000E+00 & 0 & 0,0000E+00 & 3 & 0,0000E+00 & 0 & 0,0000E+00 \\ \hline
ran10x10b & 3 & 0,0000E+00 & 0 & 0,0000E+00 & 3 & 0,0000E+00 & 0 & 0,0000E+00 \\ \hline
ran10x10c & 3 & 0,0000E+00 & 0 & 0,0000E+00 & 3 & 0,0000E+00 & 0 & 0,0000E+00 \\ \hline
ran10x12 & 3 & 0,0000E+00 & 0 & 0,0000E+00 & 3 & 0,0000E+00 & 0 & 0,0000E+00 \\ \hline
ran10x26 & 3 & 0,0000E+00 & 0 & 0,0000E+00 & 3 & 0,0000E+00 & 0 & 0,0000E+00 \\ \hline
ran12x12 & 3 & 0,0000E+00 & 0 & 0,0000E+00 & 3 & 0,0000E+00 & 0 & 0,0000E+00 \\ \hline
ran12x21 & 3 & 0,0000E+00 & 0 & 0,0000E+00 & 3 & 0,0000E+00 & 0 & 0,0000E+00 \\ \hline
ran13x13 & 3 & 0,0000E+00 & 0 & 0,0000E+00 & 3 & 0,0000E+00 & 0 & 0,0000E+00 \\ \hline
ran14x18 & 3 & 0,0000E+00 & 0 & 0,0000E+00 & 3 & 0,0000E+00 & 0 & 0,0000E+00 \\ \hline
ran16x16 & 3 & 0,0000E+00 & 0 & 0,0000E+00 & 3 & 0,0000E+00 & 0 & 0,0000E+00 \\ \hline
ran17x17 & 3 & 0,0000E+00 & 0 & 0,0000E+00 & 3 & 0,0000E+00 & 0 & 0,0000E+00 \\ \hline
ran4x64 & 3 & 0,0000E+00 & 0 & 0,0000E+00 & 3 & 0,0000E+00 & 0 & 0,0000E+00 \\ \hline
ran6x43 & 3 & 0,0000E+00 & 0 & 0,0000E+00 & 3 & 0,0000E+00 & 0 & 0,0000E+00 \\ \hline
ran8x32 & 3 & 0,0000E+00 & 0 & 0,0000E+00 & 3 & 0,0000E+00 & 0 & 0,0000E+00 \\ \hline
\end{tabular}

%\end{table}
\begin{table}
  \centering
  \caption{Resultados experimentais da biblioteca Kennington.}
  \label{tab:resulken}
  \begin{tabular}{|l|r|r|r|r|r|r|r|r|}
\hline
\multicolumn{1}{|c|}{Problema} & \multicolumn{4}{|c|}{MMD} &         \multicolumn{4}{|c|}{RCM} \\ \hline
Nome & R & NNZ & IT & T & R & NNZ & IT & T \\ \hline
cre-a & 0 & 33212 & 23 & 1,800000e-01 & 0 & 2638057 & 18 & 2,390000e+01 \\ \hline
cre-b & 0 & 248629 & 41 & 2,090000e+00 & 0 & 5399904 & 30 & 1,019200e+02 \\ \hline
cre-c & 0 & 28528 & 25 & 1,500000e-01 & 0 & 342840 & 23 & 1,900000e+00 \\ \hline
cre-d & 0 & 212094 & 41 & 1,730000e+00 & 0 & 5080054 & 30 & 1,085200e+02 \\ \hline
ken-07 & 0 & 10034 & 13 & 4,000000e-02 & 0 & 96033 & 13 & 2,800000e-01 \\ \hline
ken-11 & 0 & 102906 & 20 & 4,500000e-01 & 0 & 10638459 & 15 & 2,906300e+02 \\ \hline
ken-13 & 0 & 298417 & 23 & 1,240000e+00 & 0 & 26535690 & 20 & 1,414100e+03 \\ \hline
ken-18 & 0 & 1928863 & 29 & 8,510000e+00 & -1 & 0 & 0 & 0,000000e+00 \\ \hline
osa-07 & 0 & 28276 & 22 & 5,200000e-01 & 0 & 188983 & 22 & 1,010000e+00 \\ \hline
osa-14 & 0 & 60795 & 25 & 1,390000e+00 & 0 & 856962 & 24 & 6,740000e+00 \\ \hline
osa-30 & 0 & 115081 & 24 & 2,980000e+00 & 0 & 3126238 & 19 & 3,554000e+01 \\ \hline
osa-60 & 0 & 265909 & 33 & 1,089000e+01 & 0 & 15942080 & 20 & 4,913100e+02 \\ \hline
pds-02 & 0 & 44301 & 24 & 1,900000e-01 & 0 & 768328 & 20 & 5,130000e+00 \\ \hline
pds-06 & 0 & 589339 & 31 & 4,390000e+00 & 0 & 7715856 & 27 & 2,452800e+02 \\ \hline
pds-10 & 0 & 1687660 & 33 & 2,066000e+01 & 0 & 15180087 & 31 & 6,343200e+02 \\ \hline
pds-20 & 0 & 7089645 & 43 & 1,884300e+02 & 0 & 41662111 & 41 & 2,888740e+03 \\ \hline
\end{tabular}

\end{table}
\begin{table}
  \centering
  \caption{Resultados experimentais da biblioteca Meszaros (1/3).}
  \label{tab:resulmes0}
  \begin{tabular}{|l|r|r|r|r|r|r|r|r|}
\hline
\multicolumn{1}{|c|}{Problema} & \multicolumn{4}{|c|}{MMD} &         \multicolumn{4}{|c|}{RCM} \\ \hline
Nome & R & NNZ & IT & T & R & NNZ & IT & T \\ \hline
3-mix & 0 & 541 & 16 & 1,000000e-02 & 0 & 613 & 16 & 1,000000e-02 \\ \hline
aa01,mps & 3 & 0 & 0 & 0,000000e+00 & 3 & 0 & 0 & 0,000000e+00 \\ \hline
aa03,mps & 3 & 0 & 0 & 0,000000e+00 & 3 & 0 & 0 & 0,000000e+00 \\ \hline
air02 & 0 & 1180 & 18 & 2,700000e-01 & 0 & 1190 & 18 & 3,000000e-01 \\ \hline
air03 & 0 & 5303 & 34 & 7,100000e-01 & 0 & 5861 & 34 & 7,200000e-01 \\ \hline
air04 & 0 & 206536 & 76 & 3,770000e+00 & 0 & 254342 & 96 & 7,010000e+00 \\ \hline
air05 & 0 & 62864 & 33 & 6,100000e-01 & 0 & 71846 & 25 & 5,500000e-01 \\ \hline
air06 & 15 & 185860 & 53 & 2,320000e+00 & 0 & 248122 & 28 & 2,160000e+00 \\ \hline
aircraft & 14 & 3754 & 10 & 1,200000e-01 & 14 & 3754 & 10 & 1,800000e-01 \\ \hline
aramco & 0 & 95 & 9 & 0,000000e+00 & 0 & 118 & 9 & 0,000000e+00 \\ \hline
bas1lp & 0 & 2216754 & 9 & 1,854000e+01 & 0 & 3282517 & 9 & 5,188000e+01 \\ \hline
baxter-mat & 0 & 5893406 & 30 & 3,996000e+01 & 0 & 29028747 & 27 & 1,053690e+03 \\ \hline
bpmpd & 6 & 0 & 0 & 0,000000e+00 & 6 & 0 & 0 & 0,000000e+00 \\ \hline
complex1 & 14 & 471404 & 16 & 4,030000e+00 & 15 & 491909 & 37 & 1,014000e+01 \\ \hline
cr42 & 14 & 1804 & 15 & 4,000000e-02 & 14 & 1804 & 15 & 4,000000e-02 \\ \hline
crew & 0 & 8588 & 12 & 1,600000e-01 & 0 & 8823 & 12 & 1,600000e-01 \\ \hline
dano3mip & 12 & 1090198 & 34 & 1,360000e+01 & 12 & 3446554 & 32 & 7,868000e+01 \\ \hline
dbah00 & 0 & 280991 & 43 & 1,240000e+00 & 0 & 4849864 & 32 & 1,005100e+02 \\ \hline
dbic1 & 0 & 1863063 & 45 & 2,870000e+01 & -1 & 0 & 0 & 0,000000e+00 \\ \hline
dbir1 & 0 & 2523535 & 24 & 6,881000e+01 & 0 & 5797420 & 23 & 2,858900e+02 \\ \hline
dbir2 & 0 & 2818493 & 26 & 8,401000e+01 & 0 & 6749933 & 26 & 3,902200e+02 \\ \hline
delfland & 0 & 3461 & 46 & 4,000000e-02 & 0 & 12234 & 46 & 8,000000e-02 \\ \hline
disp3,mps & 3 & 0 & 0 & 0,000000e+00 & 3 & 0 & 0 & 0,000000e+00 \\ \hline
ds90 & 0 & 99700 & 28 & 3,700000e-01 & 0 & 1121565 & 21 & 9,920000e+00 \\ \hline
dsbmip & 0 & 16789 & 48 & 1,600000e-01 & 0 & 59151 & 48 & 4,300000e-01 \\ \hline
emsdz & 0 & 274969 & 37 & 1,500000e+00 & 0 & 2783193 & 29 & 5,140000e+01 \\ \hline
f2177 & 0 & 12932313 & 2 & 2,852000e+01 & 0 & 25059597 & 2 & 1,009500e+02 \\ \hline
from-lp-file & -1 & 0 & 0 & 0,000000e+00 & -1 & 0 & 0 & 0,000000e+00 \\ \hline
gamsmod & 0 & 1661 & 3 & 1,000000e-02 & 0 & 5836 & 3 & 0,000000e+00 \\ \hline
iiasa & 0 & 4525 & 15 & 3,000000e-02 & 0 & 4654 & 15 & 4,000000e-02 \\ \hline
indata & 0 & 1000839 & 14 & 5,460000e+00 & 0 & 838329 & 14 & 1,015000e+01 \\ \hline
jendrec1 & 6 & 0 & 0 & 0,000000e+00 & 6 & 0 & 0 & 0,000000e+00 \\ \hline
kl02,mps & 3 & 0 & 0 & 0,000000e+00 & 3 & 0 & 0 & 0,000000e+00 \\ \hline
kleemin3 & 0 & 6 & 5 & 0,000000e+00 & 0 & 6 & 5 & 0,000000e+00 \\ \hline
kleemin4 & 0 & 10 & 6 & 0,000000e+00 & 0 & 10 & 6 & 0,000000e+00 \\ \hline
kleemin5 & 0 & 15 & 8 & 0,000000e+00 & 0 & 15 & 8 & 0,000000e+00 \\ \hline
kleemin6 & 0 & 21 & 8 & 0,000000e+00 & 0 & 21 & 8 & 0,000000e+00 \\ \hline
kleemin7 & 0 & 28 & 8 & 0,000000e+00 & 0 & 28 & 8 & 0,000000e+00 \\ \hline
kleemin8 & 0 & 36 & 9 & 0,000000e+00 & 0 & 36 & 9 & 0,000000e+00 \\ \hline
l09a13l1d & 14 & 6845 & 12 & 2,000000e-02 & 14 & 7091 & 13 & 2,000000e-02 \\ \hline
leader & 0 & 395537 & 49 & 2,860000e+00 & 14 & 4759383 & 48 & 5,188300e+02 \\ \hline
lindo & 0 & 20 & 9 & 0,000000e+00 & 0 & 20 & 9 & 0,000000e+00 \\ \hline
lp22 & 0 & 942726 & 56 & 1,569000e+01 & 0 & 2014931 & 37 & 3,540000e+01 \\ \hline
lpl1 & 0 & 961244 & 68 & 1,176000e+01 & 0 & 51342474 & 52 & 6,778570e+03 \\ \hline
lpl2 & 0 & 42869 & 21 & 1,800000e-01 & 0 & 505912 & 20 & 3,070000e+00 \\ \hline
lpl3 & 0 & 148905 & 36 & 1,120000e+00 & 0 & 4499523 & 31 & 1,212400e+02 \\ \hline
model10 & 0 & 321444 & 45 & 2,080000e+00 & 0 & 1193797 & 37 & 1,464000e+01 \\ \hline
model11 & 0 & 140071 & 25 & 6,400000e-01 & 0 & 773206 & 21 & 5,250000e+00 \\ \hline
model1 & 0 & 6861 & 8 & 1,000000e-02 & 0 & 15093 & 8 & 2,000000e-02 \\ \hline
model2 & 0 & 12167 & 23 & 5,000000e-02 & 0 & 15749 & 23 & 8,000000e-02 \\ \hline
model3 & 5 & 0 & 0 & 0,000000e+00 & 5 & 0 & 0 & 0,000000e+00 \\ \hline
model4 & 0 & 88743 & 35 & 5,000000e-01 & 0 & 185714 & 35 & 1,320000e+00 \\ \hline
model5 & 0 & 140520 & 44 & 1,230000e+00 & 0 & 288074 & 44 & 3,230000e+00 \\ \hline
model6 & 0 & 106345 & 33 & 4,300000e-01 & 0 & 389087 & 33 & 2,140000e+00 \\ \hline
model7 & 0 & 110005 & 39 & 7,100000e-01 & 0 & 256006 & 39 & 2,200000e+00 \\ \hline
model8 & 0 & 223762 & 17 & 4,200000e-01 & 0 & 275797 & 17 & 1,140000e+00 \\ \hline
model9 & 0 & 70839 & 41 & 6,000000e-01 & 0 & 105039 & 42 & 1,440000e+00 \\ \hline
nemsafm & 0 & 1016 & 16 & 1,000000e-02 & 0 & 1324 & 16 & 2,000000e-02 \\ \hline
nemscem & 0 & 2535 & 18 & 2,000000e-02 & 0 & 14749 & 18 & 3,000000e-02 \\ \hline
nemsemm1 & 0 & 242977 & 57 & 1,571000e+01 & 0 & 398667 & 57 & 1,962000e+01 \\ \hline
nemsemm2 & 0 & 62496 & 35 & 8,700000e-01 & 0 & 381403 & 35 & 3,370000e+00 \\ \hline
nemspmm1 & 0 & 114639 & 39 & 7,400000e-01 & 0 & 865600 & 31 & 9,040000e+00 \\ \hline
nemspmm2,in & 0 & 118904 & 44 & 1,010000e+00 & 0 & 714895 & 34 & 7,490000e+00 \\ \hline
nemswrld & 0 & 783118 & 42 & 7,800000e+00 & 0 & 4472021 & 38 & 9,303000e+01 \\ \hline
nsct1 & 0 & 5417976 & 21 & 1,708000e+02 & 0 & 8593501 & 22 & 6,114200e+02 \\ \hline
nsct2 & 0 & 5645609 & 28 & 2,330500e+02 & 0 & 8350198 & 30 & 7,059900e+02 \\ \hline
nsic1 & 0 & 5042 & 10 & 2,000000e-02 & 0 & 7240 & 10 & 3,000000e-02 \\ \hline
nsic2 & 0 & 5516 & 14 & 2,000000e-02 & 0 & 8011 & 14 & 4,000000e-02 \\ \hline
nsir1 & 0 & 303456 & 21 & 3,660000e+00 & 0 & 630388 & 17 & 1,039000e+01 \\ \hline
nsir2 & 0 & 327947 & 30 & 5,940000e+00 & 0 & 657428 & 26 & 1,442000e+01 \\ \hline
nw14,mps & 3 & 0 & 0 & 0,000000e+00 & 3 & 0 & 0 & 0,000000e+00 \\ \hline
olivier & 14 & 461370 & 50 & 4,570000e+00 & 14 & 5619409 & 43 & 4,481400e+02 \\ \hline
or4,mps & 3 & 0 & 0 & 0,000000e+00 & 3 & 0 & 0 & 0,000000e+00 \\ \hline
orna1,mps & 3 & 0 & 0 & 0,000000e+00 & 3 & 0 & 0 & 0,000000e+00 \\ \hline
orna2,mps & 3 & 0 & 0 & 0,000000e+00 & 3 & 0 & 0 & 0,000000e+00 \\ \hline
orna3,mps & 3 & 0 & 0 & 0,000000e+00 & 3 & 0 & 0 & 0,000000e+00 \\ \hline
orna7,mps & 3 & 0 & 0 & 0,000000e+00 & 3 & 0 & 0 & 0,000000e+00 \\ \hline
orswq2,mps & 3 & 0 & 0 & 0,000000e+00 & 3 & 0 & 0 & 0,000000e+00 \\ \hline
p0033 & 0 & 58 & 8 & 0,000000e+00 & 0 & 60 & 8 & 0,000000e+00 \\ \hline
p0040 & 0 & 96 & 9 & 0,000000e+00 & 0 & 96 & 9 & 0,000000e+00 \\ \hline
p0201 & 0 & 2922 & 5 & 1,000000e-02 & 0 & 7816 & 5 & 1,000000e-02 \\ \hline
p0282 & 0 & 9587 & 13 & 2,000000e-02 & 0 & 13065 & 13 & 4,000000e-02 \\ \hline
p0291 & 0 & 4889 & 12 & 1,000000e-02 & 0 & 8705 & 12 & 2,000000e-02 \\ \hline
p0548 & 0 & 1053 & 24 & 2,000000e-02 & 0 & 1189 & 24 & 2,000000e-02 \\ \hline
p05 & 0 & 236471 & 23 & 7,100000e-01 & 0 & 2288707 & 19 & 2,331000e+01 \\ \hline
p10 & 0 & 469621 & 27 & 1,690000e+00 & 0 & 7780830 & 29 & 2,536700e+02 \\ \hline
p12345 & 0 & 6436 & 14 & 8,000000e-02 & 0 & 6436 & 14 & 8,000000e-02 \\ \hline
p19328 & 0 & 16932 & 15 & 2,700000e-01 & 0 & 16932 & 15 & 3,100000e-01 \\ \hline
p19 & 0 & 2971 & 19 & 2,000000e-02 & 0 & 9815 & 19 & 3,000000e-02 \\ \hline
p2756 & 0 & 6524 & 15 & 5,000000e-02 & 0 & 67656 & 15 & 2,000000e-01 \\ \hline
pcb1000 & 0 & 29753 & 23 & 1,800000e-01 & 0 & 394401 & 23 & 1,820000e+00 \\ \hline
pcb3000 & 0 & 101094 & 25 & 5,600000e-01 & 0 & 4355129 & 19 & 4,932000e+01 \\ \hline
primagaz & 0 & 12404 & 13 & 1,100000e-01 & 0 & 12404 & 13 & 1,400000e-01 \\ \hline
progas & 0 & 30946 & 15 & 6,000000e-02 & 0 & 48144 & 15 & 2,300000e-01 \\ \hline
ps & 15 & 11053969 & 40 & 5,160000e+02 & 15 & 12683340 & 48 & 6,234900e+02 \\ \hline
qiu & 0 & 46492 & 5 & 2,000000e-02 & 0 & 102072 & 5 & 8,000000e-02 \\ \hline
r05 & 0 & 433081 & 23 & 1,780000e+00 & 0 & 2125011 & 18 & 2,623000e+01 \\ \hline
radio & -1 & 0 & 0 & 0,000000e+00 & -11 & 0 & 0 & 0,000000e+00 \\ \hline
rlf,pre,dual & 0 & 840782 & 13 & 2,340000e+00 & 0 & 6095091 & 13 & 4,884100e+02 \\ \hline
routing & 0 & 3114301 & 20 & 6,060000e+00 & 0 & 12818787 & 17 & 8,757000e+01 \\ \hline
sc205 & 12 & 50422 & 15 & 5,300000e-01 & 12 & 53625 & 15 & 6,000000e-01 \\ \hline
seymour & 0 & 4514122 & 15 & 3,633000e+01 & 0 & 8216468 & 15 & 1,342300e+02 \\ \hline
slp-tsk & 0 & 4087774 & 19 & 6,486000e+01 & 0 & 4091840 & 19 & 1,251400e+02 \\ \hline
southern1 & 0 & 5670981 & 15 & 6,500000e+01 & 0 & 7765846 & 15 & 6,662800e+02 \\ \hline
sturing & 0 & 18861 & 36 & 1,500000e-01 & 0 & 36899 & 36 & 3,000000e-01 \\ \hline
sws & 0 & 58235 & 10 & 2,100000e-01 & 0 & 466050 & 10 & 5,900000e-01 \\ \hline
t0331-4l & 0 & 201666 & 37 & 6,580000e+00 & 0 & 215559 & 37 & 7,060000e+00 \\ \hline
test & 0 & 6554 & 46 & 2,700000e-01 & 0 & 6554 & 46 & 2,900000e-01 \\ \hline
testps,mod & 0 & 31 & 13 & 0,000000e+00 & 0 & 31 & 13 & 0,000000e+00 \\ \hline
unilever2 & 15 & 422761 & 36 & 3,470000e+00 & 0 & 5804982 & 73 & 2,466400e+02 \\ \hline
us04,mps & 3 & 0 & 0 & 0,000000e+00 & 3 & 0 & 0 & 0,000000e+00 \\ \hline
world & 14 & 1111240 & 60 & 9,280000e+00 & -1 & 0 & 0 & 0,000000e+00 \\ \hline
zed & 0 & 3380 & 9 & 2,000000e-02 & 0 & 3952 & 9 & 2,000000e-02 \\ \hline
zz & 15 & 285700 & 43 & 3,700000e+00 & 15 & 992596 & 43 & 1,070000e+01 \\ \hline
\end{tabular}

\end{table}
\begin{table}
  \centering
  \caption{Resultados experimentais da biblioteca Meszaros (2/3).}
  \label{tab:resulmes1}
  \begin{tabular}{|l|r|r|r|r|r|r|r|r|}
\hline
\multicolumn{1}{|c|}{Problema} & \multicolumn{4}{|c|}{MMD} &         \multicolumn{4}{|c|}{RCM} \\ \hline
Nome & R & NNZ & IT & T & R & NNZ & IT & T \\ \hline
3-mix & 0 & 541 & 16 & 1,000000e-02 & 0 & 613 & 16 & 1,000000e-02 \\ \hline
aa01,mps & 3 & 0 & 0 & 0,000000e+00 & 3 & 0 & 0 & 0,000000e+00 \\ \hline
aa03,mps & 3 & 0 & 0 & 0,000000e+00 & 3 & 0 & 0 & 0,000000e+00 \\ \hline
air02 & 0 & 1180 & 18 & 2,700000e-01 & 0 & 1190 & 18 & 3,000000e-01 \\ \hline
air03 & 0 & 5303 & 34 & 7,100000e-01 & 0 & 5861 & 34 & 7,200000e-01 \\ \hline
air04 & 0 & 206536 & 76 & 3,770000e+00 & 0 & 254342 & 96 & 7,010000e+00 \\ \hline
air05 & 0 & 62864 & 33 & 6,100000e-01 & 0 & 71846 & 25 & 5,500000e-01 \\ \hline
air06 & 15 & 185860 & 53 & 2,320000e+00 & 0 & 248122 & 28 & 2,160000e+00 \\ \hline
aircraft & 14 & 3754 & 10 & 1,200000e-01 & 14 & 3754 & 10 & 1,800000e-01 \\ \hline
aramco & 0 & 95 & 9 & 0,000000e+00 & 0 & 118 & 9 & 0,000000e+00 \\ \hline
bas1lp & 0 & 2216754 & 9 & 1,854000e+01 & 0 & 3282517 & 9 & 5,188000e+01 \\ \hline
baxter-mat & 0 & 5893406 & 30 & 3,996000e+01 & 0 & 29028747 & 27 & 1,053690e+03 \\ \hline
bpmpd & 6 & 0 & 0 & 0,000000e+00 & 6 & 0 & 0 & 0,000000e+00 \\ \hline
complex1 & 14 & 471404 & 16 & 4,030000e+00 & 15 & 491909 & 37 & 1,014000e+01 \\ \hline
cr42 & 14 & 1804 & 15 & 4,000000e-02 & 14 & 1804 & 15 & 4,000000e-02 \\ \hline
crew & 0 & 8588 & 12 & 1,600000e-01 & 0 & 8823 & 12 & 1,600000e-01 \\ \hline
dano3mip & 12 & 1090198 & 34 & 1,360000e+01 & 12 & 3446554 & 32 & 7,868000e+01 \\ \hline
dbah00 & 0 & 280991 & 43 & 1,240000e+00 & 0 & 4849864 & 32 & 1,005100e+02 \\ \hline
dbic1 & 0 & 1863063 & 45 & 2,870000e+01 & -1 & 0 & 0 & 0,000000e+00 \\ \hline
dbir1 & 0 & 2523535 & 24 & 6,881000e+01 & 0 & 5797420 & 23 & 2,858900e+02 \\ \hline
dbir2 & 0 & 2818493 & 26 & 8,401000e+01 & 0 & 6749933 & 26 & 3,902200e+02 \\ \hline
delfland & 0 & 3461 & 46 & 4,000000e-02 & 0 & 12234 & 46 & 8,000000e-02 \\ \hline
disp3,mps & 3 & 0 & 0 & 0,000000e+00 & 3 & 0 & 0 & 0,000000e+00 \\ \hline
ds90 & 0 & 99700 & 28 & 3,700000e-01 & 0 & 1121565 & 21 & 9,920000e+00 \\ \hline
dsbmip & 0 & 16789 & 48 & 1,600000e-01 & 0 & 59151 & 48 & 4,300000e-01 \\ \hline
emsdz & 0 & 274969 & 37 & 1,500000e+00 & 0 & 2783193 & 29 & 5,140000e+01 \\ \hline
f2177 & 0 & 12932313 & 2 & 2,852000e+01 & 0 & 25059597 & 2 & 1,009500e+02 \\ \hline
from-lp-file & -1 & 0 & 0 & 0,000000e+00 & -1 & 0 & 0 & 0,000000e+00 \\ \hline
gamsmod & 0 & 1661 & 3 & 1,000000e-02 & 0 & 5836 & 3 & 0,000000e+00 \\ \hline
iiasa & 0 & 4525 & 15 & 3,000000e-02 & 0 & 4654 & 15 & 4,000000e-02 \\ \hline
indata & 0 & 1000839 & 14 & 5,460000e+00 & 0 & 838329 & 14 & 1,015000e+01 \\ \hline
jendrec1 & 6 & 0 & 0 & 0,000000e+00 & 6 & 0 & 0 & 0,000000e+00 \\ \hline
kl02,mps & 3 & 0 & 0 & 0,000000e+00 & 3 & 0 & 0 & 0,000000e+00 \\ \hline
kleemin3 & 0 & 6 & 5 & 0,000000e+00 & 0 & 6 & 5 & 0,000000e+00 \\ \hline
kleemin4 & 0 & 10 & 6 & 0,000000e+00 & 0 & 10 & 6 & 0,000000e+00 \\ \hline
kleemin5 & 0 & 15 & 8 & 0,000000e+00 & 0 & 15 & 8 & 0,000000e+00 \\ \hline
kleemin6 & 0 & 21 & 8 & 0,000000e+00 & 0 & 21 & 8 & 0,000000e+00 \\ \hline
kleemin7 & 0 & 28 & 8 & 0,000000e+00 & 0 & 28 & 8 & 0,000000e+00 \\ \hline
kleemin8 & 0 & 36 & 9 & 0,000000e+00 & 0 & 36 & 9 & 0,000000e+00 \\ \hline
l09a13l1d & 14 & 6845 & 12 & 2,000000e-02 & 14 & 7091 & 13 & 2,000000e-02 \\ \hline
leader & 0 & 395537 & 49 & 2,860000e+00 & 14 & 4759383 & 48 & 5,188300e+02 \\ \hline
lindo & 0 & 20 & 9 & 0,000000e+00 & 0 & 20 & 9 & 0,000000e+00 \\ \hline
lp22 & 0 & 942726 & 56 & 1,569000e+01 & 0 & 2014931 & 37 & 3,540000e+01 \\ \hline
lpl1 & 0 & 961244 & 68 & 1,176000e+01 & 0 & 51342474 & 52 & 6,778570e+03 \\ \hline
lpl2 & 0 & 42869 & 21 & 1,800000e-01 & 0 & 505912 & 20 & 3,070000e+00 \\ \hline
lpl3 & 0 & 148905 & 36 & 1,120000e+00 & 0 & 4499523 & 31 & 1,212400e+02 \\ \hline
model10 & 0 & 321444 & 45 & 2,080000e+00 & 0 & 1193797 & 37 & 1,464000e+01 \\ \hline
model11 & 0 & 140071 & 25 & 6,400000e-01 & 0 & 773206 & 21 & 5,250000e+00 \\ \hline
model1 & 0 & 6861 & 8 & 1,000000e-02 & 0 & 15093 & 8 & 2,000000e-02 \\ \hline
model2 & 0 & 12167 & 23 & 5,000000e-02 & 0 & 15749 & 23 & 8,000000e-02 \\ \hline
model3 & 5 & 0 & 0 & 0,000000e+00 & 5 & 0 & 0 & 0,000000e+00 \\ \hline
model4 & 0 & 88743 & 35 & 5,000000e-01 & 0 & 185714 & 35 & 1,320000e+00 \\ \hline
model5 & 0 & 140520 & 44 & 1,230000e+00 & 0 & 288074 & 44 & 3,230000e+00 \\ \hline
model6 & 0 & 106345 & 33 & 4,300000e-01 & 0 & 389087 & 33 & 2,140000e+00 \\ \hline
model7 & 0 & 110005 & 39 & 7,100000e-01 & 0 & 256006 & 39 & 2,200000e+00 \\ \hline
model8 & 0 & 223762 & 17 & 4,200000e-01 & 0 & 275797 & 17 & 1,140000e+00 \\ \hline
model9 & 0 & 70839 & 41 & 6,000000e-01 & 0 & 105039 & 42 & 1,440000e+00 \\ \hline
nemsafm & 0 & 1016 & 16 & 1,000000e-02 & 0 & 1324 & 16 & 2,000000e-02 \\ \hline
nemscem & 0 & 2535 & 18 & 2,000000e-02 & 0 & 14749 & 18 & 3,000000e-02 \\ \hline
nemsemm1 & 0 & 242977 & 57 & 1,571000e+01 & 0 & 398667 & 57 & 1,962000e+01 \\ \hline
nemsemm2 & 0 & 62496 & 35 & 8,700000e-01 & 0 & 381403 & 35 & 3,370000e+00 \\ \hline
nemspmm1 & 0 & 114639 & 39 & 7,400000e-01 & 0 & 865600 & 31 & 9,040000e+00 \\ \hline
nemspmm2,in & 0 & 118904 & 44 & 1,010000e+00 & 0 & 714895 & 34 & 7,490000e+00 \\ \hline
nemswrld & 0 & 783118 & 42 & 7,800000e+00 & 0 & 4472021 & 38 & 9,303000e+01 \\ \hline
nsct1 & 0 & 5417976 & 21 & 1,708000e+02 & 0 & 8593501 & 22 & 6,114200e+02 \\ \hline
nsct2 & 0 & 5645609 & 28 & 2,330500e+02 & 0 & 8350198 & 30 & 7,059900e+02 \\ \hline
nsic1 & 0 & 5042 & 10 & 2,000000e-02 & 0 & 7240 & 10 & 3,000000e-02 \\ \hline
nsic2 & 0 & 5516 & 14 & 2,000000e-02 & 0 & 8011 & 14 & 4,000000e-02 \\ \hline
nsir1 & 0 & 303456 & 21 & 3,660000e+00 & 0 & 630388 & 17 & 1,039000e+01 \\ \hline
nsir2 & 0 & 327947 & 30 & 5,940000e+00 & 0 & 657428 & 26 & 1,442000e+01 \\ \hline
nw14,mps & 3 & 0 & 0 & 0,000000e+00 & 3 & 0 & 0 & 0,000000e+00 \\ \hline
olivier & 14 & 461370 & 50 & 4,570000e+00 & 14 & 5619409 & 43 & 4,481400e+02 \\ \hline
or4,mps & 3 & 0 & 0 & 0,000000e+00 & 3 & 0 & 0 & 0,000000e+00 \\ \hline
orna1,mps & 3 & 0 & 0 & 0,000000e+00 & 3 & 0 & 0 & 0,000000e+00 \\ \hline
orna2,mps & 3 & 0 & 0 & 0,000000e+00 & 3 & 0 & 0 & 0,000000e+00 \\ \hline
orna3,mps & 3 & 0 & 0 & 0,000000e+00 & 3 & 0 & 0 & 0,000000e+00 \\ \hline
orna7,mps & 3 & 0 & 0 & 0,000000e+00 & 3 & 0 & 0 & 0,000000e+00 \\ \hline
orswq2,mps & 3 & 0 & 0 & 0,000000e+00 & 3 & 0 & 0 & 0,000000e+00 \\ \hline
p0033 & 0 & 58 & 8 & 0,000000e+00 & 0 & 60 & 8 & 0,000000e+00 \\ \hline
p0040 & 0 & 96 & 9 & 0,000000e+00 & 0 & 96 & 9 & 0,000000e+00 \\ \hline
p0201 & 0 & 2922 & 5 & 1,000000e-02 & 0 & 7816 & 5 & 1,000000e-02 \\ \hline
p0282 & 0 & 9587 & 13 & 2,000000e-02 & 0 & 13065 & 13 & 4,000000e-02 \\ \hline
p0291 & 0 & 4889 & 12 & 1,000000e-02 & 0 & 8705 & 12 & 2,000000e-02 \\ \hline
p0548 & 0 & 1053 & 24 & 2,000000e-02 & 0 & 1189 & 24 & 2,000000e-02 \\ \hline
p05 & 0 & 236471 & 23 & 7,100000e-01 & 0 & 2288707 & 19 & 2,331000e+01 \\ \hline
p10 & 0 & 469621 & 27 & 1,690000e+00 & 0 & 7780830 & 29 & 2,536700e+02 \\ \hline
p12345 & 0 & 6436 & 14 & 8,000000e-02 & 0 & 6436 & 14 & 8,000000e-02 \\ \hline
p19328 & 0 & 16932 & 15 & 2,700000e-01 & 0 & 16932 & 15 & 3,100000e-01 \\ \hline
p19 & 0 & 2971 & 19 & 2,000000e-02 & 0 & 9815 & 19 & 3,000000e-02 \\ \hline
p2756 & 0 & 6524 & 15 & 5,000000e-02 & 0 & 67656 & 15 & 2,000000e-01 \\ \hline
pcb1000 & 0 & 29753 & 23 & 1,800000e-01 & 0 & 394401 & 23 & 1,820000e+00 \\ \hline
pcb3000 & 0 & 101094 & 25 & 5,600000e-01 & 0 & 4355129 & 19 & 4,932000e+01 \\ \hline
primagaz & 0 & 12404 & 13 & 1,100000e-01 & 0 & 12404 & 13 & 1,400000e-01 \\ \hline
progas & 0 & 30946 & 15 & 6,000000e-02 & 0 & 48144 & 15 & 2,300000e-01 \\ \hline
ps & 15 & 11053969 & 40 & 5,160000e+02 & 15 & 12683340 & 48 & 6,234900e+02 \\ \hline
qiu & 0 & 46492 & 5 & 2,000000e-02 & 0 & 102072 & 5 & 8,000000e-02 \\ \hline
r05 & 0 & 433081 & 23 & 1,780000e+00 & 0 & 2125011 & 18 & 2,623000e+01 \\ \hline
radio & -1 & 0 & 0 & 0,000000e+00 & -11 & 0 & 0 & 0,000000e+00 \\ \hline
rlf,pre,dual & 0 & 840782 & 13 & 2,340000e+00 & 0 & 6095091 & 13 & 4,884100e+02 \\ \hline
routing & 0 & 3114301 & 20 & 6,060000e+00 & 0 & 12818787 & 17 & 8,757000e+01 \\ \hline
sc205 & 12 & 50422 & 15 & 5,300000e-01 & 12 & 53625 & 15 & 6,000000e-01 \\ \hline
seymour & 0 & 4514122 & 15 & 3,633000e+01 & 0 & 8216468 & 15 & 1,342300e+02 \\ \hline
slp-tsk & 0 & 4087774 & 19 & 6,486000e+01 & 0 & 4091840 & 19 & 1,251400e+02 \\ \hline
southern1 & 0 & 5670981 & 15 & 6,500000e+01 & 0 & 7765846 & 15 & 6,662800e+02 \\ \hline
sturing & 0 & 18861 & 36 & 1,500000e-01 & 0 & 36899 & 36 & 3,000000e-01 \\ \hline
sws & 0 & 58235 & 10 & 2,100000e-01 & 0 & 466050 & 10 & 5,900000e-01 \\ \hline
t0331-4l & 0 & 201666 & 37 & 6,580000e+00 & 0 & 215559 & 37 & 7,060000e+00 \\ \hline
test & 0 & 6554 & 46 & 2,700000e-01 & 0 & 6554 & 46 & 2,900000e-01 \\ \hline
testps,mod & 0 & 31 & 13 & 0,000000e+00 & 0 & 31 & 13 & 0,000000e+00 \\ \hline
unilever2 & 15 & 422761 & 36 & 3,470000e+00 & 0 & 5804982 & 73 & 2,466400e+02 \\ \hline
us04,mps & 3 & 0 & 0 & 0,000000e+00 & 3 & 0 & 0 & 0,000000e+00 \\ \hline
world & 14 & 1111240 & 60 & 9,280000e+00 & -1 & 0 & 0 & 0,000000e+00 \\ \hline
zed & 0 & 3380 & 9 & 2,000000e-02 & 0 & 3952 & 9 & 2,000000e-02 \\ \hline
zz & 15 & 285700 & 43 & 3,700000e+00 & 15 & 992596 & 43 & 1,070000e+01 \\ \hline
\end{tabular}

\end{table}
\begin{table}
  \centering
  \caption{Resultados experimentais da biblioteca Meszaros (3/3).}
  \label{tab:resulmes2}
  \begin{tabular}{|l|r|r|r|r|r|r|r|r|}
\hline
\multicolumn{1}{|c|}{Problema} & \multicolumn{4}{|c|}{MMD} &         \multicolumn{4}{|c|}{RCM} \\ \hline
Nome & R & NNZ & IT & T & R & NNZ & IT & T \\ \hline
3-mix & 0 & 541 & 16 & 1,000000e-02 & 0 & 613 & 16 & 1,000000e-02 \\ \hline
aa01,mps & 3 & 0 & 0 & 0,000000e+00 & 3 & 0 & 0 & 0,000000e+00 \\ \hline
aa03,mps & 3 & 0 & 0 & 0,000000e+00 & 3 & 0 & 0 & 0,000000e+00 \\ \hline
air02 & 0 & 1180 & 18 & 2,700000e-01 & 0 & 1190 & 18 & 3,000000e-01 \\ \hline
air03 & 0 & 5303 & 34 & 7,100000e-01 & 0 & 5861 & 34 & 7,200000e-01 \\ \hline
air04 & 0 & 206536 & 76 & 3,770000e+00 & 0 & 254342 & 96 & 7,010000e+00 \\ \hline
air05 & 0 & 62864 & 33 & 6,100000e-01 & 0 & 71846 & 25 & 5,500000e-01 \\ \hline
air06 & 15 & 185860 & 53 & 2,320000e+00 & 0 & 248122 & 28 & 2,160000e+00 \\ \hline
aircraft & 14 & 3754 & 10 & 1,200000e-01 & 14 & 3754 & 10 & 1,800000e-01 \\ \hline
aramco & 0 & 95 & 9 & 0,000000e+00 & 0 & 118 & 9 & 0,000000e+00 \\ \hline
bas1lp & 0 & 2216754 & 9 & 1,854000e+01 & 0 & 3282517 & 9 & 5,188000e+01 \\ \hline
baxter-mat & 0 & 5893406 & 30 & 3,996000e+01 & 0 & 29028747 & 27 & 1,053690e+03 \\ \hline
bpmpd & 6 & 0 & 0 & 0,000000e+00 & 6 & 0 & 0 & 0,000000e+00 \\ \hline
complex1 & 14 & 471404 & 16 & 4,030000e+00 & 15 & 491909 & 37 & 1,014000e+01 \\ \hline
cr42 & 14 & 1804 & 15 & 4,000000e-02 & 14 & 1804 & 15 & 4,000000e-02 \\ \hline
crew & 0 & 8588 & 12 & 1,600000e-01 & 0 & 8823 & 12 & 1,600000e-01 \\ \hline
dano3mip & 12 & 1090198 & 34 & 1,360000e+01 & 12 & 3446554 & 32 & 7,868000e+01 \\ \hline
dbah00 & 0 & 280991 & 43 & 1,240000e+00 & 0 & 4849864 & 32 & 1,005100e+02 \\ \hline
dbic1 & 0 & 1863063 & 45 & 2,870000e+01 & -1 & 0 & 0 & 0,000000e+00 \\ \hline
dbir1 & 0 & 2523535 & 24 & 6,881000e+01 & 0 & 5797420 & 23 & 2,858900e+02 \\ \hline
dbir2 & 0 & 2818493 & 26 & 8,401000e+01 & 0 & 6749933 & 26 & 3,902200e+02 \\ \hline
delfland & 0 & 3461 & 46 & 4,000000e-02 & 0 & 12234 & 46 & 8,000000e-02 \\ \hline
disp3,mps & 3 & 0 & 0 & 0,000000e+00 & 3 & 0 & 0 & 0,000000e+00 \\ \hline
ds90 & 0 & 99700 & 28 & 3,700000e-01 & 0 & 1121565 & 21 & 9,920000e+00 \\ \hline
dsbmip & 0 & 16789 & 48 & 1,600000e-01 & 0 & 59151 & 48 & 4,300000e-01 \\ \hline
emsdz & 0 & 274969 & 37 & 1,500000e+00 & 0 & 2783193 & 29 & 5,140000e+01 \\ \hline
f2177 & 0 & 12932313 & 2 & 2,852000e+01 & 0 & 25059597 & 2 & 1,009500e+02 \\ \hline
from-lp-file & -1 & 0 & 0 & 0,000000e+00 & -1 & 0 & 0 & 0,000000e+00 \\ \hline
gamsmod & 0 & 1661 & 3 & 1,000000e-02 & 0 & 5836 & 3 & 0,000000e+00 \\ \hline
iiasa & 0 & 4525 & 15 & 3,000000e-02 & 0 & 4654 & 15 & 4,000000e-02 \\ \hline
indata & 0 & 1000839 & 14 & 5,460000e+00 & 0 & 838329 & 14 & 1,015000e+01 \\ \hline
jendrec1 & 6 & 0 & 0 & 0,000000e+00 & 6 & 0 & 0 & 0,000000e+00 \\ \hline
kl02,mps & 3 & 0 & 0 & 0,000000e+00 & 3 & 0 & 0 & 0,000000e+00 \\ \hline
kleemin3 & 0 & 6 & 5 & 0,000000e+00 & 0 & 6 & 5 & 0,000000e+00 \\ \hline
kleemin4 & 0 & 10 & 6 & 0,000000e+00 & 0 & 10 & 6 & 0,000000e+00 \\ \hline
kleemin5 & 0 & 15 & 8 & 0,000000e+00 & 0 & 15 & 8 & 0,000000e+00 \\ \hline
kleemin6 & 0 & 21 & 8 & 0,000000e+00 & 0 & 21 & 8 & 0,000000e+00 \\ \hline
kleemin7 & 0 & 28 & 8 & 0,000000e+00 & 0 & 28 & 8 & 0,000000e+00 \\ \hline
kleemin8 & 0 & 36 & 9 & 0,000000e+00 & 0 & 36 & 9 & 0,000000e+00 \\ \hline
l09a13l1d & 14 & 6845 & 12 & 2,000000e-02 & 14 & 7091 & 13 & 2,000000e-02 \\ \hline
leader & 0 & 395537 & 49 & 2,860000e+00 & 14 & 4759383 & 48 & 5,188300e+02 \\ \hline
lindo & 0 & 20 & 9 & 0,000000e+00 & 0 & 20 & 9 & 0,000000e+00 \\ \hline
lp22 & 0 & 942726 & 56 & 1,569000e+01 & 0 & 2014931 & 37 & 3,540000e+01 \\ \hline
lpl1 & 0 & 961244 & 68 & 1,176000e+01 & 0 & 51342474 & 52 & 6,778570e+03 \\ \hline
lpl2 & 0 & 42869 & 21 & 1,800000e-01 & 0 & 505912 & 20 & 3,070000e+00 \\ \hline
lpl3 & 0 & 148905 & 36 & 1,120000e+00 & 0 & 4499523 & 31 & 1,212400e+02 \\ \hline
model10 & 0 & 321444 & 45 & 2,080000e+00 & 0 & 1193797 & 37 & 1,464000e+01 \\ \hline
model11 & 0 & 140071 & 25 & 6,400000e-01 & 0 & 773206 & 21 & 5,250000e+00 \\ \hline
model1 & 0 & 6861 & 8 & 1,000000e-02 & 0 & 15093 & 8 & 2,000000e-02 \\ \hline
model2 & 0 & 12167 & 23 & 5,000000e-02 & 0 & 15749 & 23 & 8,000000e-02 \\ \hline
model3 & 5 & 0 & 0 & 0,000000e+00 & 5 & 0 & 0 & 0,000000e+00 \\ \hline
model4 & 0 & 88743 & 35 & 5,000000e-01 & 0 & 185714 & 35 & 1,320000e+00 \\ \hline
model5 & 0 & 140520 & 44 & 1,230000e+00 & 0 & 288074 & 44 & 3,230000e+00 \\ \hline
model6 & 0 & 106345 & 33 & 4,300000e-01 & 0 & 389087 & 33 & 2,140000e+00 \\ \hline
model7 & 0 & 110005 & 39 & 7,100000e-01 & 0 & 256006 & 39 & 2,200000e+00 \\ \hline
model8 & 0 & 223762 & 17 & 4,200000e-01 & 0 & 275797 & 17 & 1,140000e+00 \\ \hline
model9 & 0 & 70839 & 41 & 6,000000e-01 & 0 & 105039 & 42 & 1,440000e+00 \\ \hline
nemsafm & 0 & 1016 & 16 & 1,000000e-02 & 0 & 1324 & 16 & 2,000000e-02 \\ \hline
nemscem & 0 & 2535 & 18 & 2,000000e-02 & 0 & 14749 & 18 & 3,000000e-02 \\ \hline
nemsemm1 & 0 & 242977 & 57 & 1,571000e+01 & 0 & 398667 & 57 & 1,962000e+01 \\ \hline
nemsemm2 & 0 & 62496 & 35 & 8,700000e-01 & 0 & 381403 & 35 & 3,370000e+00 \\ \hline
nemspmm1 & 0 & 114639 & 39 & 7,400000e-01 & 0 & 865600 & 31 & 9,040000e+00 \\ \hline
nemspmm2,in & 0 & 118904 & 44 & 1,010000e+00 & 0 & 714895 & 34 & 7,490000e+00 \\ \hline
nemswrld & 0 & 783118 & 42 & 7,800000e+00 & 0 & 4472021 & 38 & 9,303000e+01 \\ \hline
nsct1 & 0 & 5417976 & 21 & 1,708000e+02 & 0 & 8593501 & 22 & 6,114200e+02 \\ \hline
nsct2 & 0 & 5645609 & 28 & 2,330500e+02 & 0 & 8350198 & 30 & 7,059900e+02 \\ \hline
nsic1 & 0 & 5042 & 10 & 2,000000e-02 & 0 & 7240 & 10 & 3,000000e-02 \\ \hline
nsic2 & 0 & 5516 & 14 & 2,000000e-02 & 0 & 8011 & 14 & 4,000000e-02 \\ \hline
nsir1 & 0 & 303456 & 21 & 3,660000e+00 & 0 & 630388 & 17 & 1,039000e+01 \\ \hline
nsir2 & 0 & 327947 & 30 & 5,940000e+00 & 0 & 657428 & 26 & 1,442000e+01 \\ \hline
nw14,mps & 3 & 0 & 0 & 0,000000e+00 & 3 & 0 & 0 & 0,000000e+00 \\ \hline
olivier & 14 & 461370 & 50 & 4,570000e+00 & 14 & 5619409 & 43 & 4,481400e+02 \\ \hline
or4,mps & 3 & 0 & 0 & 0,000000e+00 & 3 & 0 & 0 & 0,000000e+00 \\ \hline
orna1,mps & 3 & 0 & 0 & 0,000000e+00 & 3 & 0 & 0 & 0,000000e+00 \\ \hline
orna2,mps & 3 & 0 & 0 & 0,000000e+00 & 3 & 0 & 0 & 0,000000e+00 \\ \hline
orna3,mps & 3 & 0 & 0 & 0,000000e+00 & 3 & 0 & 0 & 0,000000e+00 \\ \hline
orna7,mps & 3 & 0 & 0 & 0,000000e+00 & 3 & 0 & 0 & 0,000000e+00 \\ \hline
orswq2,mps & 3 & 0 & 0 & 0,000000e+00 & 3 & 0 & 0 & 0,000000e+00 \\ \hline
p0033 & 0 & 58 & 8 & 0,000000e+00 & 0 & 60 & 8 & 0,000000e+00 \\ \hline
p0040 & 0 & 96 & 9 & 0,000000e+00 & 0 & 96 & 9 & 0,000000e+00 \\ \hline
p0201 & 0 & 2922 & 5 & 1,000000e-02 & 0 & 7816 & 5 & 1,000000e-02 \\ \hline
p0282 & 0 & 9587 & 13 & 2,000000e-02 & 0 & 13065 & 13 & 4,000000e-02 \\ \hline
p0291 & 0 & 4889 & 12 & 1,000000e-02 & 0 & 8705 & 12 & 2,000000e-02 \\ \hline
p0548 & 0 & 1053 & 24 & 2,000000e-02 & 0 & 1189 & 24 & 2,000000e-02 \\ \hline
p05 & 0 & 236471 & 23 & 7,100000e-01 & 0 & 2288707 & 19 & 2,331000e+01 \\ \hline
p10 & 0 & 469621 & 27 & 1,690000e+00 & 0 & 7780830 & 29 & 2,536700e+02 \\ \hline
p12345 & 0 & 6436 & 14 & 8,000000e-02 & 0 & 6436 & 14 & 8,000000e-02 \\ \hline
p19328 & 0 & 16932 & 15 & 2,700000e-01 & 0 & 16932 & 15 & 3,100000e-01 \\ \hline
p19 & 0 & 2971 & 19 & 2,000000e-02 & 0 & 9815 & 19 & 3,000000e-02 \\ \hline
p2756 & 0 & 6524 & 15 & 5,000000e-02 & 0 & 67656 & 15 & 2,000000e-01 \\ \hline
pcb1000 & 0 & 29753 & 23 & 1,800000e-01 & 0 & 394401 & 23 & 1,820000e+00 \\ \hline
pcb3000 & 0 & 101094 & 25 & 5,600000e-01 & 0 & 4355129 & 19 & 4,932000e+01 \\ \hline
primagaz & 0 & 12404 & 13 & 1,100000e-01 & 0 & 12404 & 13 & 1,400000e-01 \\ \hline
progas & 0 & 30946 & 15 & 6,000000e-02 & 0 & 48144 & 15 & 2,300000e-01 \\ \hline
ps & 15 & 11053969 & 40 & 5,160000e+02 & 15 & 12683340 & 48 & 6,234900e+02 \\ \hline
qiu & 0 & 46492 & 5 & 2,000000e-02 & 0 & 102072 & 5 & 8,000000e-02 \\ \hline
r05 & 0 & 433081 & 23 & 1,780000e+00 & 0 & 2125011 & 18 & 2,623000e+01 \\ \hline
radio & -1 & 0 & 0 & 0,000000e+00 & -11 & 0 & 0 & 0,000000e+00 \\ \hline
rlf,pre,dual & 0 & 840782 & 13 & 2,340000e+00 & 0 & 6095091 & 13 & 4,884100e+02 \\ \hline
routing & 0 & 3114301 & 20 & 6,060000e+00 & 0 & 12818787 & 17 & 8,757000e+01 \\ \hline
sc205 & 12 & 50422 & 15 & 5,300000e-01 & 12 & 53625 & 15 & 6,000000e-01 \\ \hline
seymour & 0 & 4514122 & 15 & 3,633000e+01 & 0 & 8216468 & 15 & 1,342300e+02 \\ \hline
slp-tsk & 0 & 4087774 & 19 & 6,486000e+01 & 0 & 4091840 & 19 & 1,251400e+02 \\ \hline
southern1 & 0 & 5670981 & 15 & 6,500000e+01 & 0 & 7765846 & 15 & 6,662800e+02 \\ \hline
sturing & 0 & 18861 & 36 & 1,500000e-01 & 0 & 36899 & 36 & 3,000000e-01 \\ \hline
sws & 0 & 58235 & 10 & 2,100000e-01 & 0 & 466050 & 10 & 5,900000e-01 \\ \hline
t0331-4l & 0 & 201666 & 37 & 6,580000e+00 & 0 & 215559 & 37 & 7,060000e+00 \\ \hline
test & 0 & 6554 & 46 & 2,700000e-01 & 0 & 6554 & 46 & 2,900000e-01 \\ \hline
testps,mod & 0 & 31 & 13 & 0,000000e+00 & 0 & 31 & 13 & 0,000000e+00 \\ \hline
unilever2 & 15 & 422761 & 36 & 3,470000e+00 & 0 & 5804982 & 73 & 2,466400e+02 \\ \hline
us04,mps & 3 & 0 & 0 & 0,000000e+00 & 3 & 0 & 0 & 0,000000e+00 \\ \hline
world & 14 & 1111240 & 60 & 9,280000e+00 & -1 & 0 & 0 & 0,000000e+00 \\ \hline
zed & 0 & 3380 & 9 & 2,000000e-02 & 0 & 3952 & 9 & 2,000000e-02 \\ \hline
zz & 15 & 285700 & 43 & 3,700000e+00 & 15 & 992596 & 43 & 1,070000e+01 \\ \hline
\end{tabular}

\end{table}
\begin{table}
  \centering
  \caption{Resultados experimentais da biblioteca Netlib (1/3).}
  \label{tab:resulnet0}
  \begin{tabular}{|l|r|r|r|r|r|r|r|r|}
\hline
\multicolumn{1}{|c|}{Problema} & \multicolumn{4}{|c|}{MMD} &         \multicolumn{4}{|c|}{RCM} \\ \hline
\multicolumn{1}{|c|}{Nome} & \multicolumn{1}{|c|}{R} &
        \multicolumn{1}{|c|}{NNZ} & \multicolumn{1}{|c|}{IT} &
        \multicolumn{1}{|c|}{T} & \multicolumn{1}{|c|}{R} &
        \multicolumn{1}{|c|}{NNZ} & \multicolumn{1}{|c|}{IT} &
        \multicolumn{1}{|c|}{T} \\ \hline
25fv47 & 0 & 3,3809E+04 & 25 & 1,1000E-01 & 0 & 7,0740E+04 & 25 & 2,8000E-01 \\ \hline
80bau3b & 0 & 4,1367E+04 & 36 & 3,1000E-01 & 0 & 2,3030E+05 & 36 & 1,4400E+00 \\ \hline
adlittle & 0 & 4,0400E+02 & 11 & 0,0000E+00 & 0 & 5,0300E+02 & 11 & 0,0000E+00 \\ \hline
afiro & 0 & 1,0700E+02 & 7 & 0,0000E+00 & 0 & 1,6600E+02 & 7 & 0,0000E+00 \\ \hline
agg2 & 0 & 2,1482E+04 & 21 & 6,0000E-02 & 0 & 5,9242E+04 & 21 & 1,7000E-01 \\ \hline
agg3 & 0 & 2,1482E+04 & 19 & 6,0000E-02 & 0 & 5,9242E+04 & 19 & 1,5000E-01 \\ \hline
agg & 0 & 1,2297E+04 & 18 & 3,0000E-02 & 0 & 2,6647E+04 & 18 & 6,0000E-02 \\ \hline
bandm & 0 & 3,9360E+03 & 16 & 2,0000E-02 & 0 & 7,0830E+03 & 16 & 2,0000E-02 \\ \hline
beaconfd & 0 & 8,2000E+02 & 10 & 1,0000E-02 & 0 & 1,0290E+03 & 10 & 0,0000E+00 \\ \hline
blend & 0 & 9,1300E+02 & 9 & 0,0000E+00 & 0 & 1,8130E+03 & 9 & 1,0000E-02 \\ \hline
bnl1 & 0 & 1,2089E+04 & 33 & 6,0000E-02 & 0 & 3,1463E+04 & 33 & 1,4000E-01 \\ \hline
bnl2 & 0 & 8,1275E+04 & 33 & 3,1000E-01 & 0 & 2,3394E+05 & 33 & 1,3400E+00 \\ \hline
boeing1 & 0 & 5,7250E+03 & 18 & 2,0000E-02 & 0 & 3,9025E+04 & 19 & 9,0000E-02 \\ \hline
boeing2 & 0 & 2,0290E+03 & 13 & 1,0000E-02 & 0 & 3,0930E+03 & 13 & 2,0000E-02 \\ \hline
bore3d & 0 & 1,0340E+03 & 15 & 1,0000E-02 & 0 & 1,4080E+03 & 15 & 1,0000E-02 \\ \hline
brandy & 14 & 2,7550E+03 & 17 & 2,0000E-02 & 14 & 3,7080E+03 & 17 & 2,0000E-02 \\ \hline
capri & 0 & 3,9620E+03 & 17 & 2,0000E-02 & 0 & 5,7110E+03 & 17 & 2,0000E-02 \\ \hline
cycle & 0 & 5,6102E+04 & 23 & 1,6000E-01 & 0 & 1,7816E+05 & 23 & 8,0000E-01 \\ \hline
czprob & 0 & 3,5200E+03 & 26 & 4,0000E-02 & 0 & 9,1320E+04 & 26 & 2,2000E-01 \\ \hline
d2q06c & 0 & 1,3735E+05 & 27 & 5,4000E-01 & 0 & 4,5543E+05 & 24 & 3,0100E+00 \\ \hline
d6cube & 0 & 5,4840E+04 & 17 & 2,2000E-01 & 0 & 6,5467E+04 & 17 & 2,8000E-01 \\ \hline
degen2 & 0 & 1,6319E+04 & 11 & 3,0000E-02 & 0 & 4,0896E+04 & 11 & 8,0000E-02 \\ \hline
degen3 & 0 & 1,2091E+05 & 16 & 5,0000E-01 & 0 & 5,3148E+05 & 13 & 2,5300E+00 \\ \hline
dfl001 & 0 & 1,6381E+06 & 45 & 2,4430E+01 & 15 & 5,7201E+06 & 60 & 3,4708E+02 \\ \hline
e226 & 0 & 3,2290E+03 & 18 & 2,0000E-02 & 0 & 7,7150E+03 & 18 & 3,0000E-02 \\ \hline
etamacro & 0 & 1,0843E+04 & 26 & 3,0000E-02 & 0 & 2,3260E+04 & 26 & 7,0000E-02 \\ \hline
fffff800 & 0 & 9,5730E+03 & 29 & 7,0000E-02 & 0 & 3,3386E+04 & 29 & 1,3000E-01 \\ \hline
finnis & 0 & 4,9840E+03 & 22 & 3,0000E-02 & 0 & 1,0466E+04 & 22 & 4,0000E-02 \\ \hline
fit1d & 0 & 2,9600E+02 & 17 & 6,0000E-02 & 0 & 2,9800E+02 & 17 & 6,0000E-02 \\ \hline
fit1p & 0 & 6,2700E+02 & 16 & 7,0000E-02 & 0 & 6,2700E+02 & 16 & 8,0000E-02 \\ \hline
fit2d & 0 & 3,2400E+02 & 22 & 7,1000E-01 & 0 & 3,2500E+02 & 22 & 7,1000E-01 \\ \hline
fit2p & 0 & 3,0000E+03 & 20 & 5,4000E-01 & 0 & 3,0000E+03 & 20 & 5,8000E-01 \\ \hline
forplan & 0 & 3,3040E+03 & 21 & 3,0000E-02 & 0 & 4,3520E+03 & 21 & 3,0000E-02 \\ \hline
ganges & 0 & 2,9677E+04 & 16 & 6,0000E-02 & 0 & 4,8834E+04 & 16 & 1,6000E-01 \\ \hline
gfrd-pnc & 0 & 2,1120E+03 & 16 & 2,0000E-02 & 0 & 3,3160E+03 & 16 & 2,0000E-02 \\ \hline
greenbea & 14 & 4,9055E+04 & 50 & 4,4000E-01 & 14 & 1,5087E+05 & 49 & 1,5200E+00 \\ \hline
greenbeb & 14 & 4,7783E+04 & 40 & 3,5000E-01 & 14 & 1,6769E+05 & 39 & 1,1800E+00 \\ \hline
grow15 & 0 & 6,0900E+03 & 19 & 4,0000E-02 & 0 & 6,0900E+03 & 19 & 5,0000E-02 \\ \hline
grow22 & 0 & 9,0580E+03 & 21 & 6,0000E-02 & 0 & 9,0300E+03 & 21 & 8,0000E-02 \\ \hline
grow7 & 0 & 2,7300E+03 & 16 & 2,0000E-02 & 0 & 2,7300E+03 & 16 & 2,0000E-02 \\ \hline
israel & 0 & 1,1488E+04 & 19 & 4,0000E-02 & 0 & 1,1506E+04 & 19 & 6,0000E-02 \\ \hline
kb2 & 0 & 5,0300E+02 & 11 & 0,0000E+00 & 0 & 7,4700E+02 & 11 & 0,0000E+00 \\ \hline
lotfi & 0 & 1,3690E+03 & 13 & 1,0000E-02 & 0 & 2,6940E+03 & 13 & 1,0000E-02 \\ \hline
maros & 0 & 1,3454E+04 & 18 & 4,0000E-02 & 0 & 3,3456E+04 & 18 & 1,0000E-01 \\ \hline
maros-r7 & 0 & 5,3419E+05 & 14 & 1,6100E+00 & 0 & 4,3984E+05 & 14 & 4,1000E+00 \\ \hline
modszk1 & 0 & 1,0550E+04 & 20 & 3,0000E-02 & 0 & 8,2365E+04 & 20 & 1,7000E-01 \\ \hline
nesm & 0 & 2,1776E+04 & 25 & 1,2000E-01 & 0 & 2,9048E+04 & 25 & 1,6000E-01 \\ \hline
perold & 0 & 2,1782E+04 & 33 & 9,0000E-02 & 0 & 4,4505E+04 & 33 & 2,1000E-01 \\ \hline
pilot4 & 0 & 1,4279E+04 & 46 & 1,4000E-01 & 0 & 2,6482E+04 & 46 & 2,1000E-01 \\ \hline
pilot87 & 0 & 4,2565E+05 & 29 & 3,3500E+00 & 0 & 7,7134E+05 & 25 & 7,7100E+00 \\ \hline
pilot,ja & 0 & 4,7924E+04 & 29 & 2,0000E-01 & 0 & 9,5733E+04 & 29 & 4,8000E-01 \\ \hline
pilot & 0 & 2,0081E+05 & 34 & 1,3000E+00 & 0 & 4,3589E+05 & 30 & 4,1800E+00 \\ \hline
pilotnov & 0 & 4,6353E+04 & 16 & 1,1000E-01 & 0 & 9,5971E+04 & 16 & 2,6000E-01 \\ \hline
pilot,we & 0 & 1,5605E+04 & 45 & 1,3000E-01 & 0 & 3,2919E+04 & 45 & 2,1000E-01 \\ \hline
recipe & 0 & 2,7700E+02 & 8 & 0,0000E+00 & 0 & 2,7800E+02 & 8 & 1,0000E-02 \\ \hline
sc105 & 0 & 5,6900E+02 & 9 & 0,0000E+00 & 0 & 8,2500E+02 & 9 & 0,0000E+00 \\ \hline
sc205 & 0 & 1,1560E+03 & 10 & 1,0000E-02 & 0 & 1,7160E+03 & 10 & 1,0000E-02 \\ \hline
sc50a & 0 & 2,4200E+02 & 7 & 0,0000E+00 & 0 & 3,2600E+02 & 7 & 0,0000E+00 \\ \hline
sc50b & 0 & 2,3100E+02 & 6 & 0,0000E+00 & 0 & 3,6100E+02 & 6 & 0,0000E+00 \\ \hline
scagr25 & 0 & 2,9440E+03 & 17 & 2,0000E-02 & 0 & 4,2580E+03 & 17 & 3,0000E-02 \\ \hline
scagr7 & 0 & 7,3000E+02 & 13 & 1,0000E-02 & 0 & 1,0360E+03 & 13 & 1,0000E-02 \\ \hline
scfxm1 & 0 & 4,4300E+03 & 17 & 2,0000E-02 & 0 & 7,4190E+03 & 17 & 3,0000E-02 \\ \hline
scfxm2 & 14 & 9,2840E+03 & 19 & 4,0000E-02 & 14 & 1,5458E+04 & 19 & 7,0000E-02 \\ \hline
scfxm3 & 0 & 1,4138E+04 & 19 & 6,0000E-02 & 0 & 2,3442E+04 & 19 & 1,3000E-01 \\ \hline
scorpion & 0 & 2,2750E+03 & 11 & 1,0000E-02 & 0 & 4,9260E+03 & 11 & 2,0000E-02 \\ \hline
scrs8 & 0 & 4,4770E+03 & 21 & 2,0000E-02 & 0 & 8,3040E+03 & 21 & 3,0000E-02 \\ \hline
scsd1 & 0 & 1,3920E+03 & 8 & 1,0000E-02 & 0 & 1,4720E+03 & 8 & 0,0000E+00 \\ \hline
scsd6 & 0 & 2,5450E+03 & 11 & 2,0000E-02 & 0 & 2,7400E+03 & 11 & 2,0000E-02 \\ \hline
scsd8 & 0 & 5,8790E+03 & 10 & 3,0000E-02 & 0 & 5,9420E+03 & 10 & 3,0000E-02 \\ \hline
sctap1 & 0 & 2,4350E+03 & 14 & 1,0000E-02 & 0 & 5,3670E+03 & 14 & 2,0000E-02 \\ \hline
sctap2 & 0 & 1,1736E+04 & 12 & 3,0000E-02 & 0 & 5,5394E+04 & 12 & 1,1000E-01 \\ \hline
sctap3 & 0 & 1,6100E+04 & 14 & 6,0000E-02 & 0 & 1,0257E+05 & 14 & 2,6000E-01 \\ \hline
seba & 0 & 5,3711E+04 & 13 & 1,7000E-01 & 0 & 5,5513E+04 & 13 & 4,0000E-01 \\ \hline
share1b & 0 & 1,4170E+03 & 18 & 1,0000E-02 & 0 & 1,7620E+03 & 18 & 2,0000E-02 \\ \hline
share2b & 0 & 1,0410E+03 & 17 & 1,0000E-02 & 0 & 1,5420E+03 & 17 & 1,0000E-02 \\ \hline
shell & 0 & 3,9830E+03 & 20 & 2,0000E-02 & 0 & 9,5390E+03 & 20 & 4,0000E-02 \\ \hline
ship04l & 0 & 2,6410E+03 & 12 & 2,0000E-02 & 0 & 7,9440E+03 & 12 & 2,0000E-02 \\ \hline
ship04s & 0 & 1,7780E+03 & 12 & 1,0000E-02 & 0 & 4,0910E+03 & 12 & 2,0000E-02 \\ \hline
ship08l & 0 & 4,4420E+03 & 14 & 3,0000E-02 & 0 & 1,5947E+04 & 14 & 4,0000E-02 \\ \hline
ship08s & 0 & 2,2950E+03 & 11 & 2,0000E-02 & 0 & 4,2900E+03 & 11 & 2,0000E-02 \\ \hline
ship12l & 0 & 5,5060E+03 & 15 & 4,0000E-02 & 0 & 1,6980E+04 & 15 & 5,0000E-02 \\ \hline
ship12s & 0 & 2,5070E+03 & 12 & 2,0000E-02 & 0 & 3,8010E+03 & 12 & 2,0000E-02 \\ \hline
sierra & 0 & 1,2862E+04 & 18 & 6,0000E-02 & 0 & 4,3552E+04 & 18 & 1,4000E-01 \\ \hline
stair & 0 & 1,4682E+04 & 13 & 3,0000E-02 & 0 & 1,5047E+04 & 13 & 5,0000E-02 \\ \hline
standata & 0 & 2,3950E+03 & 13 & 1,0000E-02 & 0 & 3,9530E+03 & 13 & 2,0000E-02 \\ \hline
standgub & 0 & 2,3950E+03 & 13 & 1,0000E-02 & 0 & 3,9530E+03 & 13 & 1,0000E-02 \\ \hline
standmps & 0 & 3,9570E+03 & 23 & 2,0000E-02 & 0 & 9,1360E+03 & 23 & 3,0000E-02 \\ \hline
stocfor1 & 0 & 8,0500E+02 & 11 & 0,0000E+00 & 0 & 1,3880E+03 & 11 & 1,0000E-02 \\ \hline
stocfor2 & 0 & 2,2841E+04 & 19 & 7,0000E-02 & 0 & 3,2610E+04 & 19 & 2,4000E-01 \\ \hline
tuff & 0 & 7,0510E+03 & 18 & 2,0000E-02 & 0 & 9,8200E+03 & 18 & 4,0000E-02 \\ \hline
vtp,base & 0 & 5,0500E+02 & 10 & 0,0000E+00 & 0 & 5,0400E+02 & 10 & 0,0000E+00 \\ \hline
wood1p & 0 & 1,1645E+04 & 22 & 3,4000E-01 & 0 & 1,4537E+04 & 22 & 3,6000E-01 \\ \hline
woodw & 0 & 3,0027E+04 & 30 & 1,8000E-01 & 0 & 1,2672E+05 & 30 & 5,8000E-01 \\ \hline
\end{tabular}

\end{table}
\begin{table}
  \centering
  \caption{Resultados experimentais da biblioteca Netlib (2/3).}
  \label{tab:resulnet1}
  \begin{tabular}{|l|r|r|r|r|r|r|r|r|}
\hline
\multicolumn{1}{|c|}{Problema} & \multicolumn{4}{|c|}{MMD} &         \multicolumn{4}{|c|}{RCM} \\ \hline
\multicolumn{1}{|c|}{Nome} & \multicolumn{1}{|c|}{R} &
        \multicolumn{1}{|c|}{NNZ} & \multicolumn{1}{|c|}{IT} &
        \multicolumn{1}{|c|}{T} & \multicolumn{1}{|c|}{R} &
        \multicolumn{1}{|c|}{NNZ} & \multicolumn{1}{|c|}{IT} &
        \multicolumn{1}{|c|}{T} \\ \hline
25fv47 & 0 & 3,3809E+04 & 25 & 1,1000E-01 & 0 & 7,0740E+04 & 25 & 2,8000E-01 \\ \hline
80bau3b & 0 & 4,1367E+04 & 36 & 3,1000E-01 & 0 & 2,3030E+05 & 36 & 1,4400E+00 \\ \hline
adlittle & 0 & 4,0400E+02 & 11 & 0,0000E+00 & 0 & 5,0300E+02 & 11 & 0,0000E+00 \\ \hline
afiro & 0 & 1,0700E+02 & 7 & 0,0000E+00 & 0 & 1,6600E+02 & 7 & 0,0000E+00 \\ \hline
agg2 & 0 & 2,1482E+04 & 21 & 6,0000E-02 & 0 & 5,9242E+04 & 21 & 1,7000E-01 \\ \hline
agg3 & 0 & 2,1482E+04 & 19 & 6,0000E-02 & 0 & 5,9242E+04 & 19 & 1,5000E-01 \\ \hline
agg & 0 & 1,2297E+04 & 18 & 3,0000E-02 & 0 & 2,6647E+04 & 18 & 6,0000E-02 \\ \hline
bandm & 0 & 3,9360E+03 & 16 & 2,0000E-02 & 0 & 7,0830E+03 & 16 & 2,0000E-02 \\ \hline
beaconfd & 0 & 8,2000E+02 & 10 & 1,0000E-02 & 0 & 1,0290E+03 & 10 & 0,0000E+00 \\ \hline
blend & 0 & 9,1300E+02 & 9 & 0,0000E+00 & 0 & 1,8130E+03 & 9 & 1,0000E-02 \\ \hline
bnl1 & 0 & 1,2089E+04 & 33 & 6,0000E-02 & 0 & 3,1463E+04 & 33 & 1,4000E-01 \\ \hline
bnl2 & 0 & 8,1275E+04 & 33 & 3,1000E-01 & 0 & 2,3394E+05 & 33 & 1,3400E+00 \\ \hline
boeing1 & 0 & 5,7250E+03 & 18 & 2,0000E-02 & 0 & 3,9025E+04 & 19 & 9,0000E-02 \\ \hline
boeing2 & 0 & 2,0290E+03 & 13 & 1,0000E-02 & 0 & 3,0930E+03 & 13 & 2,0000E-02 \\ \hline
bore3d & 0 & 1,0340E+03 & 15 & 1,0000E-02 & 0 & 1,4080E+03 & 15 & 1,0000E-02 \\ \hline
brandy & 14 & 2,7550E+03 & 17 & 2,0000E-02 & 14 & 3,7080E+03 & 17 & 2,0000E-02 \\ \hline
capri & 0 & 3,9620E+03 & 17 & 2,0000E-02 & 0 & 5,7110E+03 & 17 & 2,0000E-02 \\ \hline
cycle & 0 & 5,6102E+04 & 23 & 1,6000E-01 & 0 & 1,7816E+05 & 23 & 8,0000E-01 \\ \hline
czprob & 0 & 3,5200E+03 & 26 & 4,0000E-02 & 0 & 9,1320E+04 & 26 & 2,2000E-01 \\ \hline
d2q06c & 0 & 1,3735E+05 & 27 & 5,4000E-01 & 0 & 4,5543E+05 & 24 & 3,0100E+00 \\ \hline
d6cube & 0 & 5,4840E+04 & 17 & 2,2000E-01 & 0 & 6,5467E+04 & 17 & 2,8000E-01 \\ \hline
degen2 & 0 & 1,6319E+04 & 11 & 3,0000E-02 & 0 & 4,0896E+04 & 11 & 8,0000E-02 \\ \hline
degen3 & 0 & 1,2091E+05 & 16 & 5,0000E-01 & 0 & 5,3148E+05 & 13 & 2,5300E+00 \\ \hline
dfl001 & 0 & 1,6381E+06 & 45 & 2,4430E+01 & 15 & 5,7201E+06 & 60 & 3,4708E+02 \\ \hline
e226 & 0 & 3,2290E+03 & 18 & 2,0000E-02 & 0 & 7,7150E+03 & 18 & 3,0000E-02 \\ \hline
etamacro & 0 & 1,0843E+04 & 26 & 3,0000E-02 & 0 & 2,3260E+04 & 26 & 7,0000E-02 \\ \hline
fffff800 & 0 & 9,5730E+03 & 29 & 7,0000E-02 & 0 & 3,3386E+04 & 29 & 1,3000E-01 \\ \hline
finnis & 0 & 4,9840E+03 & 22 & 3,0000E-02 & 0 & 1,0466E+04 & 22 & 4,0000E-02 \\ \hline
fit1d & 0 & 2,9600E+02 & 17 & 6,0000E-02 & 0 & 2,9800E+02 & 17 & 6,0000E-02 \\ \hline
fit1p & 0 & 6,2700E+02 & 16 & 7,0000E-02 & 0 & 6,2700E+02 & 16 & 8,0000E-02 \\ \hline
fit2d & 0 & 3,2400E+02 & 22 & 7,1000E-01 & 0 & 3,2500E+02 & 22 & 7,1000E-01 \\ \hline
fit2p & 0 & 3,0000E+03 & 20 & 5,4000E-01 & 0 & 3,0000E+03 & 20 & 5,8000E-01 \\ \hline
forplan & 0 & 3,3040E+03 & 21 & 3,0000E-02 & 0 & 4,3520E+03 & 21 & 3,0000E-02 \\ \hline
ganges & 0 & 2,9677E+04 & 16 & 6,0000E-02 & 0 & 4,8834E+04 & 16 & 1,6000E-01 \\ \hline
gfrd-pnc & 0 & 2,1120E+03 & 16 & 2,0000E-02 & 0 & 3,3160E+03 & 16 & 2,0000E-02 \\ \hline
greenbea & 14 & 4,9055E+04 & 50 & 4,4000E-01 & 14 & 1,5087E+05 & 49 & 1,5200E+00 \\ \hline
greenbeb & 14 & 4,7783E+04 & 40 & 3,5000E-01 & 14 & 1,6769E+05 & 39 & 1,1800E+00 \\ \hline
grow15 & 0 & 6,0900E+03 & 19 & 4,0000E-02 & 0 & 6,0900E+03 & 19 & 5,0000E-02 \\ \hline
grow22 & 0 & 9,0580E+03 & 21 & 6,0000E-02 & 0 & 9,0300E+03 & 21 & 8,0000E-02 \\ \hline
grow7 & 0 & 2,7300E+03 & 16 & 2,0000E-02 & 0 & 2,7300E+03 & 16 & 2,0000E-02 \\ \hline
israel & 0 & 1,1488E+04 & 19 & 4,0000E-02 & 0 & 1,1506E+04 & 19 & 6,0000E-02 \\ \hline
kb2 & 0 & 5,0300E+02 & 11 & 0,0000E+00 & 0 & 7,4700E+02 & 11 & 0,0000E+00 \\ \hline
lotfi & 0 & 1,3690E+03 & 13 & 1,0000E-02 & 0 & 2,6940E+03 & 13 & 1,0000E-02 \\ \hline
maros & 0 & 1,3454E+04 & 18 & 4,0000E-02 & 0 & 3,3456E+04 & 18 & 1,0000E-01 \\ \hline
maros-r7 & 0 & 5,3419E+05 & 14 & 1,6100E+00 & 0 & 4,3984E+05 & 14 & 4,1000E+00 \\ \hline
modszk1 & 0 & 1,0550E+04 & 20 & 3,0000E-02 & 0 & 8,2365E+04 & 20 & 1,7000E-01 \\ \hline
nesm & 0 & 2,1776E+04 & 25 & 1,2000E-01 & 0 & 2,9048E+04 & 25 & 1,6000E-01 \\ \hline
perold & 0 & 2,1782E+04 & 33 & 9,0000E-02 & 0 & 4,4505E+04 & 33 & 2,1000E-01 \\ \hline
pilot4 & 0 & 1,4279E+04 & 46 & 1,4000E-01 & 0 & 2,6482E+04 & 46 & 2,1000E-01 \\ \hline
pilot87 & 0 & 4,2565E+05 & 29 & 3,3500E+00 & 0 & 7,7134E+05 & 25 & 7,7100E+00 \\ \hline
pilot,ja & 0 & 4,7924E+04 & 29 & 2,0000E-01 & 0 & 9,5733E+04 & 29 & 4,8000E-01 \\ \hline
pilot & 0 & 2,0081E+05 & 34 & 1,3000E+00 & 0 & 4,3589E+05 & 30 & 4,1800E+00 \\ \hline
pilotnov & 0 & 4,6353E+04 & 16 & 1,1000E-01 & 0 & 9,5971E+04 & 16 & 2,6000E-01 \\ \hline
pilot,we & 0 & 1,5605E+04 & 45 & 1,3000E-01 & 0 & 3,2919E+04 & 45 & 2,1000E-01 \\ \hline
recipe & 0 & 2,7700E+02 & 8 & 0,0000E+00 & 0 & 2,7800E+02 & 8 & 1,0000E-02 \\ \hline
sc105 & 0 & 5,6900E+02 & 9 & 0,0000E+00 & 0 & 8,2500E+02 & 9 & 0,0000E+00 \\ \hline
sc205 & 0 & 1,1560E+03 & 10 & 1,0000E-02 & 0 & 1,7160E+03 & 10 & 1,0000E-02 \\ \hline
sc50a & 0 & 2,4200E+02 & 7 & 0,0000E+00 & 0 & 3,2600E+02 & 7 & 0,0000E+00 \\ \hline
sc50b & 0 & 2,3100E+02 & 6 & 0,0000E+00 & 0 & 3,6100E+02 & 6 & 0,0000E+00 \\ \hline
scagr25 & 0 & 2,9440E+03 & 17 & 2,0000E-02 & 0 & 4,2580E+03 & 17 & 3,0000E-02 \\ \hline
scagr7 & 0 & 7,3000E+02 & 13 & 1,0000E-02 & 0 & 1,0360E+03 & 13 & 1,0000E-02 \\ \hline
scfxm1 & 0 & 4,4300E+03 & 17 & 2,0000E-02 & 0 & 7,4190E+03 & 17 & 3,0000E-02 \\ \hline
scfxm2 & 14 & 9,2840E+03 & 19 & 4,0000E-02 & 14 & 1,5458E+04 & 19 & 7,0000E-02 \\ \hline
scfxm3 & 0 & 1,4138E+04 & 19 & 6,0000E-02 & 0 & 2,3442E+04 & 19 & 1,3000E-01 \\ \hline
scorpion & 0 & 2,2750E+03 & 11 & 1,0000E-02 & 0 & 4,9260E+03 & 11 & 2,0000E-02 \\ \hline
scrs8 & 0 & 4,4770E+03 & 21 & 2,0000E-02 & 0 & 8,3040E+03 & 21 & 3,0000E-02 \\ \hline
scsd1 & 0 & 1,3920E+03 & 8 & 1,0000E-02 & 0 & 1,4720E+03 & 8 & 0,0000E+00 \\ \hline
scsd6 & 0 & 2,5450E+03 & 11 & 2,0000E-02 & 0 & 2,7400E+03 & 11 & 2,0000E-02 \\ \hline
scsd8 & 0 & 5,8790E+03 & 10 & 3,0000E-02 & 0 & 5,9420E+03 & 10 & 3,0000E-02 \\ \hline
sctap1 & 0 & 2,4350E+03 & 14 & 1,0000E-02 & 0 & 5,3670E+03 & 14 & 2,0000E-02 \\ \hline
sctap2 & 0 & 1,1736E+04 & 12 & 3,0000E-02 & 0 & 5,5394E+04 & 12 & 1,1000E-01 \\ \hline
sctap3 & 0 & 1,6100E+04 & 14 & 6,0000E-02 & 0 & 1,0257E+05 & 14 & 2,6000E-01 \\ \hline
seba & 0 & 5,3711E+04 & 13 & 1,7000E-01 & 0 & 5,5513E+04 & 13 & 4,0000E-01 \\ \hline
share1b & 0 & 1,4170E+03 & 18 & 1,0000E-02 & 0 & 1,7620E+03 & 18 & 2,0000E-02 \\ \hline
share2b & 0 & 1,0410E+03 & 17 & 1,0000E-02 & 0 & 1,5420E+03 & 17 & 1,0000E-02 \\ \hline
shell & 0 & 3,9830E+03 & 20 & 2,0000E-02 & 0 & 9,5390E+03 & 20 & 4,0000E-02 \\ \hline
ship04l & 0 & 2,6410E+03 & 12 & 2,0000E-02 & 0 & 7,9440E+03 & 12 & 2,0000E-02 \\ \hline
ship04s & 0 & 1,7780E+03 & 12 & 1,0000E-02 & 0 & 4,0910E+03 & 12 & 2,0000E-02 \\ \hline
ship08l & 0 & 4,4420E+03 & 14 & 3,0000E-02 & 0 & 1,5947E+04 & 14 & 4,0000E-02 \\ \hline
ship08s & 0 & 2,2950E+03 & 11 & 2,0000E-02 & 0 & 4,2900E+03 & 11 & 2,0000E-02 \\ \hline
ship12l & 0 & 5,5060E+03 & 15 & 4,0000E-02 & 0 & 1,6980E+04 & 15 & 5,0000E-02 \\ \hline
ship12s & 0 & 2,5070E+03 & 12 & 2,0000E-02 & 0 & 3,8010E+03 & 12 & 2,0000E-02 \\ \hline
sierra & 0 & 1,2862E+04 & 18 & 6,0000E-02 & 0 & 4,3552E+04 & 18 & 1,4000E-01 \\ \hline
stair & 0 & 1,4682E+04 & 13 & 3,0000E-02 & 0 & 1,5047E+04 & 13 & 5,0000E-02 \\ \hline
standata & 0 & 2,3950E+03 & 13 & 1,0000E-02 & 0 & 3,9530E+03 & 13 & 2,0000E-02 \\ \hline
standgub & 0 & 2,3950E+03 & 13 & 1,0000E-02 & 0 & 3,9530E+03 & 13 & 1,0000E-02 \\ \hline
standmps & 0 & 3,9570E+03 & 23 & 2,0000E-02 & 0 & 9,1360E+03 & 23 & 3,0000E-02 \\ \hline
stocfor1 & 0 & 8,0500E+02 & 11 & 0,0000E+00 & 0 & 1,3880E+03 & 11 & 1,0000E-02 \\ \hline
stocfor2 & 0 & 2,2841E+04 & 19 & 7,0000E-02 & 0 & 3,2610E+04 & 19 & 2,4000E-01 \\ \hline
tuff & 0 & 7,0510E+03 & 18 & 2,0000E-02 & 0 & 9,8200E+03 & 18 & 4,0000E-02 \\ \hline
vtp,base & 0 & 5,0500E+02 & 10 & 0,0000E+00 & 0 & 5,0400E+02 & 10 & 0,0000E+00 \\ \hline
wood1p & 0 & 1,1645E+04 & 22 & 3,4000E-01 & 0 & 1,4537E+04 & 22 & 3,6000E-01 \\ \hline
woodw & 0 & 3,0027E+04 & 30 & 1,8000E-01 & 0 & 1,2672E+05 & 30 & 5,8000E-01 \\ \hline
\end{tabular}

\end{table}
\begin{table}
  \centering
  \caption{Resultados experimentais da biblioteca Netlib (3/3).}
  \label{tab:resulnet2}
  \begin{tabular}{|l|r|r|r|r|r|r|r|r|}
\hline
\multicolumn{1}{|c|}{Problema} & \multicolumn{4}{|c|}{MMD} &         \multicolumn{4}{|c|}{RCM} \\ \hline
\multicolumn{1}{|c|}{Nome} & \multicolumn{1}{|c|}{R} &
        \multicolumn{1}{|c|}{NNZ} & \multicolumn{1}{|c|}{IT} &
        \multicolumn{1}{|c|}{T} & \multicolumn{1}{|c|}{R} &
        \multicolumn{1}{|c|}{NNZ} & \multicolumn{1}{|c|}{IT} &
        \multicolumn{1}{|c|}{T} \\ \hline
25fv47 & 0 & 3,3809E+04 & 25 & 1,1000E-01 & 0 & 7,0740E+04 & 25 & 2,8000E-01 \\ \hline
80bau3b & 0 & 4,1367E+04 & 36 & 3,1000E-01 & 0 & 2,3030E+05 & 36 & 1,4400E+00 \\ \hline
adlittle & 0 & 4,0400E+02 & 11 & 0,0000E+00 & 0 & 5,0300E+02 & 11 & 0,0000E+00 \\ \hline
afiro & 0 & 1,0700E+02 & 7 & 0,0000E+00 & 0 & 1,6600E+02 & 7 & 0,0000E+00 \\ \hline
agg2 & 0 & 2,1482E+04 & 21 & 6,0000E-02 & 0 & 5,9242E+04 & 21 & 1,7000E-01 \\ \hline
agg3 & 0 & 2,1482E+04 & 19 & 6,0000E-02 & 0 & 5,9242E+04 & 19 & 1,5000E-01 \\ \hline
agg & 0 & 1,2297E+04 & 18 & 3,0000E-02 & 0 & 2,6647E+04 & 18 & 6,0000E-02 \\ \hline
bandm & 0 & 3,9360E+03 & 16 & 2,0000E-02 & 0 & 7,0830E+03 & 16 & 2,0000E-02 \\ \hline
beaconfd & 0 & 8,2000E+02 & 10 & 1,0000E-02 & 0 & 1,0290E+03 & 10 & 0,0000E+00 \\ \hline
blend & 0 & 9,1300E+02 & 9 & 0,0000E+00 & 0 & 1,8130E+03 & 9 & 1,0000E-02 \\ \hline
bnl1 & 0 & 1,2089E+04 & 33 & 6,0000E-02 & 0 & 3,1463E+04 & 33 & 1,4000E-01 \\ \hline
bnl2 & 0 & 8,1275E+04 & 33 & 3,1000E-01 & 0 & 2,3394E+05 & 33 & 1,3400E+00 \\ \hline
boeing1 & 0 & 5,7250E+03 & 18 & 2,0000E-02 & 0 & 3,9025E+04 & 19 & 9,0000E-02 \\ \hline
boeing2 & 0 & 2,0290E+03 & 13 & 1,0000E-02 & 0 & 3,0930E+03 & 13 & 2,0000E-02 \\ \hline
bore3d & 0 & 1,0340E+03 & 15 & 1,0000E-02 & 0 & 1,4080E+03 & 15 & 1,0000E-02 \\ \hline
brandy & 14 & 2,7550E+03 & 17 & 2,0000E-02 & 14 & 3,7080E+03 & 17 & 2,0000E-02 \\ \hline
capri & 0 & 3,9620E+03 & 17 & 2,0000E-02 & 0 & 5,7110E+03 & 17 & 2,0000E-02 \\ \hline
cycle & 0 & 5,6102E+04 & 23 & 1,6000E-01 & 0 & 1,7816E+05 & 23 & 8,0000E-01 \\ \hline
czprob & 0 & 3,5200E+03 & 26 & 4,0000E-02 & 0 & 9,1320E+04 & 26 & 2,2000E-01 \\ \hline
d2q06c & 0 & 1,3735E+05 & 27 & 5,4000E-01 & 0 & 4,5543E+05 & 24 & 3,0100E+00 \\ \hline
d6cube & 0 & 5,4840E+04 & 17 & 2,2000E-01 & 0 & 6,5467E+04 & 17 & 2,8000E-01 \\ \hline
degen2 & 0 & 1,6319E+04 & 11 & 3,0000E-02 & 0 & 4,0896E+04 & 11 & 8,0000E-02 \\ \hline
degen3 & 0 & 1,2091E+05 & 16 & 5,0000E-01 & 0 & 5,3148E+05 & 13 & 2,5300E+00 \\ \hline
dfl001 & 0 & 1,6381E+06 & 45 & 2,4430E+01 & 15 & 5,7201E+06 & 60 & 3,4708E+02 \\ \hline
e226 & 0 & 3,2290E+03 & 18 & 2,0000E-02 & 0 & 7,7150E+03 & 18 & 3,0000E-02 \\ \hline
etamacro & 0 & 1,0843E+04 & 26 & 3,0000E-02 & 0 & 2,3260E+04 & 26 & 7,0000E-02 \\ \hline
fffff800 & 0 & 9,5730E+03 & 29 & 7,0000E-02 & 0 & 3,3386E+04 & 29 & 1,3000E-01 \\ \hline
finnis & 0 & 4,9840E+03 & 22 & 3,0000E-02 & 0 & 1,0466E+04 & 22 & 4,0000E-02 \\ \hline
fit1d & 0 & 2,9600E+02 & 17 & 6,0000E-02 & 0 & 2,9800E+02 & 17 & 6,0000E-02 \\ \hline
fit1p & 0 & 6,2700E+02 & 16 & 7,0000E-02 & 0 & 6,2700E+02 & 16 & 8,0000E-02 \\ \hline
fit2d & 0 & 3,2400E+02 & 22 & 7,1000E-01 & 0 & 3,2500E+02 & 22 & 7,1000E-01 \\ \hline
fit2p & 0 & 3,0000E+03 & 20 & 5,4000E-01 & 0 & 3,0000E+03 & 20 & 5,8000E-01 \\ \hline
forplan & 0 & 3,3040E+03 & 21 & 3,0000E-02 & 0 & 4,3520E+03 & 21 & 3,0000E-02 \\ \hline
ganges & 0 & 2,9677E+04 & 16 & 6,0000E-02 & 0 & 4,8834E+04 & 16 & 1,6000E-01 \\ \hline
gfrd-pnc & 0 & 2,1120E+03 & 16 & 2,0000E-02 & 0 & 3,3160E+03 & 16 & 2,0000E-02 \\ \hline
greenbea & 14 & 4,9055E+04 & 50 & 4,4000E-01 & 14 & 1,5087E+05 & 49 & 1,5200E+00 \\ \hline
greenbeb & 14 & 4,7783E+04 & 40 & 3,5000E-01 & 14 & 1,6769E+05 & 39 & 1,1800E+00 \\ \hline
grow15 & 0 & 6,0900E+03 & 19 & 4,0000E-02 & 0 & 6,0900E+03 & 19 & 5,0000E-02 \\ \hline
grow22 & 0 & 9,0580E+03 & 21 & 6,0000E-02 & 0 & 9,0300E+03 & 21 & 8,0000E-02 \\ \hline
grow7 & 0 & 2,7300E+03 & 16 & 2,0000E-02 & 0 & 2,7300E+03 & 16 & 2,0000E-02 \\ \hline
israel & 0 & 1,1488E+04 & 19 & 4,0000E-02 & 0 & 1,1506E+04 & 19 & 6,0000E-02 \\ \hline
kb2 & 0 & 5,0300E+02 & 11 & 0,0000E+00 & 0 & 7,4700E+02 & 11 & 0,0000E+00 \\ \hline
lotfi & 0 & 1,3690E+03 & 13 & 1,0000E-02 & 0 & 2,6940E+03 & 13 & 1,0000E-02 \\ \hline
maros & 0 & 1,3454E+04 & 18 & 4,0000E-02 & 0 & 3,3456E+04 & 18 & 1,0000E-01 \\ \hline
maros-r7 & 0 & 5,3419E+05 & 14 & 1,6100E+00 & 0 & 4,3984E+05 & 14 & 4,1000E+00 \\ \hline
modszk1 & 0 & 1,0550E+04 & 20 & 3,0000E-02 & 0 & 8,2365E+04 & 20 & 1,7000E-01 \\ \hline
nesm & 0 & 2,1776E+04 & 25 & 1,2000E-01 & 0 & 2,9048E+04 & 25 & 1,6000E-01 \\ \hline
perold & 0 & 2,1782E+04 & 33 & 9,0000E-02 & 0 & 4,4505E+04 & 33 & 2,1000E-01 \\ \hline
pilot4 & 0 & 1,4279E+04 & 46 & 1,4000E-01 & 0 & 2,6482E+04 & 46 & 2,1000E-01 \\ \hline
pilot87 & 0 & 4,2565E+05 & 29 & 3,3500E+00 & 0 & 7,7134E+05 & 25 & 7,7100E+00 \\ \hline
pilot,ja & 0 & 4,7924E+04 & 29 & 2,0000E-01 & 0 & 9,5733E+04 & 29 & 4,8000E-01 \\ \hline
pilot & 0 & 2,0081E+05 & 34 & 1,3000E+00 & 0 & 4,3589E+05 & 30 & 4,1800E+00 \\ \hline
pilotnov & 0 & 4,6353E+04 & 16 & 1,1000E-01 & 0 & 9,5971E+04 & 16 & 2,6000E-01 \\ \hline
pilot,we & 0 & 1,5605E+04 & 45 & 1,3000E-01 & 0 & 3,2919E+04 & 45 & 2,1000E-01 \\ \hline
recipe & 0 & 2,7700E+02 & 8 & 0,0000E+00 & 0 & 2,7800E+02 & 8 & 1,0000E-02 \\ \hline
sc105 & 0 & 5,6900E+02 & 9 & 0,0000E+00 & 0 & 8,2500E+02 & 9 & 0,0000E+00 \\ \hline
sc205 & 0 & 1,1560E+03 & 10 & 1,0000E-02 & 0 & 1,7160E+03 & 10 & 1,0000E-02 \\ \hline
sc50a & 0 & 2,4200E+02 & 7 & 0,0000E+00 & 0 & 3,2600E+02 & 7 & 0,0000E+00 \\ \hline
sc50b & 0 & 2,3100E+02 & 6 & 0,0000E+00 & 0 & 3,6100E+02 & 6 & 0,0000E+00 \\ \hline
scagr25 & 0 & 2,9440E+03 & 17 & 2,0000E-02 & 0 & 4,2580E+03 & 17 & 3,0000E-02 \\ \hline
scagr7 & 0 & 7,3000E+02 & 13 & 1,0000E-02 & 0 & 1,0360E+03 & 13 & 1,0000E-02 \\ \hline
scfxm1 & 0 & 4,4300E+03 & 17 & 2,0000E-02 & 0 & 7,4190E+03 & 17 & 3,0000E-02 \\ \hline
scfxm2 & 14 & 9,2840E+03 & 19 & 4,0000E-02 & 14 & 1,5458E+04 & 19 & 7,0000E-02 \\ \hline
scfxm3 & 0 & 1,4138E+04 & 19 & 6,0000E-02 & 0 & 2,3442E+04 & 19 & 1,3000E-01 \\ \hline
scorpion & 0 & 2,2750E+03 & 11 & 1,0000E-02 & 0 & 4,9260E+03 & 11 & 2,0000E-02 \\ \hline
scrs8 & 0 & 4,4770E+03 & 21 & 2,0000E-02 & 0 & 8,3040E+03 & 21 & 3,0000E-02 \\ \hline
scsd1 & 0 & 1,3920E+03 & 8 & 1,0000E-02 & 0 & 1,4720E+03 & 8 & 0,0000E+00 \\ \hline
scsd6 & 0 & 2,5450E+03 & 11 & 2,0000E-02 & 0 & 2,7400E+03 & 11 & 2,0000E-02 \\ \hline
scsd8 & 0 & 5,8790E+03 & 10 & 3,0000E-02 & 0 & 5,9420E+03 & 10 & 3,0000E-02 \\ \hline
sctap1 & 0 & 2,4350E+03 & 14 & 1,0000E-02 & 0 & 5,3670E+03 & 14 & 2,0000E-02 \\ \hline
sctap2 & 0 & 1,1736E+04 & 12 & 3,0000E-02 & 0 & 5,5394E+04 & 12 & 1,1000E-01 \\ \hline
sctap3 & 0 & 1,6100E+04 & 14 & 6,0000E-02 & 0 & 1,0257E+05 & 14 & 2,6000E-01 \\ \hline
seba & 0 & 5,3711E+04 & 13 & 1,7000E-01 & 0 & 5,5513E+04 & 13 & 4,0000E-01 \\ \hline
share1b & 0 & 1,4170E+03 & 18 & 1,0000E-02 & 0 & 1,7620E+03 & 18 & 2,0000E-02 \\ \hline
share2b & 0 & 1,0410E+03 & 17 & 1,0000E-02 & 0 & 1,5420E+03 & 17 & 1,0000E-02 \\ \hline
shell & 0 & 3,9830E+03 & 20 & 2,0000E-02 & 0 & 9,5390E+03 & 20 & 4,0000E-02 \\ \hline
ship04l & 0 & 2,6410E+03 & 12 & 2,0000E-02 & 0 & 7,9440E+03 & 12 & 2,0000E-02 \\ \hline
ship04s & 0 & 1,7780E+03 & 12 & 1,0000E-02 & 0 & 4,0910E+03 & 12 & 2,0000E-02 \\ \hline
ship08l & 0 & 4,4420E+03 & 14 & 3,0000E-02 & 0 & 1,5947E+04 & 14 & 4,0000E-02 \\ \hline
ship08s & 0 & 2,2950E+03 & 11 & 2,0000E-02 & 0 & 4,2900E+03 & 11 & 2,0000E-02 \\ \hline
ship12l & 0 & 5,5060E+03 & 15 & 4,0000E-02 & 0 & 1,6980E+04 & 15 & 5,0000E-02 \\ \hline
ship12s & 0 & 2,5070E+03 & 12 & 2,0000E-02 & 0 & 3,8010E+03 & 12 & 2,0000E-02 \\ \hline
sierra & 0 & 1,2862E+04 & 18 & 6,0000E-02 & 0 & 4,3552E+04 & 18 & 1,4000E-01 \\ \hline
stair & 0 & 1,4682E+04 & 13 & 3,0000E-02 & 0 & 1,5047E+04 & 13 & 5,0000E-02 \\ \hline
standata & 0 & 2,3950E+03 & 13 & 1,0000E-02 & 0 & 3,9530E+03 & 13 & 2,0000E-02 \\ \hline
standgub & 0 & 2,3950E+03 & 13 & 1,0000E-02 & 0 & 3,9530E+03 & 13 & 1,0000E-02 \\ \hline
standmps & 0 & 3,9570E+03 & 23 & 2,0000E-02 & 0 & 9,1360E+03 & 23 & 3,0000E-02 \\ \hline
stocfor1 & 0 & 8,0500E+02 & 11 & 0,0000E+00 & 0 & 1,3880E+03 & 11 & 1,0000E-02 \\ \hline
stocfor2 & 0 & 2,2841E+04 & 19 & 7,0000E-02 & 0 & 3,2610E+04 & 19 & 2,4000E-01 \\ \hline
tuff & 0 & 7,0510E+03 & 18 & 2,0000E-02 & 0 & 9,8200E+03 & 18 & 4,0000E-02 \\ \hline
vtp,base & 0 & 5,0500E+02 & 10 & 0,0000E+00 & 0 & 5,0400E+02 & 10 & 0,0000E+00 \\ \hline
wood1p & 0 & 1,1645E+04 & 22 & 3,4000E-01 & 0 & 1,4537E+04 & 22 & 3,6000E-01 \\ \hline
woodw & 0 & 3,0027E+04 & 30 & 1,8000E-01 & 0 & 1,2672E+05 & 30 & 5,8000E-01 \\ \hline
\end{tabular}

\end{table}
\begin{table}
  \centering
  \caption{Resultados experimentais da biblioteca Netlib-QAP.}
  \label{tab:resulnetqap}
  \begin{tabular}{|l|r|r|r|r|r|r|r|r|}
\hline
\multicolumn{1}{|c|}{Problema} & \multicolumn{4}{|c|}{MMD} &         \multicolumn{4}{|c|}{RCM} \\ \hline
\multicolumn{1}{|c|}{Nome} & \multicolumn{1}{|c|}{R} &
        \multicolumn{1}{|c|}{NNZ} & \multicolumn{1}{|c|}{IT} &
        \multicolumn{1}{|c|}{T} & \multicolumn{1}{|c|}{R} &
        \multicolumn{1}{|c|}{NNZ} & \multicolumn{1}{|c|}{IT} &
        \multicolumn{1}{|c|}{T} \\ \hline
qap12 & 15 & 2,1386E+06 & 41 & 3,4270E+01 & 15 & 3,4398E+06 & 42 & 8,6770E+01 \\ \hline
qap15 & 15 & 8,1980E+06 & 46 & 3,3534E+02 & 15 & 1,3674E+07 & 54 & 8,1852E+02 \\ \hline
qap8 & 0 & 1,9394E+05 & 8 & 2,4000E-01 & 0 & 2,7351E+05 & 32 & 1,7200E+00 \\ \hline
\end{tabular}

\end{table}
\begin{table}
  \centering
  \caption{Resultados experimentais da biblioteca PDS.}
  \label{tab:resulpds}
  \begin{tabular}{|l|r|r|r|r|r|r|r|r|}
\hline
\multicolumn{1}{|c|}{Problema} & \multicolumn{4}{|c|}{MMD} &         \multicolumn{4}{|c|}{RCM} \\ \hline
Nome & R & NNZ & IT & T & R & NNZ & IT & T \\ \hline
pds-100 & -1 & 0 & 0 & 0,000000e+00 & -11 & 0 & 0 & 0,000000e+00 \\ \hline
pds-20 & 0 & 7089645 & 43 & 1,915400e+02 & 0 & 41662111 & 41 & 2,895260e+03 \\ \hline
pds-30 & 0 & 0 & 48 & 0,000000e+00 & -1 & 0 & 0 & 0,000000e+00 \\ \hline
pds-40 & 0 & 0 & 50 & 0,000000e+00 & -1 & 0 & 0 & 0,000000e+00 \\ \hline
pds-50 & 0 & 0 & 54 & 0,000000e+00 & -1 & 0 & 0 & 0,000000e+00 \\ \hline
pds-60 & 0 & 58118583 & 53 & 4,963710e+03 & -11 & 0 & 0 & 0,000000e+00 \\ \hline
pds-70 & -1 & 0 & 0 & 0,000000e+00 & -11 & 0 & 0 & 0,000000e+00 \\ \hline
pds-80 & -1 & 0 & 0 & 0,000000e+00 & -11 & 0 & 0 & 0,000000e+00 \\ \hline
pds-90 & -1 & 0 & 0 & 0,000000e+00 & -11 & 0 & 0 & 0,000000e+00 \\ \hline
\end{tabular}

\end{table}
\begin{table}
  \centering
  \caption{Resultados experimentais da biblioteca Rail.}
  \label{tab:resulrail}
  \begin{tabular}{|l|r|r|r|r|r|r|r|r|}
\hline
\multicolumn{1}{|c|}{Problema} & \multicolumn{4}{|c|}{MMD} &         \multicolumn{4}{|c|}{RCM} \\ \hline
Nome & R & NNZ & IT & T & R & NNZ & IT & T \\ \hline
rail2586 & 0 & 1236796 & 90 & 2,384400e+02 & 0 & 1442489 & 90 & 2,624200e+02 \\ \hline
rail4284 & 0 & 5804811 & 71 & 6,398400e+02 & 0 & 7045286 & 71 & 8,488900e+02 \\ \hline
rail507 & 0 & 68847 & 36 & 3,340000e+00 & 0 & 77942 & 30 & 3,630000e+00 \\ \hline
rail516 & 0 & 57220 & 29 & 2,160000e+00 & 0 & 73268 & 29 & 2,370000e+00 \\ \hline
rail582 & 0 & 86529 & 35 & 4,040000e+00 & 0 & 114984 & 35 & 4,380000e+00 \\ \hline
\end{tabular}

\end{table}


% Resultados
%     Descrição dos resultados: deve ser clara e objetiva, resumindo os achados
%     principais que serão detalhados em tabelas e figuras.
%     Ilustrações dos resultados: tabelas e figuras são muito importantes; seu
%     número deve ser o menor possível, e elas devem ser construídas com cuidado
%     para incluir todas as informações necessárias com clareza.
%     Tabelas: devem ser numeradas sequencialmente (Tabela 1, Tabela 2, etc).
%     Seu título deve ser informativo, colocado acima e justificado à esquerda.
%     Notas de rodapé (a, b, c...) podem ser colocados diretamente abaixo da
%     mesma.
%     Figuras (fotos, esquemas, gráficos): devem ser numeradas sequencialmente
%     (Figura 1, Figura 2, etc). Seu título deve ser informativo, colocado
%     abaixo e justificado à esquerda, descrevendo o que é mostrado.
%\input{resultados@ms877.tex}

% Discussão / Conclusões
%     Descrição dos dados à luz da literatura
%     Descrição de possíveis fontes de erro e seu efeito sobre os dados
%     Se seus experimentos falharam, quais as sugestões para corrigir o
%     problema?
\section{Conclusões}
Concluímos, pelo menos para os problemas testados, que a heurística de mínimo
grau múltiplo é superior à heurística de Cuthill e McKee por gerar menos
elementos não nulos na decomposição de Cholesky (levando em conta o tamanho do
envelope).


% Matéria encaminhada para publicação
%     Quando houver, referir resumos ou artigos científicos publicados ou
%     encaminhados para publicação
\input{publicacoes@relatorio_parcial.tex}

% Perspectivas de continuidade ou desdobramento do trabalho
%     O projeto foi concluído ou será continuado?
%\section{Trabalhos Futuros}
Como continuação deste trabalho, será modificado o PCx, um pacote para
problemas de programação linear que possui o Método de Pontos Interiores
Predictor-Corrector implementado e disponibilizado em 
\url{http://pages.cs.wisc.edu/~swright/PCx/}, para utilizar o reordenamento
obtido pelo Método Cuthill-McKee Reverso e investigar os resultados obtidos com
essa mudança com respeito ao tempo para resolver os problemas e a qualidade da
solução.


% Outras atividades de interesse universitário
%     Descrever participações em congressos, cursos extra-curriculares, etc
\section{Outras atividades}
No período de bolsa o aluno participou dos seguintes treinamentos:
\begin{itemize}
  \item ``Introdução ao MPI'', de 23 a 26/04/2013 com a carga horário de 15
    horas,
  \item ``Introdução ao OpenMP'', de 14 a 16/05/2013 com carga horária de 09
    horas.
\end{itemize}

O aluno também teve o trabalho de iniciação pré-selecionado para o Prêmio de
Iniciação Científica do XLV Simpósio Brasileiro de Pesquisa Operacional como um
dos cinco melhores.


% TODO Para a versao final, comentar as linhas a seguir.
\appendix
% Apoio
%     Citar as agências que financiaram o projeto
% Filename: info.tex
% This code is part of 'CNPq 126874/2012-3'.
% 
% Description: Informa\c{c}\~{o}es.
% 
% Created: 20.08.12 07:27:58 PM
% Last Change: 20.08.12 07:27:58 PM
% 
% Author: Raniere Silva, <r.gaia.cs@gmail.com>
% 
% Copyright (c) 2012, Raniere Silva. All rights reserved.
% 
% This file is license under the terms of the GNU General Public License as published by the Free Software Foundation, either version 3 of the License, or (at your option) any later version. More details at <http://www.gnu.org/licenses/>
%
\section{Informa\c{c}\~{o}es}
Este trabalho foi financiado pelo Conselho Nacional de Desenvolvimento Cient\'{i}fico e Tecnol\'{o}gico pelo Processo 126874/2012-3.

Este trabalho \'{e} licenciado sob a Licen\c{c}a Creative Commons Atribui\c{c}\~{a}o 3.0 N\~{a}o Adaptada License. Para ver uma c\'{o}pia desta licen\c{c}a, visite \url{http://creativecommons.org/licenses/by/3.0/}.
\begin{center}
    \includegraphics{figures/cc-by.png}
\end{center}

% % Copyright (C) 2012 Raniere Silva
% 
% This file is part of 'CNPq 126874/2012-3'.
% 
% 'CNPq 126874/2012-3' is licensed under the Creative Commons
% Attribution 3.0 Unported License. To view a copy of this license,
% visit http://creativecommons.org/licenses/by/3.0/.
% 
% 'CNPq 126874/2012-3' is distributed in the hope that it will be
% useful, but WITHOUT ANY WARRANTY; without even the implied warranty of
% MERCHANTABILITY or FITNESS FOR A PARTICULAR PURPOSE.

\section{FAQ}
\begin{enumerate}
    \item \textbf{Justifique o problema de reduzir a largura de banda ser
        convertido em encontrar uma renumeração dos vértices do grafo tal que
        a diferença entre os índices seja mínima?}
        \cite{Fernanda:2005:ReordenacaoCCCG}

        O ato de permutar linhas e colunas de uma matrix corresponde a renumerar
        os nós de um grafo. \cite{Gibbs:1976:ReducingBandwidth}

    \item \textbf{Por que nós de alta excentricidade produzem bons
        resultados?} \cite{Fernanda:2005:ReordenacaoCCCG}

    \item \textbf{Por que um limitante inferior para a medida a largura de
        banda de $P A P^t$ para qualquer permutação $P$ é dado pelo menor
        inteiro maior ou igual a $D/2$, onde $D$ é o grau máximo de
        qualquer nó do grafo $G(A)$?} \cite{Cuthill:1969:ReducingBandwidth}

        Tomando o nó de grau máximo a maneira de obter a menor largura de
        banda para ele é posicionando-o na diagonal de modo que metade dos nós
        adjacentes esteja de um lado da banda e a outra metade do outro lado.

    \item \textbf{Por que iniciar com um nó de grau mínimo é uma boa escolha?}
        \cite{Cuthill:1969:ReducingBandwidth}

    \item \textbf{Por que numerar os nós em ordem crescente de grau?}
        \cite{Cuthill:1969:ReducingBandwidth}


\end{enumerate}

% % Copyright (C) 2012 Raniere Silva
% 
% This file is part of 'CNPq 126874/2012-3'.
% 
% 'CNPq 126874/2012-3' is licensed under the Creative Commons
% Attribution 3.0 Unported License. To view a copy of this license,
% visit http://creativecommons.org/licenses/by/3.0/.
% 
% 'CNPq 126874/2012-3' is distributed in the hope that it will be
% useful, but WITHOUT ANY WARRANTY; without even the implied warranty of
% MERCHANTABILITY or FITNESS FOR A PARTICULAR PURPOSE.

\section{Implementações}
A seguir enumeramos algumas das implementações disponíveis.

\subsection{C}
\subsubsection{Independente}
David Fritzsche, implementou o RCM para sua tese de mestrado (``Graph
Theoretical Methods for Preconditioners'', defendida em 2004 no
Departamento de Matemática da Bergische Universität Wuppertal. 

Disponível em \url{http://math.temple.edu/~daffi/software/rcm/}.

\subsection{C++}
\subsubsection{The Boost Graph Library (BGL)}
É uma biblioteca para trabalhar com grafos.

Encontra-se disponível em
\url{http://www.boost.org/doc/libs/1_51_0/libs/graph/doc/index.html}

\subsection{Python}
\subsubsection{PyOrder}
É uma biblioteca para reordenamento de matrizes esparsas.

Encontra-se disponível em \url{https://github.com/dpo/pyorder}.

\subsubsection{NetworkX}
É uma biblioteca para trabalhar com grafos.

Encontra-se disponível em \url{http://networkx.lanl.gov/index.html} e a
implementação do RCM em
\url{http://networkx.lanl.gov/examples/algorithms/rcm.html?highlight=cuthill}


% Bibliografia
%     Diversos formatos: definir qual o mais apropriado
%     IMPORTANTE
%     - Não liste se não citar.
%     - Não cite se não listar.
\bibliographystyle{alpha}
\bibliography{../references}
\end{document}
