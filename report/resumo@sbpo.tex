% Máximo de 150 palavras.
% 3 palavras chaves
\begin{minipage}{\textwidth}
\begin{abstract}
Neste trabalho revisitou-se a heurística de reordenamento Cuthill-McKee Reversa,
proposta em 1969, que busca um reordenamento para matrizes esparsas que reduza a
largura de banda destas. O autor deste trabalho implementou a heurística
revisitada, testou várias bibliotecas entre as quais a ``Netlib LP'' e concluiu, pelo menos
para os problemas testados, que a heurística Cuthill-McKee Reversa é inferior a
heurística de mínimo grau múltiplo por gerar mais elementos não nulos na
decomposição de Cholesky.
\end{abstract}

Palavras-chaves: Métodos de pontos interiores, decomposição de Cholesky,
reordenamento de matrizes esparsas.

Área principal: Programação Matemática.
\selectlanguage{english}
\begin{abstract}
In this work we revisited the reordering heuristic Reverse Cuthill-McKee,
proposed in 1969, that search for a sparse matrices reordering that reduces the
bandwidth. The author of this paper has implemented the revisited heuristic,
tested it with many libraries including the ``Netlib LP'' library and concluded, at least
for the tested problems, that the heuristic Cuthill-McKee Reverse is worse than
the multiple minimum degree heuristic because it generates more non-zero entries in
Cholesky decomposition.
\end{abstract}

Keywords: Interior point method, Cholesky decomposition, Sparse Matrix
Reordering.

Main area: Mathematical Programming.
\end{minipage}
