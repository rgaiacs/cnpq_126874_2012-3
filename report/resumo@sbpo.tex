% Máximo de 150 palavras.
% 3 palavras chaves
\begin{minipage}{\textwidth}
\begin{abstract}
Neste trabalho revisitou-se a heurística de reordenamento Cuthill-McKee
Reversa, proposta em 1969, que busca um reordenamento para matrizes esparsas
que reduza a largura de banda destas. O autor deste trabalho implementou a heurística
revisitada, testou com parte da biblioteca
``Netlib LP'' para os quais obteve bons resultados visto que, aproximadamente,
75\% das matrizes testadas tiveram o envelope reduzido com o
reordenamento obtido pela heurística e essa redução foi em média de 35\%.
\end{abstract}

Palavras-chaves: Métodos de pontos interiores, decomposição de Cholesky,
reordenamento de matrizes esparsas.

Área principal: Programação Matemática.
\selectlanguage{english}
\begin{abstract}
In this work we revisited the Reverse Cuthill-McKee reordering heuristic,
proposed in 1969, that sarching for a sparse matrices reordering which
reduces the bandwidth. The author of this work has implemented the revisited
heuristic, tested it with part of the ``Netlib LP'' library for which obtained
good results since 75\% of the tested matrix had the envelope reduced by
reordering obtained with the heuristic and this reduction was on average
35\%.
\end{abstract}

Keywords: Interior point method, Cholesky decomposition, Sparse Matrix
Reordering.

Main area: Mathematical Programming.
\end{minipage}
