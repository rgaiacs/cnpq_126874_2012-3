% Máximo de 150 palavras.
% 3 palavras chaves
\begin{abstract}
Neste trabalho revisitou-se a heurística de reordenamento Cuthill-McKee
Reversa, proposta em 1969, que busca um reordenamento para matrizes esparsas
que reduza a largura de banda destas. O autor deste trabalho implementou a heurística
revisitada, testou com parte da biblioteca
``Netlib LP'' para os quais obteve bons resultados visto que, aproximadamente,
75\% das matrizes testadas tiveram o envelope reduzido com o
reordenamento obtido pela heurística e essa redução foi em média de 35\%.

Palavras-chaves: Pontos Interiores, Preprocessamento.
\end{abstract}

\begin{abstract}
Neste trabalho revisitou-se a heurística de reordenamento Cuthill-McKee
Reversa, proposta em 1969, que busca um reordenamento para matrizes esparsas
que reduza a largura de banda destas. O autor deste trabalho implementou a heurística
revisitada, testou com parte da biblioteca
``Netlib LP'' para os quais obteve bons resultados visto que, aproximadamente,
75\% das matrizes testadas tiveram o envelope reduzido com o
reordenamento obtido pela heurística e essa redução foi em média de 35\%.

Palavras-chaves: Pontos Interiores, Preprocessamento.
\end{abstract}
