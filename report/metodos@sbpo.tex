\section{Implementação do Método e Testes Computacionais}
O Método Cuthill-McKee Reverso foi implementado, na linguagem C, pelo autor
deste trabalho como um patch para o PCx
(\url{http://pages.cs.wisc.edu/~swright/PCx/}), e a implementação desenvolvida
encontra-se disponível em \url{https://github.com/r-gaia-cs/PCx.git}.

O PCx é um solver de Programação Linear que utilizar a variante de Mehrotra do
Método Preditor Corretor com a estratégia de correção de Gondzio e o Método do
Mínimo Grau Múltiplo de Liu\cite{George:1981:ComputerSolutionPD} (MMD) para o
reordenamento da matriz $A D^{-1} A^T$.

Testou-se a implementação (tendo como último commit aquele cuja identificação
inicia com a27fab4) com os problemas das bibliotecas:
\begin{itemize}
  \item ``Netlib LP'' (\url{http://www.netlib.org/lp/}),
  \item ``Kennington'' (\url{http://www.netlib.org/lp/data/kennington/}),
  \item ``Mészáros'' (\url{http://www.sztaki.hu/~meszaros/public_ftp/lptestset/misc/}),
  \item ``PDS'' (\url{http://plato.asu.edu/ftp/lptestset/pds/}),
  \item ``Rail'' (\url{http://plato.asu.edu/ftp/lptestset/rail/}).
\end{itemize}

Juntando os problemas de todas a bibliotecas, foram utilizados mais de 200
problemas sendo que o RCM falhou em 2 problemas a mais que o MMD e o RCM
apresentou um memor número de elementos não zeros em 4 problemas em comparação
com o MMD. Na Tabela~\ref{tab:resul2} encontram-se os resultados obtidos para
os 25 maiores problemas das bibliotecas ``Netlib LP`` e ``Mészaros'' sendo que
``R'', ``NNZ'', ``IT'' e ``T'' significam, respectivamente, ``código de retorno do
PCx'', ``número de elementos não zeros'', ``número de iterações'' e ``tempo
computacional'' (em segundos). Em relação ao ``código de retorno do PCx'', -1
indica que o tempo limite foi atingido e outros valores diferentes de 0 que
alguma instabilidade numérica foi detectada.
