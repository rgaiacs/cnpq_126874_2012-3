\section{Implementação do Método e Testes Computacionais}
O Método Cuthill-McKee Reverso foi implementado, na linguagem C, pelo autor
deste trabalho como um patch para o PCx
(\url{http://pages.cs.wisc.edu/~swright/PCx/}), e a implementação desenvolvida
encontra-se disponível em \url{https://github.com/r-gaia-cs/PCx.git}.

O PCx é um solver de Programação Linear que utilizar a variante de Mehrotra do
Método Preditor Corretor com a estratégia de correção de Gondzio e o Método do
Mínimo Grau Múltiplo de Liu\cite{George:1981:ComputerSolutionPD} (MMD) para o
reordenamento da matriz $A D^{-1} A^T$.

Testou-se a implementação (tendo como último commit aquele cuja identificação
inicia com c8e1e1d) com 94 problemas da biblioteca ``Netlib LP''
(\url{http://www.netlib.org/lp/}), cujo acesso é livre.

Dos 94 problemas testados, ao utilizar reordenamento obtido com o MMD 87
problemas foram resolvidos enquanto que com o reordenamento do RCM apenas 85
problemas foram resolvidos. Dentre estes 85 problemas resolvidos por ambos, os
dados referentes aos testes computacionais dos 43 maiores problemas encontram-se
na Tabela~\ref{tab:resul2} sendo que ``LIN'', ``COL'', ``ENV'', ``IT'' e ``T''
significam, respectivamente, ``número de linhas originais'', ``número de colunas
originais'', ``tamanho do envelope'', ``número de iterações'' e ``tempo
computacional'' (em segundos).
\begin{table}
  \caption{Resultados experimentais.}
  \label{tab:resul2}
  \input{bench2@sbpo.csv}
\end{table}

O envelope decorrente do uso do RCM foi menor que o do MMD em apenas 4 problemas
e empatou em outros 5. Para os problemas em que o envelope decorrente do RCM foi
maior que o do MMD, esse aumento foi em média de 55\%. Em relação ao número de
iterações do PCx, o RCM foi melhor em 2 problemas e empatou nos outros 81. Já em
relação ao tempo, o RCM foi melhor em 6 problemas e empatou em outros 17.
