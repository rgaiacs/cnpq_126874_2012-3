\documentclass[10pt]{beamer}
%%%%%%%%%%%%%%%%%%%%%%%%%%%% Configuração: pacotes %%%%%%%%%%%%%%%%%%%%%%%%%%%%
%%%%%%%%%%%%%%%%%%%%%%%%%%%%%% Pacotes: básicos %%%%%%%%%%%%%%%%%%%%%%%%%%%%%%
\usepackage[utf8]{inputenc}
\usepackage{cmap}
\usepackage[T1]{fontenc}
\usepackage[portuguese]{babel}
\usepackage{indentfirst}
\usetheme{CambridgeUS}
%\usepackage{biblatex}
%\addbibresource{tese.bib}


%%%%%%%%%%%%%%%%%%%%%%%%%%%%%%% Pacotes: links %%%%%%%%%%%%%%%%%%%%%%%%%%%%%%%
\usepackage{url}
\usepackage{breakurl}
\usepackage{hyperref}


%%%%%%%%%%%%%%%%%%%%%%%%%%%%%%%% Pacotes: ams %%%%%%%%%%%%%%%%%%%%%%%%%%%%%%%%
\usepackage{amsmath}
\usepackage{amsfonts}
\usepackage{amssymb}
\usepackage{amsthm}
\usepackage{breqn}


%%%%%%%%%%%%%%%%%%%%%%%%%%%%%% Pacotes: tabelas %%%%%%%%%%%%%%%%%%%%%%%%%%%%%%
\usepackage{multirow}


%%%%%%%%%%%%%%%%%%%%%%%%%%%%%% Pacotes: figuras %%%%%%%%%%%%%%%%%%%%%%%%%%%%%%
\usepackage{tikz} %Pacote para desenho de figuras
\usetikzlibrary{matrix}

%%%%%%%%%%%%%%%%%%%%%%%%%%%%% Pacotes: algoritmos %%%%%%%%%%%%%%%%%%%%%%%%%%%%%
\usepackage{algorithmic} % Pacote para algoritmos
\usepackage{algorithm} % Pacote para algoritmos
\usepackage{listings} % Pacote para c\'{o}digos
\renewcommand{\algorithmicrequire}{\textbf{Entrada:}}
\renewcommand{\algorithmicensure}{\textbf{Saída:}}
\renewcommand{\algorithmicend}{\textbf{fim}}
\renewcommand{\algorithmicif}{\textbf{se}}
\renewcommand{\algorithmicthen}{\textbf{ent\~{a}o}}
\renewcommand{\algorithmicelse}{\textbf{caso contr\'{a}rio}}
\renewcommand{\algorithmicendif}{\algorithmicend}
\renewcommand{\algorithmicfor}{\textbf{para}}
\renewcommand{\algorithmicforall}{\textbf{para todo}}
\renewcommand{\algorithmicdo}{\textbf{fa\c{c}a}}
\renewcommand{\algorithmicendfor}{\algorithmicend}
\renewcommand{\algorithmicwhile}{\textbf{enquanto}}
\renewcommand{\algorithmicendwhile}{\algorithmicend}
\renewcommand{\algorithmicrepeat}{\textbf{repita}}
\renewcommand{\algorithmicuntil}{\textbf{at\'{e}}}
\renewcommand{\algorithmicreturn}{\textbf{retorne}}
\renewcommand{\algorithmiccomment}[1]{\hspace{2em}/* #1 */}

%%%%%%%%%%%%%%%%%%%%%%%%%%%%%%% Início do poster %%%%%%%%%%%%%%%%%%%%%%%%%%%%%%%
\begin{document}
\title[Cuthill-McKee Reversa]{Implementa\c{c}\~{a}o eficiente da heur\'{i}stica de reordenamento de Cuthill-McKee Reversa}
\author[Raniere Silva]{Raniere Gaia Costa da
Silva\footnote{ra092767@ime.unicamp.br}
\and Aurelio Ribeiro Leite de Oliveira\footnote{aurelio@ime.unicamp.br}}
\date{}
\begin{frame}
  \maketitle

  \footnotesize{Este trabalho foi financiado pelo Conselho Nacional de
  Desenvolvimento Cient\'{i}fico e Tecnol\'{o}gico pelo Processo 126874/2012-3 e
  encontra-se disponível em
  \url{https://github.com/r-gaia-cs/cnpq_126874_2012-3}.}

  \footnotesize{Salvo indicado o contrário, esta apresentação está licenciada sob a licença
  Creative Commons Atribuição 3.0 Não Adaptada. Para ver uma cópia desta
  licença, visite \url{http://creativecommons.org/licenses/by/3.0/}.}
  \begin{center}
    \includegraphics[scale=.5]{figures/cc-by.png}
  \end{center}
\end{frame}

\begin{frame}
  \tableofcontents
\end{frame}

\section{Motivação}
\begin{frame}{Problema de Programação Linear}
  \begin{align*}
      \text{minimizar } & c^T x
      &&& \text{maximizar } & b^T y \\
      \text{sujeito a } & A x = b
      &&\text{e}& \text{sujeito a } & A^T y + z = c \\
      & x \geq 0
      &&& & z \geq 0.
  \end{align*}
\end{frame}

\begin{frame}{Métodos de Pontos Interiores Primal-Dual (1)}
  Método iterativo descrito por
  \begin{align*}
      x^{i + 1} &= x^i + \alpha^i \Delta x^i, \\
      y^{i + 1} &= y^i + \alpha^i \Delta y^i, \\
      z^{i + 1} &= z^i + \alpha^i \Delta z^i,
  \end{align*}
  onde
  \begin{align*}
      \begin{bmatrix}
           0 & A^T & I \\
           A & 0 & 0 \\
           Z & 0 & X
       \end{bmatrix} \begin{bmatrix}
           \Delta x^i \\
           \Delta y^i \\
           \Delta z^i
       \end{bmatrix} &= \begin{bmatrix}
           r_d \\
           r_p \\
           r_a
       \end{bmatrix}.
  \end{align*}
\end{frame}

\begin{frame}{Métodos de Pontos Interiores Primal-Dual (2)}
  Atráves da eliminação de variáveis do sistema anterior,
  \begin{align*}
      A D^{-1} A^T \Delta y^i &= A D^{-1} r_1 - r_2, \\
      \Delta x^i &= D^{-1} \left( r_1 - A^T \Delta y^i \right), \\
      \Delta z^i &= X^{-1} \left( r_a - Z \Delta x^i \right).
  \end{align*}
\end{frame}

\begin{frame}{Decomposição de Cholesky e preenchimento (1)}
  % O vetor de permutação é
  % [3, 1, 2, 0, 4, 5, 6]
  \begin{align*}
      A &= \begin{bmatrix}
          1 & 1 & 1 & 1 & 1 & 0 & 0 \\
          1 & 2 & 1 & 1 & 1 & 1 & 1 \\
          1 & 1 & 2 & 1 & 1 & 1 & 1 \\
          1 & 1 & 1 & 2 & 1 & 0 & 0 \\
          1 & 1 & 1 & 1 & 2 & 0 & 0 \\
          0 & 1 & 1 & 0 & 0 & 3 & 2 \\
          0 & 1 & 1 & 0 & 0 & 2 & 3
      \end{bmatrix}, \\
      G &\sim \begin{bmatrix}
          1,4 & 0   & 0   &  0    & 0                   &  0                   &  0   \\
          0,70& 1,2 & 0   &  0    & 0                   &  0                   &  0   \\
          0,70& 0,40& 1,1 &  0    & 0                   &  0                   &  0   \\
          0,70& 0,40& 0,28&  0,50 & 0                   &  0                   &  0   \\
          0,70& 0,40& 0,28&  0,50 & 1,0                 &  0                   &  0   \\
          0   & 0,81& 0,57& -1,0  & 2,4 \times 10^{-16} &  1,0                 &  0   \\
          0   & 0,81& 0,57& -1,0  & 2,4 \times 10^{-16} & -1,7 \times 10^{-15} &  1,0
      \end{bmatrix}.
  \end{align*}
\end{frame}

\begin{frame}{Decomposição de Cholesky e preenchimento (2)}
  % O vetor de permutação é
  % [4, 3, 0, 2, 1, 6, 5]
  \begin{align*}
      A' &= \begin{bmatrix}
          2 & 1 & 1 & 1 & 1 & 0 & 0 \\
          1 & 2 & 1 & 1 & 1 & 0 & 0 \\
          1 & 1 & 1 & 1 & 1 & 0 & 0 \\
          1 & 1 & 1 & 2 & 1 & 1 & 1 \\
          1 & 1 & 1 & 1 & 2 & 1 & 1 \\
          0 & 0 & 0 & 1 & 1 & 3 & 2 \\
          0 & 0 & 0 & 1 & 1 & 2 & 3
      \end{bmatrix}, \\
      G' &\sim \begin{bmatrix}
          1,4 & 0 & 0 & 0 & 0 & 0 & 0 \\
          0,70 & 1,2 & 0 & 0 & 0 & 0 & 0 \\
          0,70 & 0,40 & 0,57 & 0 & 0 & 0 & 0 \\
          0,70 & 0,40 & 0,57 & 1 & 0 & 0 & 0 \\
          0,70 & 0,40 & 0,57 & 0 & 1 & 0 & 0 \\
          0 & 0 & 0 & 1 & 1 & 1 & 0 \\
          0 & 0 & 0 & 1 & 1 & 0 & 1
      \end{bmatrix}.
  \end{align*}
\end{frame}

\section{A Heurística}
\begin{frame}[fragile]{Método Cuthill-McKee}
  \begin{algorithmic}[1]
      \REQUIRE Grafo $G(A)$ e um vértice inicial $v$.
      \ENSURE $o$, novo ordenamento dos vértices de $G(A)$.
      \STATE Marca todos os vértices como não visitados.
      \STATE $o \longleftarrow \text{ vetor de zeros}$, $i \longleftarrow
      1$, $f \longleftarrow \text{ fila vazia}$
      \STATE Adicionar $v$ na fila $f$.
      \STATE Marca $v$ como visitado.
      \WHILE{$f$ não for vazia}
          \STATE Desenfileira $f$ em $v$.
          \STATE $o_i \longleftarrow v$
          \STATE $i \longleftarrow i + 1$
          \FORALL{vértice $w$ adjacente a $v$, em ordem crescente de grau,}
              \IF{$w$ ainda não foi visitado}
                  \STATE Adicionar $v$ na fila $f$.
                  \STATE Marca $w$ como visitado.
              \ENDIF
          \ENDFOR
      \ENDWHILE
  \end{algorithmic}
\end{frame}

\begin{frame}[fragile]{Método Cuthill-McKee Reverso}
  Utilizar o ordenamento ``inverso'' obtido pelo algoritmo anterior, i.e.,
  \begin{dmath*}
    o_i' = o_{n - i + 1}, \condition{$\forall i = 1, 2, \ldots, n$}
  \end{dmath*}
\end{frame}

\begin{frame}[fragile]{Vértice pseudo-periférico}
  \begin{algorithmic}[1]
      \REQUIRE Grafo $G(A)$.
      \ENSURE $x$.
      \STATE $r \longleftarrow \text{Nó arbitrário em }G(A)$
      \STATE Construir estrutura de nível a partir de $r$, $L(r)$.
      \label{alg:ppn:brle}
      \STATE Escolher um vértice $x$ pertencente ao último nível de
      $L(r)$.
      \STATE Construir estrutura de nível a partir de $x$, $L(x)$.
      \IF{$l(x) > l(r)$}
          \STATE $r \longleftarrow x$
          \STATE Retorna para a linha~\ref{alg:ppn:brle}
      \ENDIF
  \end{algorithmic}
\end{frame}

\section{Experimentos Computacionais}
\begin{frame}{Implementação}
  Ambos as heurísticas foram implementadas em C e integradas ao PCx,
  inicialmente escrito por Joe Czyzyk, Sanjay Mehrotra, Michael Wagner, Stephen
  Wright, \textit{et al}.

  A implementação encontra-se disponível em
  \url{https://github.com/r-gaia-cs/PCx}.
\end{frame}

\section{Resultados}
\begin{frame}[fragile]{Resultados experimentais}
  \begin{center}
\begin{tabular}{|l|r|r|r|r|r|r|}
\hline
\multicolumn{7}{|c|}{9 Problema da biblioteca Netlib com elevado número de
elementos não zero} \\ \hline
\multicolumn{1}{|c|}{Problema} & \multicolumn{3}{|c|}{MMD} &         \multicolumn{3}{|c|}{RCM} \\ \hline
\multicolumn{1}{|c|}{Nome} &
        \multicolumn{1}{|c|}{NNZ} & \multicolumn{1}{|c|}{IT} &
        \multicolumn{1}{|c|}{T} &
        \multicolumn{1}{|c|}{NNZ} & \multicolumn{1}{|c|}{IT} &
        \multicolumn{1}{|c|}{T} \\ \hline
dfl001       & 1,64E+06 & 45 & 2,44E+01 & 5,72E+06 & 60 & 3,47E+02 \\ \hline
fit2p        & 3,00E+03 & 20 & 5,40E-01 & 3,00E+03 & 20 & 5,80E-01 \\ \hline
80bau3b      & 4,14E+04 & 36 & 3,10E-01 & 2,30E+05 & 36 & 1,44E+00 \\ \hline
maros-r7     & 5,34E+05 & 14 & 1,61E+00 & 4,40E+05 & 14 & 4,10E+00 \\ \hline
pilot87      & 4,26E+05 & 29 & 3,35E+00 & 7,71E+05 & 25 & 7,71E+00 \\ \hline
d2q06c       & 1,37E+05 & 27 & 5,40E-01 & 4,55E+05 & 24 & 3,01E+00 \\ \hline
greenbea     & 4,91E+04 & 50 & 4,40E-01 & 1,51E+05 & 49 & 1,52E+00 \\ \hline
greenbeb     & 4,78E+04 & 40 & 3,50E-01 & 1,68E+05 & 39 & 1,18E+00 \\ \hline
bnl2         & 8,13E+04 & 33 & 3,10E-01 & 2,34E+05 & 33 & 1,34E+00 \\ \hline
\end{tabular}
  \end{center}

Dentre todos os problemas testados o RCM apresentou um menor número de elementos
não zeros em apenas 4 problemas quando comparado ao Método do Mínimo Grau
Múltiplo.

Resultados completos disponíveis em \url{https://github.com/r-gaia-cs/PCx}.
\end{frame}

\begin{frame}{Conclusões}
  O RCM foi descartado para investigação futuras uma vez que é inferior
  ao Mínimo Grau Múltiplo.
\end{frame}

\begin{frame}
  \begin{center}
    Perguntas?
  \end{center}

  \begin{block}{Links}
    \begin{itemize}
      \item Email: \url{ra092767@ime.unicamp.br}
      \item Código: \url{https://github.com/r-gaia-cs/PCx}.
      \item Relatório: \url{https://github.com/r-gaia-cs/cnpq_126874_2012-3}
    \end{itemize}
  \end{block}
\end{frame}
\end{document}
