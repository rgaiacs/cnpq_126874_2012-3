% Copyright (C) 2012 Raniere Silva
% 
% This file is part of 'CNPq 126874/2012-3'.
% 
% 'CNPq 126874/2012-3' is licensed under the Creative Commons
% Attribution 3.0 Unported License. To view a copy of this license,
% visit http://creativecommons.org/licenses/by/3.0/.
% 
% 'CNPq 126874/2012-3' is distributed in the hope that it will be
% useful, but WITHOUT ANY WARRANTY; without even the implied warranty of
% MERCHANTABILITY or FITNESS FOR A PARTICULAR PURPOSE.

\documentclass[12pt,a4paper]{article}
% Filename: packages.tex
% This code is part of 'CNPq 126874/2012-3'.
% 
% Description: Pacotes utilizados.
% 
% Created: 20.08.12 07:27:58 PM
% Last Change: 20.08.12 07:27:58 PM
% 
% Author: Raniere Silva, <r.gaia.cs@gmail.com>
% 
% Copyright (c) 2012, Raniere Silva. All rights reserved.
% 
% This file is license under the terms of the GNU General Public License as published by the Free Software Foundation, either version 3 of the License, or (at your option) any later version. More details at <http://www.gnu.org/licenses/>
%
%Pacotes utilizados
\usepackage[top=3cm,left=2cm,right=2cm,bottom=3cm]{geometry} %Borda das p\'{a}gina
\usepackage[utf8]{inputenc} %Ser\'{a} utilizado a codifica\c{c}ao UTF8
\usepackage[T1]{fontenc}
%\usepackage[latin1]{inputenc} % Ser\'{a} utilizado a codifica\c{c}ao latin1
\usepackage[brazil]{babel} %Ser\'{a} utilizado o idioma portugu\^{e}s
\usepackage{indentfirst} %Identa\c{c}\~{a}o de linha
\usepackage{setspace} %Espa\c{c}amento entre linas
\usepackage{amsmath, amsfonts, amssymb, amsthm} % Pacote matem\'{a}tico
\usepackage{graphicx} %Pacote para incluir figuras
\usepackage{color} %Pacote para cores
\usepackage{tikz} %Pacote para desenho de figuras
\usepackage{pgfplots} %Pacote para desenho de gr\'{a}ficos
\usepackage{subfigure} % Pacote para subfiguras
\usepackage{url} % Pacote para url
\usepackage{hyperref} % Pacote para hyperlink
\usepackage{algorithmic} % Pacote para algoritmos
\usepackage{algorithm} % Pacote para algoritmos
\usepackage{listings} % Pacote para c\'{o}digos

%Personalização
%\pgfplotsset{
%height=0.35\textheight,
%width=0.8\textwidth,
%legend style={
%at={(0.3,0.97)},
%anchor=north,
%font=\footnotesize
%},
%xlabel={},
%ylabel={},
%xtick={},
%grid=major,
%}

\hypersetup{
%colorlinks = true,
}

\floatname{algorithm}{Algoritmo}

\algsetup{linenosize=\small}
\renewcommand{\algorithmicrequire}{\textbf{Entrada:}}
\renewcommand{\algorithmicensure}{\textbf{Saída:}}
\renewcommand{\algorithmicend}{\textbf{fim}}
\renewcommand{\algorithmicif}{\textbf{se}}
\renewcommand{\algorithmicthen}{\textbf{ent\~{a}o}}
\renewcommand{\algorithmicelse}{\textbf{caso contr\'{a}rio}}
\renewcommand{\algorithmicendif}{\algorithmicend}
\renewcommand{\algorithmicfor}{\textbf{para}}
\renewcommand{\algorithmicforall}{\textbf{para todo}}
\renewcommand{\algorithmicdo}{\textbf{fa\c{c}a}}
\renewcommand{\algorithmicendfor}{\algorithmicend}
\renewcommand{\algorithmicwhile}{\textbf{enquanto}}
\renewcommand{\algorithmicendwhile}{\algorithmicend}
\renewcommand{\algorithmicrepeat}{\textbf{repita}}
\renewcommand{\algorithmicuntil}{\textbf{at\'{e}}}
\renewcommand{\algorithmicreturn}{\textbf{retorne}}
\renewcommand{\algorithmiccomment}[1]{\hspace{2em}/* #1 */}

\renewcommand{\lstlistingname}{C\'{o}digo}
\lstset{ %
% language=Octave,                % the language of the code
basicstyle=\ttfamily\small,       % the size of the fonts that are used for the code
numbers=left,                   % where to put the line-numbers
numberstyle=\footnotesize,      % the size of the fonts that are used for the line-numbers
stepnumber=5,                   % the step between two line-numbers. If it's 1, each line 
% will be numbered
numbersep=5pt,                  % how far the line-numbers are from the code
% backgroundcolor=\color{white},  % choose the background color. You must add \usepackage{color}
showspaces=false,               % show spaces adding particular underscores
showstringspaces=false,         % underline spaces within strings
showtabs=false,                 % show tabs within strings adding particular underscores
% frame=single,                   % adds a frame around the code
tabsize=4,                      % sets default tabsize to 2 spaces
captionpos=t,                   % sets the caption-position to bottom
breaklines=true,                % sets automatic line breaking
breakatwhitespace=false,        % sets if automatic breaks should only happen at whitespace
caption={\texttt{\lstname}},                 % show the filename of files included with \lstinputlisting;
% also try caption instead of title
% escapeinside={\%*}{*)},         % if you want to add a comment within your code
% morekeywords={#}            % if you want to add more keywords to the set
}

\newtheorem{defi}{Definição}
\newtheorem{prop}{Proposição}

\begin{document}
\title{Implementa\c{c}\~{a}o eficiente da heur\'{i}stica de reordenamento de Cuthill-McKee reversa}
\author{Raniere Silva \\ Orientando \and Aurelio Oliveira \\ Orientador}
\maketitle

% ver orientações em http://www.prp.unicamp.br/pibic/relatorio_final.php
% Relatório deverá conter no máximo 20 páginas, incluindo bibliografia e anexos.
\tableofcontents

% Identificação
%     Projeto
%     Bolsista / RA
%     Orientador
%     Local de execução
%     Vigência

% Introdução
%     Introdução ao assunto: deve ser bastante geral
%     Informações da literatura: tornam a introdução mais específica ao assunto
%     Colocação da questão estudada: especificar os objetivos do trabalho
%     Atividades desenvolvidas: dar uma idéia geral de como foi desenvolvido o
%     trabalho

% Materiais e Métodos
%     Materiais: citar os equipamentos, reagentes e outros ítens utilizados,
%     informando fabricante ou fornecedor
%     Métodos: descrever os procedimentos detalhados, que possam ser
%     reproduzidos com os materiais e equipamentos descritos

% Resultados
%     Descrição dos resultados: deve ser clara e objetiva, resumindo os achados
%     principais que serão detalhados em tabelas e figuras.
%     Ilustrações dos resultados: tabelas e figuras são muito importantes; seu
%     número deve ser o menor possível, e elas devem ser construídas com cuidado
%     para incluir todas as informações necessárias com clareza.
%     Tabelas: devem ser numeradas sequencialmente (Tabela 1, Tabela 2, etc).
%     Seu título deve ser informativo, colocado acima e justificado à esquerda.
%     Notas de rodapé (a, b, c...) podem ser colocados diretamente abaixo da
%     mesma.
%     Figuras (fotos, esquemas, gráficos): devem ser numeradas sequencialmente
%     (Figura 1, Figura 2, etc). Seu título deve ser informativo, colocado
%     abaixo e justificado à esquerda, descrevendo o que é mostrado.

% Discussão / Conclusões
%     Descrição dos dados à luz da literatura
%     Descrição de possíveis fontes de erro e seu efeito sobre os dados
%     Se seus experimentos falharam, quais as sugestões para corrigir o
%     problema?

% Matéria encaminhada para publicação
%     Quando houver, referir resumos ou artigos científicos publicados ou
%     encaminhados para publicação

% Bibliografia
%     Diversos formatos: definir qual o mais apropriado
%     IMPORTANTE
%     - Não liste se não citar.
%     - Não cite se não listar.

% Perspectivas de continuidade ou desdobramento do trabalho
%     O projeto foi concluído ou será continuado?

% Outras atividades de interesse universitário
%     Descrever participações em congressos, cursos extra-curriculares, etc

% Apoio
%     Citar as agências que financiaram o projeto

% Agradecimentos
%     Citar pessoas ou instituições que tenham colaborado para a execução do projeto

\newpage
\bibliographystyle{alpha}
\bibliography{references}
\end{document}
