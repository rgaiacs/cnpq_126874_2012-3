\section{Resultados Obtidos e Conclusão}

rcm melhor que mg: 6
rcm melhor que mg: 0
RCM
red: 56 - 0.7567567567567568
media: 0.35052302815089803
man: 7 - 0.0945945945945946
aum: 11 - 0.14864864864864866
MG
red: 68 - 0.918918918918919
media: 0.7283838871092406
man: 0 - 0.0
aum: 6 - 0.08108108108108109

Comparando o envelope da matriz $A D^{-1} A^T$ e da matriz $P A D^{-1} A^T P^T$,
em que a matriz de permutação $P$ é obtida pelo Método Cuthill-McKee Reverso,
observamos
\begin{itemize}
    \item redução do envelope para 56 matrizes (75,67\%),
    \item manutenção do envelope para 7 matrizes (9,45\%) e
    \item aumenta do envelope para 11 matrizes (14,86\%).
\end{itemize}
Nas matrizes onde ocorreu a redução do envelope essa foi, em média, de 35,05\%.

Comparando o envelope da matriz $A D^{-1} A^T$ e da matriz $P A D^{-1} A^T P^T$,
em que a matriz de permutação $P$ é obtida pelo Método do Mínimo Grau Múltiplo
observamos
\begin{itemize}
    \item redução do envelope para 68 matrizes (91,89\%) e
    \item aumenta do envelope para 6 matrizes (8,10\%).
\end{itemize}
Nas matrizes onde ocorreu a redução do envelope essa foi, em média, de 72,83\%.

%\begin{table}
%    \centering
%    \caption{Resultados experimentais do Método Cuthill-McKee Reverso para
%    a ``Netlib LP''.}
%    \label{tab:resul}
%    \csvautotabular{bench1@sbpo.csv}
%\end{table}
%\begin{table}
%    \centering
%    \csvautotabular{bench2@sbpo.csv}
%\end{table}
%\begin{table}
%    \centering
%    \csvautotabular{bench3@sbpo.csv}
%\end{table}

A primeira vista, pelos resultados, concluímos ser vantajoso utilizar algum
reordenamento e que o Método Cuthill-McKee Reverso é muito inferior ao Método do
Mínimo Grau Múltiplo. Infelizmente, a comparação entre o Método Cuthill-McKee
Reverso e o Método do Mínimo Grau Múltiplo é desequilibrada pois
\begin{itemize}
  \item antes de aplicar o Método do Mínimo Grau Múltiplo a matriz passou pelo
    pré-solver do PCx (enquanto que o Método Cuthill-McKee foi aplicado na
    matriz inicial) e
  \item a implementação do Método do Mínimo Grau Múltiplo utiliza super-nós
    (enquanto que o Método Cuthill-McKee não).
\end{itemize}
Ainda em relação a comparação entre o Método Cuthill-McKee Reverso e o Método do
Mínimo Grau Múltiplo gostariamos de destacar que para 6 problemas a redução do
envelope utilizando o Método Cuthill-McKee Reverso foi melhor que o Método do
Mínimo Grau.
