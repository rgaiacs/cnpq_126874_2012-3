\section{Resultados Obtidos e Conclusão}
Os dados referentes aos testes computacionais encontram-se nas
Tabelas~\ref{tab:resul1}~e~\ref{tab:resul2} sendo que ``Col. b.p'', ``Col,
a.p'', ``EI'', ``ERCM'' e ``EMG'' significam, respectivamente ``Número de
colunas antes do pré-solver'', ``Número de colunas depois do pré-solver'',
``Envelope inicial'', ``Envelope com Método Cuthill-McKee Reverso'', ``Envelope
com Mínimo Grau Múltiplo''.
\begin{table}[h]
    \centering
    \caption{Resultados experimentais sem alteração de $A$ no pré-solver.}
    \label{tab:resul1}
    \csvautotabular{bench1@sbpo.csv}
\end{table}
\begin{sidewaystable}
  \caption{Resultados experimentais com alteração de $A$ no pré-solver.}
  \label{tab:resul2}
  \csvautotabular{bench2@sbpo.csv}\hfill\vspace{-11pt}
  \csvautotabular{bench3@sbpo.csv}
\end{sidewaystable}

Comparando o envelope da matriz $A D^{-1} A^T$ e da matriz $P A D^{-1} A^T P^T$,
em que a matriz de permutação $P$ é obtida pelo RCM, observamos
\begin{itemize}
    \item redução do envelope para 56 matrizes (75,67\%),
    \item manutenção do envelope para 7 matrizes (9,45\%) e
    \item aumenta do envelope para 11 matrizes (14,86\%).
\end{itemize}
Nas matrizes onde ocorreu a redução do envelope essa foi, em média, de 35,05\%.

Comparando o envelope da matriz $A D^{-1} A^T$ e da matriz $P A D^{-1} A^T P^T$,
em que a matriz de permutação $P$ é obtida pelo MMD
observamos
\begin{itemize}
    \item redução do envelope para 68 matrizes (91,89\%) e
    \item aumenta do envelope para 6 matrizes (8,10\%).
\end{itemize}
Nas matrizes onde ocorreu a redução do envelope essa foi, em média, de 72,83\%.

A primeira vista, pelos resultados, concluímos ser vantajoso utilizar algum
reordenamento e que o RCM é muito inferior ao MMD. Infelizmente, a comparação
entre o RCM e o MMD é desequilibrada pois
\begin{itemize}
  \item antes de aplicar o MMD a matriz passou pelo pré-solver do PCx (enquanto
    que o RCM foi aplicado na matriz inicial) e
  \item a implementação do MMD utiliza supernós (enquanto que o RCM não).
\end{itemize}
Ainda em relação a comparação entre o RCM e o MMD gostariamos de destacar que
para 6 problemas a redução do envelope utilizando o RCM foi melhor que o MMD.
