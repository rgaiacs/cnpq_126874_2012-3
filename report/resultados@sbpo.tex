\section{Resultados Obtidos e Conclusão}
Os dados referentes aos testes computacionais encontram-se nas
Tabelas~\ref{tab:resul1}~e~\ref{tab:resul2} sendo que ``Col. b.p'', ``Col,
a.p'', ``EI'', ``ERCM'' e ``EMG'' significam, respectivamente ``Número de
colunas antes do preprocessamento'', ``Número de colunas depois do preprocessamento'',
``Envelope inicial'', ``Envelope com Método Cuthill-McKee Reverso'', ``Envelope
com Mínimo Grau Múltiplo''.
\begin{table}[h]
    \centering
    \caption{Resultados experimentais sem alteração de $A$ no preprocessamento.}
    \label{tab:resul1}
    \csvautotabular{bench1@sbpo.csv}
\end{table}
\begin{sidewaystable}
  \caption{Resultados experimentais com alteração de $A$ no preprocessamento.}
  \label{tab:resul2}
  \csvautotabular{bench2@sbpo.csv}\hfill\vspace{-11pt}
  \csvautotabular{bench3@sbpo.csv}
\end{sidewaystable}

Comparando o envelope da matriz $A D^{-1} A^T$ e da matriz $P A D^{-1} A^T P^T$,
em que a matriz de permutação $P$ é obtida pelo RCM, observamos
redução do envelope para 56 matrizes (75,67\%),
manutenção do envelope para 7 matrizes (9,45\%) e
aumenta do envelope para 11 matrizes (14,86\%).
Nas matrizes onde ocorreu a redução do envelope essa foi, em média, de 35,05\%.

Comparando o envelope da matriz $A D^{-1} A^T$ e da matriz $P A D^{-1} A^T P^T$,
em que a matriz de permutação $P$ é obtida pelo MMD
observamos
redução do envelope para 68 matrizes (91,89\%) e
aumenta do envelope para 6 matrizes (8,10\%).
Nas matrizes onde ocorreu a redução do envelope essa foi, em média, de 72,83\%.
