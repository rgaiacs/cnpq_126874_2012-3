% Copyright (C) 2012 Raniere Silva
% 
% This file is part of 'CNPq 126874/2012-3'.
% 
% 'CNPq 126874/2012-3' is licensed under the Creative Commons
% Attribution 3.0 Unported License. To view a copy of this license,
% visit http://creativecommons.org/licenses/by/3.0/.
% 
% 'CNPq 126874/2012-3' is distributed in the hope that it will be
% useful, but WITHOUT ANY WARRANTY; without even the implied warranty of
% MERCHANTABILITY or FITNESS FOR A PARTICULAR PURPOSE.

\section{Cuthill-McKee reverso}
No Método Preditor-Corretor pode-se utilizar a Decomposição de Cholesky para resolver o sistema linear
% TODO Identificar o sistema linear do Método Preditor-Corretor

Para problemas de grande porte, esse sistema linear é esparso e ao utilizar a Decomposição de Cholesky pode ocorrer o preenchimento da matriz.
% TODO Incluir exemplo.

É possivel permutar linhas e colunas do sistema linear de modo a minimizar o preenchimento decorrente da Decomposição de Choleslky.
% TODO Incluir exemplo.

Cuthill e McKee \cite{Cuthill:1969:ReducingBandwidth} propuseram um algoritmo de reordenação, cujo objetivo principal é reduzir a largura de banda de uma matriz simétrica.

% TODO Mostrar equivalência com grafos
\begin{algorithm}
    \caption{Pseudo-algoritmo RCM}
    \label{alg:rcm}
    \begin{algorithmic}
        \REQUIRE Grafo $G(A)$ e um vértice $v$.
        \ENSURE $n$, novo ordenamento dos vértices de $G(A)$.
        \STATE $p = \text{ vetor de zeros}$
        \STATE $n = \text{ vetor de zeros}$
        \STATE $i = 1$
        \STATE $f = \text{ fila vazia}$
        \STATE Enfileira $v$ em $f$
        \STATE $p_v = 1$
        \WHILE{$f$ não for vazia}
            \STATE Desenfeira $f$ em $v$
            \STATE $n_v = i$
            \STATE $i = i + 1$
            \FOR{vertice $w$ adjacente a $v$}
                \IF{$p_w == 0$}
                    \STATE Enfileira $w$ em $f$
                    \STATE $p_w = 1$
                \ENDIF
            \ENDFOR
        \ENDWHILE
    \end{algorithmic}
\end{algorithm}
