% Copyright (C) 2012 Raniere Silva
% 
% This file is part of 'CNPq 126874/2012-3'.
% 
% 'CNPq 126874/2012-3' is licensed under the Creative Commons
% Attribution 3.0 Unported License. To view a copy of this license,
% visit http://creativecommons.org/licenses/by/3.0/.
% 
% 'CNPq 126874/2012-3' is distributed in the hope that it will be
% useful, but WITHOUT ANY WARRANTY; without even the implied warranty of
% MERCHANTABILITY or FITNESS FOR A PARTICULAR PURPOSE.

\section{FAQ}
\begin{enumerate}
    \item \textbf{Justifique o problema de reduzir a largura de banda é
        convertido em encontrar uma renumeração dos vértices do grafo tal que
        a diferença entre os índices seja mínima?}
        \cite{Fernanda:2005:ReordenacaoCCCG}

    \item \textbf{Por que um nós de alta excentricidade produzem bons
        resultados?} \cite{Fernanda:2005:ReordenacaoCCCG}

    \item \textbf{Por que um limitante inferior para a medida a largura de
        banda de $P A P^t$ para qualquer permutação $P$ é dado pelo menor
        inteiro maior ou igual a $D/2$, onde $D$ é o grau máximo de
        qualquer nó do grafo $G(A)$?} \cite{Cuthill:1969:ReducingBandwidth}

        Tomando o nó de grau máximo a maneira de obter a menor largura de
        banda para ele é posicionando-o na diagonal de modo que metade dos nós
        adjacentes esteja de um lado da banda e a outra metade do outro lado.

    \item \textbf{Por que iniciar com um nó de grau mínimo é uma boa escolha?}
        \cite{Cuthill:1969:ReducingBandwidth}

    \item \textbf{Por que numerar os nós em ordem crescente de grau?}
        \cite{Cuthill:1969:ReducingBandwidth}


\end{enumerate}
