\section{Implementação do Método}
O Método Cuthill-McKee Reverso foi implementado pelo autor deste relatório como
uma nova funcionalidade para o Sage (\url{http://www.sagemath.org/}), um
sistema matemático \textit{free} e \textit{open-source} licenciado sob GPL que
combina vários pacotes \textit{open-source} em uma interface comum baseada em
Python. A implementação desenvolvida foi submetida nos seguintes
\textit{patchs}:
\begin{enumerate}
    \item Ticket \#13565
        (\url{http://trac.sagemath.org/sage_trac/ticket/13565})
    \item Ticket \#13564
        (\url{http://trac.sagemath.org/sage_trac/ticket/13564})
    \item Ticket \#13581
        (\url{http://trac.sagemath.org/sage_trac/ticket/13581})
    \item Ticket \#13584
        (\url{http://trac.sagemath.org/sage_trac/ticket/13584})
    \item Ticket \#13601
        (\url{http://trac.sagemath.org/sage_trac/ticket/13601})
\end{enumerate}

Para testar a implementação, utilizou-se parte da biblioteca de testes ``Netlib LP''
(\url{http://www.netlib.org/lp/}) e na construção da matriz $A D^{-1} A^T$ que
compõe o sistema linear a ser resolvido no Método de Pontos Interiores
Primal-Dual utilizou-se $D = I$, \textit{i.e.}, $D$ como a matriz identidade.
