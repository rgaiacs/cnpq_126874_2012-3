\documentclass[]{beamer}
%%%%%%%%%%%%%%%%%%%%%%%%%%%% Configuração: pacotes %%%%%%%%%%%%%%%%%%%%%%%%%%%%
%%%%%%%%%%%%%%%%%%%%%%%%%%%%%% Pacotes: básicos %%%%%%%%%%%%%%%%%%%%%%%%%%%%%%
\usepackage[utf8]{inputenc}
\usepackage{cmap}
\usepackage[T1]{fontenc}
\usepackage[portuguese]{babel}
\usepackage{indentfirst}
\usepackage{beamerposter}
\usepackage{geometry}
\usepackage{biblatex}
\addbibresource{../references.bib}


%%%%%%%%%%%%%%%%%%%%%%%%%%%%%%% Pacotes: links %%%%%%%%%%%%%%%%%%%%%%%%%%%%%%%
\usepackage{url}
\usepackage{breakurl}
\usepackage{hyperref}


%%%%%%%%%%%%%%%%%%%%%%%%%%%%%%%% Pacotes: ams %%%%%%%%%%%%%%%%%%%%%%%%%%%%%%%%
\usepackage{amsmath}
\usepackage{amsfonts}
\usepackage{amssymb}
\usepackage{amsthm}
\usepackage{breqn}


%%%%%%%%%%%%%%%%%%%%%%%%%%%%%% Pacotes: tabelas %%%%%%%%%%%%%%%%%%%%%%%%%%%%%%
\usepackage{multirow}


%%%%%%%%%%%%%%%%%%%%%%%%%%%%%% Pacotes: figuras %%%%%%%%%%%%%%%%%%%%%%%%%%%%%%
\usepackage{tikz} %Pacote para desenho de figuras
\usetikzlibrary{matrix}

%%%%%%%%%%%%%%%%%%%%%%%%%%%%% Pacotes: algoritmos %%%%%%%%%%%%%%%%%%%%%%%%%%%%%
\usepackage{algorithmic} % Pacote para algoritmos
\usepackage{algorithm} % Pacote para algoritmos
\usepackage{listings} % Pacote para c\'{o}digos
\renewcommand{\algorithmicrequire}{\textbf{Entrada:}}
\renewcommand{\algorithmicensure}{\textbf{Saída:}}
\renewcommand{\algorithmicend}{\textbf{fim}}
\renewcommand{\algorithmicif}{\textbf{se}}
\renewcommand{\algorithmicthen}{\textbf{ent\~{a}o}}
\renewcommand{\algorithmicelse}{\textbf{caso contr\'{a}rio}}
\renewcommand{\algorithmicendif}{\algorithmicend}
\renewcommand{\algorithmicfor}{\textbf{para}}
\renewcommand{\algorithmicforall}{\textbf{para todo}}
\renewcommand{\algorithmicdo}{\textbf{fa\c{c}a}}
\renewcommand{\algorithmicendfor}{\algorithmicend}
\renewcommand{\algorithmicwhile}{\textbf{enquanto}}
\renewcommand{\algorithmicendwhile}{\algorithmicend}
\renewcommand{\algorithmicrepeat}{\textbf{repita}}
\renewcommand{\algorithmicuntil}{\textbf{at\'{e}}}
\renewcommand{\algorithmicreturn}{\textbf{retorne}}
\renewcommand{\algorithmiccomment}[1]{\hspace{2em}/* #1 */}

\DeclareMathOperator*{\diag}{diag}
%%%%%%%%%%%%%%%%%%%%%%%%%%%%%%% Início do poster %%%%%%%%%%%%%%%%%%%%%%%%%%%%%%%
\geometry{paperwidth=90cm,paperheight=100cm}
\begin{document}
\begin{frame}[t,fragile]
%%%%%%%%%%%%%%%%%%%%%%%%%%%%%%% Título do poster %%%%%%%%%%%%%%%%%%%%%%%%%%%%%%%
% Com base em http://www.prp.unicamp.br/pibic/congresso_formatacaopaineis.php
  \begin{center}
    \begin{huge}
      Implementa\c{c}\~{a}o eficiente da heur\'{i}stica de reordenamento de Cuthill-McKee reversa

      \vspace{20pt}
      Instituto de Matemática, Estatística e Computação Científica
    \end{huge}

    \vspace{20pt}
    \begin{Large}
      \begin{tabular}[]{c}
        Raniere Gaia Costa da Silva \\
        \url{ra092767@ime.unicamp.br}
      \end{tabular} \hspace{5cm}
      \begin{tabular}[]{c}
        Aurelio Ribeiro Leite de Oliveira \\
        \url{aurelio@ime.unicamp.br}
      \end{tabular}
    \end{Large}
  \end{center}
  \vspace{20pt}

%%%%%%%%%%%%%%%%%%%%%%%%%%%%%%% Corpo do poster %%%%%%%%%%%%%%%%%%%%%%%%%%%%%%%
% O ambiente columns é definido na classe beamer.
  \begin{columns}[t]
    \begin{column}{0.4\textwidth}
      \textbf{Resumo}

      Neste trabalho revisitou-se a heurística de reordenamento Cuthill-McKee Reversa,
      proposta em 1969, que busca um reordenamento para matrizes esparsas que reduza a
      largura de banda destas. O autor deste trabalho implementou a heurística
      revisitada, testou com várias bibliotecas (dentre elas a ``Netlib LP'' e
      ``Meszaros'') e concluiu, pelo menos para os problemas testados, que a
      heurística Cuthill-McKee Reversa é inferior a heurística de mínimo grau
      múltiplo por gerar mais elementos não nulos na decomposição de Cholesky.

      \textbf{Métodos de Pontos Interiores Primal-Dual}

      Consideremos o Problema de Programação Linear na Forma Padrão (PPLFP)
      \begin{align*}
          \text{minimizar } & c^T x \\
          \text{sujeito a } & A x = b, \\
          & x \geq 0,
      \end{align*}
      onde $A \in \mathbb{R}^{m \times n}$ é uma matriz de posto completo $m$ e $c$,
      $b$ e $x$ são vetores colunas de dimensão apropriada. Associado a este problema
      temos o problema dual
      \begin{align*}
          \text{maximizar } & b^T y \\
          \text{sujeito a } & A^T y + z = c, \\
          & z \geq 0,
      \end{align*}
      onde $y$ e $z$ possuem dimensão adequada.

      Os Métodos de Pontos Interiores Primais-Duais para resolver PPLFP consistem em, a
      partir de uma tripla inicial $(x^0, y^0, z^0)$, construir uma
      sequência de triplas, $(x^i, y^i, z^i)$, dada por
      \begin{align*}
          x^{i + 1} &= x^i + \alpha^i \Delta x^i, \\
          y^{i + 1} &= y^i + \alpha^i \Delta y^i, \\
          z^{i + 1} &= z^i + \alpha^i \Delta z^i.
      \end{align*}

      A direção afim nos Métodos de Pontos Interiores Primais-Duais, $(\Delta x^i,
      \Delta y^i, \Delta z^i)$,  é dada por:
      \begin{align*}
          \begin{bmatrix}
               0 & A^T & I \\
               A & 0 & 0 \\
               Z & 0 & X
           \end{bmatrix} \begin{bmatrix}
               \Delta x^i \\
               \Delta y^i \\
               \Delta z^i
           \end{bmatrix} &= \begin{bmatrix}
               r_d \\
               r_p \\
               r_a
           \end{bmatrix},
      \end{align*}
      onde $X = \diag(x^i)$, $Z = \diag(z^i)$, $r_p = b - A x^i$, $r_d = c - A^T y^i
      - z^i$, $r_a = - X Z e$ e $e$ representa o vetor de uns.

      Eliminando variáveis, chegamos a
      \begin{align*}
          A D^{-1} A^T \Delta y^i &= A D^{-1} r_1 - r_2, \\
          \Delta x^i &= D^{-1} \left( r_1 - A^T \Delta y^i \right), \\
          \Delta z^i &= X^{-1} \left( r_a - Z \Delta x^i \right),
      \end{align*}
      onde $D = X^{-1} Z$, $r_1 = r_d - X^{-1} r_a$ e $r_2 = r_p$.

      \textbf{Fatoração de Cholesky}

      O sistema linear $A D^{-1} A^T \Delta y^i = A D^{-1} r_1 - r_2$ pode ser
      resolvido por meio da fatoração de Cholesky que possui algumas
      propriedades abaixo descritas.

      Seguindo Cuthill e McKee \nocite{Cuthill:1969:ReducingBandwidth}, definimos
      \begin{align*}
          \beta_i(A) &= i - f_i(A), 1 \leq i \leq n, \\
          \beta(A) &= \max\left\{ \beta_i(A) \mid 1 \leq i \leq n \right\}
      \end{align*}
      em que $\beta_i(A)$ é a largura de banda da $i$-ésima linha de $A$ e
      $\beta(A)$ é a largura de banda da matriz $A$. A banda da matriz $A$ é
      definida como
      \begin{align*}
          \text{Band}(A) &= \left\{ \left\{ i, j \right\} \mid 0 < i - j \leq
          \beta(A) \right\}
      \end{align*}
      e ilustrada abaixo.
      \begin{figure}[!htb]
          \centering
          \begin{tikzpicture}
              \matrix (A) [matrix of math nodes,%
              left delimiter  = \lbrack,%
              right delimiter = \rbrack] at (0,0)
              {%
                  1 & 1 & 1 & 1 & 1 & 0 & 0 \\
                  1 & 2 & 1 & 1 & 1 & 1 & 1 \\
                  1 & 1 & 2 & 1 & 1 & 1 & 1 \\
                  1 & 1 & 1 & 2 & 1 & 0 & 0 \\
                  1 & 1 & 1 & 1 & 2 & 0 & 0 \\
                  0 & 1 & 1 & 0 & 0 & 3 & 2 \\
                  0 & 1 & 1 & 0 & 0 & 2 & 3 \\
              };
              \node[above, shift={(0,1)}] at (A-1-1) {$1$};
              \node[above, shift={(0,1)}] at (A-1-2) {$2$};
              \node[above, shift={(0,1)}] at (A-1-3) {$3$};
              \node[above, shift={(0,1)}] at (A-1-4) {$4$};
              \node[above, shift={(0,1)}] at (A-1-5) {$5$};
              \node[above, shift={(0,1)}] at (A-1-6) {$6$};
              \node[above, shift={(0,1)}] at (A-1-7) {$7$};
              \node[right, shift={(-2.5,0)}] at (A-1-1) {$1$};
              \node[right, shift={(-2.5,0)}] at (A-2-1) {$2$};
              \node[right, shift={(-2.5,0)}] at (A-3-1) {$3$};
              \node[right, shift={(-2.5,0)}] at (A-4-1) {$4$};
              \node[right, shift={(-2.5,0)}] at (A-5-1) {$5$};
              \node[right, shift={(-2.5,0)}] at (A-6-1) {$6$};
              \node[right, shift={(-2.5,0)}] at (A-7-1) {$7$};
              % Largura de banda
              % \draw (A-1-6.north east) -- (A-2-7.north east);
              \draw (A-6-1.south west) -- (A-7-2.south west);
              % Banda
              \fill[gray, fill opacity=0.2, draw=black, dotted] (A-2-1.north west) -- (A-6-1.south west)
              -- (A-6-2.south west) -- (A-7-2.south west) -- (A-7-7.south west)
              \foreach \x in {6,5,4,3,2}{
              -- (A-\x-\x.south east) -- (A-\x-\x.south west) -- (A-\x-\x.north west)
              } -- (A-2-1.north west);

              \matrix (B) [matrix of math nodes,%
              left delimiter  = \lbrack,%
              right delimiter = \rbrack] at (12,0)
              {%
                 10 &  0 &  2 &  3 &  0 &  0 &  0 \\
                  0 & 10 &  2 &  3 &  0 &  0 &  0 \\
                  2 &  2 & 10 &  0 &  4 &  0 &  0 \\
                  3 &  3 &  0 & 10 &  4 &  0 &  0 \\
                  0 &  0 &  4 &  4 & 10 &  2 &  2 \\
                  0 &  0 &  0 &  0 &  2 & 10 &  0 \\
                  0 &  0 &  0 &  0 &  2 &  0 & 10 \\
              };
              \node[above, shift={(0,1)}] at (B-1-1) {$1$};
              \node[above, shift={(0,1)}] at (B-1-2) {$2$};
              \node[above, shift={(0,1)}] at (B-1-3) {$3$};
              \node[above, shift={(0,1)}] at (B-1-4) {$4$};
              \node[above, shift={(0,1)}] at (B-1-5) {$5$};
              \node[above, shift={(0,1)}] at (B-1-6) {$6$};
              \node[above, shift={(0,1)}] at (B-1-7) {$7$};
              \node[right, shift={(-2.5,0)}] at (B-1-1) {$1$};
              \node[right, shift={(-2.5,0)}] at (B-2-1) {$2$};
              \node[right, shift={(-2.5,0)}] at (B-3-1) {$3$};
              \node[right, shift={(-2.5,0)}] at (B-4-1) {$4$};
              \node[right, shift={(-2.5,0)}] at (B-5-1) {$5$};
              \node[right, shift={(-2.5,0)}] at (B-6-1) {$6$};
              \node[right, shift={(-2.5,0)}] at (B-7-1) {$7$};
              % Largura de banda
              % \draw (B-1-4.north east) -- (B-4-7.north east);
              \draw (B-4-1.south west) -- (B-7-4.south west);
              % Banda
              \fill[gray, fill opacity=0.2, draw=black, dotted] (B-1-1.south west) -- (B-4-1.south west)
              -- (B-4-2.south west) -- (B-5-2.south west) -- (B-5-3.south west)
              -- (B-6-3.south west) -- (B-6-4.south west) -- (B-7-4.south west)
              -- (B-7-7.south west)
              \foreach \x in {6,5,4,3,2}{
              -- (B-\x-\x.south east) -- (B-\x-\x.south west) -- (B-\x-\x.north west)
              } -- (B-1-1.south west);
          \end{tikzpicture}
          \caption{Ilustração da banda, sombreado e contornado por pontos, e da
          largura de banda, linha contínua, para duas matrizes simétricas.}
          \label{fig:exem_bandwidth}
      \end{figure}

      O envelope da matriz $A$ é definido como
      \begin{align*}
          \text{Env}(A) &= \left\{ \left\{ i, j \right\} \mid 0 < i - j \leq
          \beta_i(A) \right\}.
      \end{align*}
      e ilustrado abaixo.
      \begin{figure}[!htb]
          \centering
          \begin{tikzpicture}
              \matrix (A) [matrix of math nodes,%
              left delimiter  = \lbrack,%
              right delimiter = \rbrack] at (0,0)
              {%
                  1 & 1 & 1 & 1 & 1 & 0 & 0 \\
                  1 & 2 & 1 & 1 & 1 & 1 & 1 \\
                  1 & 1 & 2 & 1 & 1 & 1 & 1 \\
                  1 & 1 & 1 & 2 & 1 & 0 & 0 \\
                  1 & 1 & 1 & 1 & 2 & 0 & 0 \\
                  0 & 1 & 1 & 0 & 0 & 3 & 2 \\
                  0 & 1 & 1 & 0 & 0 & 2 & 3 \\
              };
              \node[above, shift={(0,1)}] at (A-1-1) {$1$};
              \node[above, shift={(0,1)}] at (A-1-2) {$2$};
              \node[above, shift={(0,1)}] at (A-1-3) {$3$};
              \node[above, shift={(0,1)}] at (A-1-4) {$4$};
              \node[above, shift={(0,1)}] at (A-1-5) {$5$};
              \node[above, shift={(0,1)}] at (A-1-6) {$6$};
              \node[above, shift={(0,1)}] at (A-1-7) {$7$};
              \node[right, shift={(-2.5,0)}] at (A-1-1) {$1$};
              \node[right, shift={(-2.5,0)}] at (A-2-1) {$2$};
              \node[right, shift={(-2.5,0)}] at (A-3-1) {$3$};
              \node[right, shift={(-2.5,0)}] at (A-4-1) {$4$};
              \node[right, shift={(-2.5,0)}] at (A-5-1) {$5$};
              \node[right, shift={(-2.5,0)}] at (A-6-1) {$6$};
              \node[right, shift={(-2.5,0)}] at (A-7-1) {$7$};
              % Perfil
              \fill[gray, fill opacity=0.2, draw=black, dotted] (A-2-1.north west) -- (A-6-1.north
              west) -- (A-6-2.north west) --
              (A-7-2.south west) -- (A-7-7.south west)
              \foreach \x in {6,5,4,3,2}{
              -- (A-\x-\x.south east) -- (A-\x-\x.south west) -- (A-\x-\x.north west)
              } -- (A-2-1.north west);

              \matrix (B) [matrix of math nodes,%
              left delimiter  = \lbrack,%
              right delimiter = \rbrack] at (12,0)
              {%
                 10 &  0 &  2 &  3 &  0 &  0 &  0 \\
                  0 & 10 &  2 &  3 &  0 &  0 &  0 \\
                  2 &  2 & 10 &  0 &  4 &  0 &  0 \\
                  3 &  3 &  0 & 10 &  4 &  0 &  0 \\
                  0 &  0 &  4 &  4 & 10 &  2 &  2 \\
                  0 &  0 &  0 &  0 &  2 & 10 &  0 \\
                  0 &  0 &  0 &  0 &  2 &  0 & 10 \\
              };
              \node[above, shift={(0,1)}] at (B-1-1) {$1$};
              \node[above, shift={(0,1)}] at (B-1-2) {$2$};
              \node[above, shift={(0,1)}] at (B-1-3) {$3$};
              \node[above, shift={(0,1)}] at (B-1-4) {$4$};
              \node[above, shift={(0,1)}] at (B-1-5) {$5$};
              \node[above, shift={(0,1)}] at (B-1-6) {$6$};
              \node[above, shift={(0,1)}] at (B-1-7) {$7$};
              \node[right, shift={(-2.5,0)}] at (B-1-1) {$1$};
              \node[right, shift={(-2.5,0)}] at (B-2-1) {$2$};
              \node[right, shift={(-2.5,0)}] at (B-3-1) {$3$};
              \node[right, shift={(-2.5,0)}] at (B-4-1) {$4$};
              \node[right, shift={(-2.5,0)}] at (B-5-1) {$5$};
              \node[right, shift={(-2.5,0)}] at (B-6-1) {$6$};
              \node[right, shift={(-2.5,0)}] at (B-7-1) {$7$};
              % Perfil
              \fill[gray, fill opacity=0.2, draw=black, dotted] (B-2-1.south west) -- (B-5-1.north
              west) -- (B-5-3.north west) --
              (B-5-3.south west) -- (B-6-5.north west) -- (B-7-5.south west)
              -- (B-7-7.south west)
              \foreach \x in {6,5,4,3}{
              -- (B-\x-\x.south east) -- (B-\x-\x.south west) -- (B-\x-\x.north west)
              } -- (B-2-1.south west);
          \end{tikzpicture}
          \caption{Ilustração do envelope, sombreado e contornado por
          pontos, para duas matrizes simétricas.}
          \label{fig:exem_profile}
      \end{figure}
    \end{column}
    \hfill
    \begin{column}{0.4\textwidth}
      É possível provar que a fatoração de Cholesky preserva a largura de banda
      e o envelope da matriz. Abaixo encontra-se um exemplo.
  \begin{align*}
      A &= \begin{bmatrix}
          1 & 1 & 1 & 1 & 1 & 0 & 0 \\
          1 & 2 & 1 & 1 & 1 & 1 & 1 \\
          1 & 1 & 2 & 1 & 1 & 1 & 1 \\
          1 & 1 & 1 & 2 & 1 & 0 & 0 \\
          1 & 1 & 1 & 1 & 2 & 0 & 0 \\
          0 & 1 & 1 & 0 & 0 & 3 & 2 \\
          0 & 1 & 1 & 0 & 0 & 2 & 3
      \end{bmatrix}, &
      G &\sim \begin{bmatrix}
          1,4 & 0   & 0   &  0    & 0                   &  0                   &  0   \\
          0,70& 1,2 & 0   &  0    & 0                   &  0                   &  0   \\
          0,70& 0,40& 1,1 &  0    & 0                   &  0                   &  0   \\
          0,70& 0,40& 0,28&  0,50 & 0                   &  0                   &  0   \\
          0,70& 0,40& 0,28&  0,50 & 1,0                 &  0                   &  0   \\
          0   & 0,81& 0,57& -1,0  & 2,4 \times 10^{-16} &  1,0                 &  0   \\
          0   & 0,81& 0,57& -1,0  & 2,4 \times 10^{-16} & -1,7 \times 10^{-15} &  1,0
      \end{bmatrix}.
  \end{align*}

  Permutando, simetricamente, as linhas e colunas da matriz é possível reduzir a
  largura de banda e o envelope da mesma. Abaixo encontra-se um exemplo de
  permutação da matriz anterior para o qual observa a redução da largura de
  banda e do envelope.
  \begin{align*}
      A' &= \begin{bmatrix}
          2 & 1 & 1 & 1 & 1 & 0 & 0 \\
          1 & 2 & 1 & 1 & 1 & 0 & 0 \\
          1 & 1 & 1 & 1 & 1 & 0 & 0 \\
          1 & 1 & 1 & 2 & 1 & 1 & 1 \\
          1 & 1 & 1 & 1 & 2 & 1 & 1 \\
          0 & 0 & 0 & 1 & 1 & 3 & 2 \\
          0 & 0 & 0 & 1 & 1 & 2 & 3
      \end{bmatrix}, &
      G' &\sim \begin{bmatrix}
          1,4 & 0 & 0 & 0 & 0 & 0 & 0 \\
          0,70 & 1,2 & 0 & 0 & 0 & 0 & 0 \\
          0,70 & 0,40 & 0,57 & 0 & 0 & 0 & 0 \\
          0,70 & 0,40 & 0,57 & 1 & 0 & 0 & 0 \\
          0,70 & 0,40 & 0,57 & 0 & 1 & 0 & 0 \\
          0 & 0 & 0 & 1 & 1 & 1 & 0 \\
          0 & 0 & 0 & 1 & 1 & 0 & 1
      \end{bmatrix}.
  \end{align*}
      \textbf{Reordenamento de Cuthill-McKee reversa}

      O reordenamento proposto por Cuthill e McKee corresponde ao algoritmo
      abaixo:

  \begin{algorithmic}[1]
      \REQUIRE Grafo $G(A)$ e um vértice inicial $v$.
      \ENSURE $o$, novo ordenamento dos vértices de $G(A)$.
      \STATE Marca todos os vértices como não visitados.
      \STATE $o \longleftarrow \text{ vetor de zeros}$, $i \longleftarrow
      1$, $f \longleftarrow \text{ fila vazia}$
      \STATE Adicionar $v$ na fila $f$.
      \STATE Marca $v$ como visitado.
      \WHILE{$f$ não for vazia}
          \STATE Desenfileira $f$ em $v$.
          \STATE $o_i \longleftarrow v$
          \STATE $i \longleftarrow i + 1$
          \FORALL{vértice $w$ adjacente a $v$, em ordem crescente de grau,}
              \IF{$w$ ainda não foi visitado}
                  \STATE Adicionar $v$ na fila $f$.
                  \STATE Marca $w$ como visitado.
              \ENDIF
          \ENDFOR
      \ENDWHILE
  \end{algorithmic}

  Utilizar o ordenamento ``inverso'' obtido pelo algoritmo anterior corresponde
  a
  \begin{dmath*}
    o_i' = o_{n - i + 1}, \condition{$\forall i = 1, 2, \ldots, n$.}
  \end{dmath*}

  O reordenamento depende do vértice inicial $v$. Experimentos computacionais
  mostram que vértices pseudo-periférico produzem bons resultados e estes podem
  ser obtidos pelo algoritmo abaixo proposto por George e Liu
  \cite{George:1979:NodeFinder}.

  \begin{algorithmic}[1]
      \REQUIRE Grafo $G(A)$.
      \ENSURE $x$.
      \STATE $r \longleftarrow \text{Nó arbitrário em }G(A)$
      \STATE Construir estrutura de nível a partir de $r$, $L(r)$.
      \label{alg:ppn:brle}
      \STATE Escolher um vértice $x$ pertencente ao último nível de
      $L(r)$.
      \STATE Construir estrutura de nível a partir de $x$, $L(x)$.
      \IF{$l(x) > l(r)$}
          \STATE $r \longleftarrow x$
          \STATE Retorna para a linha~\ref{alg:ppn:brle}
      \ENDIF
  \end{algorithmic}

  \textbf{Implementação}

  Ambos as heurísticas foram implementadas em C e integradas ao PCx \cite{PCx},
  sendo que a implementação encontra-se disponível em
  \url{https://github.com/r-gaia-cs/PCx}.

  \textbf{Resultados experimentais}
  \begin{center}
\begin{tabular}{|l|r|r|r|r|r|r|}
\hline
\multicolumn{7}{|c|}{5 Problema da biblioteca Netlib} \\ \hline
\multicolumn{1}{|c|}{Problema} & \multicolumn{3}{|c|}{MMD} &         \multicolumn{3}{|c|}{RCM} \\ \hline
\multicolumn{1}{|c|}{Nome} &
        \multicolumn{1}{|c|}{NNZ} & \multicolumn{1}{|c|}{IT} &
        \multicolumn{1}{|c|}{T} &
        \multicolumn{1}{|c|}{NNZ} & \multicolumn{1}{|c|}{IT} &
        \multicolumn{1}{|c|}{T} \\ \hline
dfl001       & 1,64E+06 & 45 & 2,44E+01 & 5,72E+06 & 60 & 3,47E+02 \\ \hline
fit2p        & 3,00E+03 & 20 & 5,40E-01 & 3,00E+03 & 20 & 5,80E-01 \\ \hline
80bau3b      & 4,14E+04 & 36 & 3,10E-01 & 2,30E+05 & 36 & 1,44E+00 \\ \hline
maros-r7     & 5,34E+05 & 14 & 1,61E+00 & 4,40E+05 & 14 & 4,10E+00 \\ \hline
pilot87      & 4,26E+05 & 29 & 3,35E+00 & 7,71E+05 & 25 & 7,71E+00 \\ \hline
\end{tabular}
  \end{center}

  \textbf{Conclusões}

  Concluí-se que a heurística Cuthill-McKee reversa é inferior ao mínimo grau
  múltiplo pois este último produz um menor número de elementos não nulos.

    \textbf{Referências}

    \printbibliography
    \end{column}
  \end{columns}

  \vfill
  % FIXME Remova as 7 linhas a seguir se não desejar licenciar o poster sob
  % CC-BY.
  \begin{center}
    \begin{tabular}[]{cc}
      \multirow{2}{*}{\includegraphics[height=60pt]{figures/cc-by}} &
      \Large{Exceto indicado o contrário, este trabalho é licenciado sob}\\
      & \Large{\url{http://creativecommons.org/licenses/by/3.0/}.}
    \end{tabular}
  \end{center}
  % FIXME Descomente as 7 linhas a seguir se não desejar licenciar o poster sob
  % CC-BY-SA.
% \begin{center}
%   \begin{tabular}[]{cc}
%     \multirow{2}{*}{\includegraphics[height=60pt]{figuras/cc-by-sa}} &
%     \Large{Exceto indicado o contrário, este trabalho é licenciado sob}\\
%     & \Large{\url{http://creativecommons.org/licenses/by-sa/3.0/}.}
%   \end{tabular}
% \end{center}
\end{frame}
\end{document}
