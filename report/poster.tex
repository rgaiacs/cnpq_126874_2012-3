\documentclass[]{beamer}
%%%%%%%%%%%%%%%%%%%%%%%%%%%% Configuração: pacotes %%%%%%%%%%%%%%%%%%%%%%%%%%%%
%%%%%%%%%%%%%%%%%%%%%%%%%%%%%% Pacotes: básicos %%%%%%%%%%%%%%%%%%%%%%%%%%%%%%
\usepackage[utf8]{inputenc}
\usepackage{cmap}
\usepackage[T1]{fontenc}
\usepackage[portuguese]{babel}
\usepackage{indentfirst}
\usepackage{beamerposter}
\usepackage{geometry}
%\usepackage{biblatex}
%\addbibresource{tese.bib}


%%%%%%%%%%%%%%%%%%%%%%%%%%%%%%% Pacotes: links %%%%%%%%%%%%%%%%%%%%%%%%%%%%%%%
\usepackage{url}
\usepackage{breakurl}
\usepackage{hyperref}


%%%%%%%%%%%%%%%%%%%%%%%%%%%%%%%% Pacotes: ams %%%%%%%%%%%%%%%%%%%%%%%%%%%%%%%%
\usepackage{amsmath}
\usepackage{amsfonts}
\usepackage{amssymb}
\usepackage{amsthm}
\usepackage{breqn}


%%%%%%%%%%%%%%%%%%%%%%%%%%%%%% Pacotes: tabelas %%%%%%%%%%%%%%%%%%%%%%%%%%%%%%
\usepackage{multirow}


%%%%%%%%%%%%%%%%%%%%%%%%%%%%%% Pacotes: figuras %%%%%%%%%%%%%%%%%%%%%%%%%%%%%%
\usepackage{tikz} %Pacote para desenho de figuras
\usetikzlibrary{matrix}

%%%%%%%%%%%%%%%%%%%%%%%%%%%%%%% Início do poster %%%%%%%%%%%%%%%%%%%%%%%%%%%%%%%
\geometry{paperwidth=90cm,paperheight=100cm}
\begin{document}
\begin{frame}[t,fragile]
%%%%%%%%%%%%%%%%%%%%%%%%%%%%%%% Título do poster %%%%%%%%%%%%%%%%%%%%%%%%%%%%%%%
% Com base em http://www.prp.unicamp.br/pibic/congresso_formatacaopaineis.php
  \begin{center}
    \begin{huge}
      Implementa\c{c}\~{a}o eficiente da heur\'{i}stica de reordenamento de Cuthill-McKee reversa

      \vspace{20pt}
      Instituto de Matemática, Estatística e Computação Científica
    \end{huge}

    \vspace{20pt}
    \begin{Large}
      \begin{tabular}[]{c}
        Raniere Gaia Costa da Silva \\
        \url{ra092767@ime.unicamp.br}
      \end{tabular} \hspace{5cm}
      \begin{tabular}[]{c}
        Aurelio Ribeiro Leite de Oliveira \\
        \url{aurelio@ime.unicamp.br}
      \end{tabular}
    \end{Large}
  \end{center}
  \vspace{20pt}

%%%%%%%%%%%%%%%%%%%%%%%%%%%%%%% Corpo do poster %%%%%%%%%%%%%%%%%%%%%%%%%%%%%%%
% O ambiente columns é definido na classe beamer.
  \begin{columns}[t]
    \begin{column}{0.3\textwidth}
      Neste trabalho revisitou-se a heurística de reordenamento Cuthill-McKee Reversa,
      proposta em 1969, que busca um reordenamento para matrizes esparsas que reduza a
      largura de banda destas. O autor deste trabalho implementou a heurística
      revisitada, testou com várias bibliotecas (dentre elas a ``Netlib LP'' e
      ``Meszaros'') e concluiu, pelo menos para os problemas testados, que a
      heurística Cuthill-McKee Reversa é inferior a heurística de mínimo grau
      múltiplo por gerar mais elementos não nulos na decomposição de Cholesky.

      Seguindo Cuthill e McKee \cite{Cuthill:1969:ReducingBandwidth}, definimos
      \begin{align*}
          \beta_i(A) &= i - f_i(A), 1 \leq i \leq n, \\
          \beta(A) &= \max\left\{ \beta_i(A) \mid 1 \leq i \leq n \right\}
      \end{align*}
      em que $\beta_i(A)$ é a largura de banda da $i$-ésima linha de $A$ e
      $\beta(A)$ é a largura de banda da matriz $A$. A banda da matriz $A$ é
      definida como
      \begin{align*}
          \text{Band}(A) &= \left\{ \left\{ i, j \right\} \mid 0 < i - j \leq
          \beta(A) \right\}
      \end{align*}
      e ilustrada na Figura~\ref{fig:exem_bandwidth}.
      \begin{figure}[!htb]
          \centering
          \begin{tikzpicture}
              \matrix (A) [matrix of math nodes,%
              left delimiter  = \lbrack,%
              right delimiter = \rbrack] at (0,0)
              {%
                  1 & 1 & 1 & 1 & 1 & 0 & 0 \\
                  1 & 2 & 1 & 1 & 1 & 1 & 1 \\
                  1 & 1 & 2 & 1 & 1 & 1 & 1 \\
                  1 & 1 & 1 & 2 & 1 & 0 & 0 \\
                  1 & 1 & 1 & 1 & 2 & 0 & 0 \\
                  0 & 1 & 1 & 0 & 0 & 3 & 2 \\
                  0 & 1 & 1 & 0 & 0 & 2 & 3 \\
              };
              \node[above, shift={(0,.5)}] at (A-1-1) {$1$};
              \node[above, shift={(0,.5)}] at (A-1-2) {$2$};
              \node[above, shift={(0,.5)}] at (A-1-3) {$3$};
              \node[above, shift={(0,.5)}] at (A-1-4) {$4$};
              \node[above, shift={(0,.5)}] at (A-1-5) {$5$};
              \node[above, shift={(0,.5)}] at (A-1-6) {$6$};
              \node[above, shift={(0,.5)}] at (A-1-7) {$7$};
              \node[right, shift={(-1.5,0)}] at (A-1-1) {$1$};
              \node[right, shift={(-1.5,0)}] at (A-2-1) {$2$};
              \node[right, shift={(-1.5,0)}] at (A-3-1) {$3$};
              \node[right, shift={(-1.5,0)}] at (A-4-1) {$4$};
              \node[right, shift={(-1.5,0)}] at (A-5-1) {$5$};
              \node[right, shift={(-1.5,0)}] at (A-6-1) {$6$};
              \node[right, shift={(-1.5,0)}] at (A-7-1) {$7$};
              % Largura de banda
              % \draw (A-1-6.north east) -- (A-2-7.north east);
              \draw (A-6-1.south west) -- (A-7-2.south west);
              % Banda
              \fill[gray, fill opacity=0.2, draw=black, dotted] (A-2-1.north west) -- (A-6-1.south west)
              -- (A-6-2.south west) -- (A-7-2.south west) -- (A-7-7.south west)
              \foreach \x in {6,5,4,3,2}{
              -- (A-\x-\x.south east) -- (A-\x-\x.south west) -- (A-\x-\x.north west)
              } -- (A-2-1.north west);

              \matrix (B) [matrix of math nodes,%
              left delimiter  = \lbrack,%
              right delimiter = \rbrack] at (8,0)
              {%
                 10 &  0 &  2 &  3 &  0 &  0 &  0 \\
                  0 & 10 &  2 &  3 &  0 &  0 &  0 \\
                  2 &  2 & 10 &  0 &  4 &  0 &  0 \\
                  3 &  3 &  0 & 10 &  4 &  0 &  0 \\
                  0 &  0 &  4 &  4 & 10 &  2 &  2 \\
                  0 &  0 &  0 &  0 &  2 & 10 &  0 \\
                  0 &  0 &  0 &  0 &  2 &  0 & 10 \\
              };
              \node[above, shift={(0,.5)}] at (B-1-1) {$1$};
              \node[above, shift={(0,.5)}] at (B-1-2) {$2$};
              \node[above, shift={(0,.5)}] at (B-1-3) {$3$};
              \node[above, shift={(0,.5)}] at (B-1-4) {$4$};
              \node[above, shift={(0,.5)}] at (B-1-5) {$5$};
              \node[above, shift={(0,.5)}] at (B-1-6) {$6$};
              \node[above, shift={(0,.5)}] at (B-1-7) {$7$};
              \node[right, shift={(-1.5,0)}] at (B-1-1) {$1$};
              \node[right, shift={(-1.5,0)}] at (B-2-1) {$2$};
              \node[right, shift={(-1.5,0)}] at (B-3-1) {$3$};
              \node[right, shift={(-1.5,0)}] at (B-4-1) {$4$};
              \node[right, shift={(-1.5,0)}] at (B-5-1) {$5$};
              \node[right, shift={(-1.5,0)}] at (B-6-1) {$6$};
              \node[right, shift={(-1.5,0)}] at (B-7-1) {$7$};
              % Largura de banda
              % \draw (B-1-4.north east) -- (B-4-7.north east);
              \draw (B-4-1.south west) -- (B-7-4.south west);
              % Banda
              \fill[gray, fill opacity=0.2, draw=black, dotted] (B-1-1.south west) -- (B-4-1.south west)
              -- (B-4-2.south west) -- (B-5-2.south west) -- (B-5-3.south west)
              -- (B-6-3.south west) -- (B-6-4.south west) -- (B-7-4.south west)
              -- (B-7-7.south west)
              \foreach \x in {6,5,4,3,2}{
              -- (B-\x-\x.south east) -- (B-\x-\x.south west) -- (B-\x-\x.north west)
              } -- (B-1-1.south west);
          \end{tikzpicture}
          \caption{Ilustração da banda, sombreado e contornado por pontos, e da
          largura de banda, linha contínua, para duas matrizes simétricas.}
          \label{fig:exem_bandwidth}
      \end{figure}

      O envelope da matriz $A$ é definido como
      \begin{align*}
          \text{Env}(A) &= \left\{ \left\{ i, j \right\} \mid 0 < i - j \leq
          \beta_i(A) \right\}.
      \end{align*}
      e ilustrado na Figura~\ref{fig:exem_profile}.
      \begin{figure}[!htb]
          \centering
          \begin{tikzpicture}
              \matrix (A) [matrix of math nodes,%
              left delimiter  = \lbrack,%
              right delimiter = \lbrack] at (0,0)
              {%
                  1 & 1 & 1 & 1 & 1 & 0 & 0 \\
                  1 & 2 & 1 & 1 & 1 & 1 & 1 \\
                  1 & 1 & 2 & 1 & 1 & 1 & 1 \\
                  1 & 1 & 1 & 2 & 1 & 0 & 0 \\
                  1 & 1 & 1 & 1 & 2 & 0 & 0 \\
                  0 & 1 & 1 & 0 & 0 & 3 & 2 \\
                  0 & 1 & 1 & 0 & 0 & 2 & 3 \\
              };
              \node[above, shift={(0,.5)}] at (A-1-1) {$1$};
              \node[above, shift={(0,.5)}] at (A-1-2) {$2$};
              \node[above, shift={(0,.5)}] at (A-1-3) {$3$};
              \node[above, shift={(0,.5)}] at (A-1-4) {$4$};
              \node[above, shift={(0,.5)}] at (A-1-5) {$5$};
              \node[above, shift={(0,.5)}] at (A-1-6) {$6$};
              \node[above, shift={(0,.5)}] at (A-1-7) {$7$};
              \node[right, shift={(-1.5,0)}] at (A-1-1) {$1$};
              \node[right, shift={(-1.5,0)}] at (A-2-1) {$2$};
              \node[right, shift={(-1.5,0)}] at (A-3-1) {$3$};
              \node[right, shift={(-1.5,0)}] at (A-4-1) {$4$};
              \node[right, shift={(-1.5,0)}] at (A-5-1) {$5$};
              \node[right, shift={(-1.5,0)}] at (A-6-1) {$6$};
              \node[right, shift={(-1.5,0)}] at (A-7-1) {$7$};
              % Perfil
              \fill[gray, fill opacity=0.2, draw=black, dotted] (A-2-1.north west) -- (A-6-1.north
              west) -- (A-6-2.north west) --
              (A-7-2.south west) -- (A-7-7.south west)
              \foreach \x in {6,5,4,3,2}{
              -- (A-\x-\x.south east) -- (A-\x-\x.south west) -- (A-\x-\x.north west)
              } -- (A-2-1.north west);

              \matrix (B) [matrix of math nodes,%
              left delimiter  = \lbrack,%
              right delimiter = \rbrack] at (8,0)
              {%
                 10 &  0 &  2 &  3 &  0 &  0 &  0 \\
                  0 & 10 &  2 &  3 &  0 &  0 &  0 \\
                  2 &  2 & 10 &  0 &  4 &  0 &  0 \\
                  3 &  3 &  0 & 10 &  4 &  0 &  0 \\
                  0 &  0 &  4 &  4 & 10 &  2 &  2 \\
                  0 &  0 &  0 &  0 &  2 & 10 &  0 \\
                  0 &  0 &  0 &  0 &  2 &  0 & 10 \\
              };
              \node[above, shift={(0,.5)}] at (B-1-1) {$1$};
              \node[above, shift={(0,.5)}] at (B-1-2) {$2$};
              \node[above, shift={(0,.5)}] at (B-1-3) {$3$};
              \node[above, shift={(0,.5)}] at (B-1-4) {$4$};
              \node[above, shift={(0,.5)}] at (B-1-5) {$5$};
              \node[above, shift={(0,.5)}] at (B-1-6) {$6$};
              \node[above, shift={(0,.5)}] at (B-1-7) {$7$};
              \node[right, shift={(-1.5,0)}] at (B-1-1) {$1$};
              \node[right, shift={(-1.5,0)}] at (B-2-1) {$2$};
              \node[right, shift={(-1.5,0)}] at (B-3-1) {$3$};
              \node[right, shift={(-1.5,0)}] at (B-4-1) {$4$};
              \node[right, shift={(-1.5,0)}] at (B-5-1) {$5$};
              \node[right, shift={(-1.5,0)}] at (B-6-1) {$6$};
              \node[right, shift={(-1.5,0)}] at (B-7-1) {$7$};
              % Perfil
              \fill[gray, fill opacity=0.2, draw=black, dotted] (B-2-1.south west) -- (B-5-1.north
              west) -- (B-5-3.north west) --
              (B-5-3.south west) -- (B-6-5.north west) -- (B-7-5.south west)
              -- (B-7-7.south west)
              \foreach \x in {6,5,4,3}{
              -- (B-\x-\x.south east) -- (B-\x-\x.south west) -- (B-\x-\x.north west)
              } -- (B-2-1.south west);
          \end{tikzpicture}
          \caption{Ilustração do envelope, sombreado e contornado por
          pontos, para duas matrizes simétricas.}
          \label{fig:exem_profile}
      \end{figure}
    \end{column}
    \hfill
    \begin{column}{0.3\textwidth}
    \end{column}
    \hfill
    \begin{column}{0.3\textwidth}
    \end{column}
  \end{columns}

  \vfill
  % FIXME Remova as 7 linhas a seguir se não desejar licenciar o poster sob
  % CC-BY.
  \begin{center}
    \begin{tabular}[]{cc}
      \multirow{2}{*}{\includegraphics[height=60pt]{figures/cc-by}} &
      \Large{Exceto indicado o contrário, este trabalho é licenciado sob}\\
      & \Large{\url{http://creativecommons.org/licenses/by/3.0/}.}
    \end{tabular}
  \end{center}
  % FIXME Descomente as 7 linhas a seguir se não desejar licenciar o poster sob
  % CC-BY-SA.
% \begin{center}
%   \begin{tabular}[]{cc}
%     \multirow{2}{*}{\includegraphics[height=60pt]{figuras/cc-by-sa}} &
%     \Large{Exceto indicado o contrário, este trabalho é licenciado sob}\\
%     & \Large{\url{http://creativecommons.org/licenses/by-sa/3.0/}.}
%   \end{tabular}
% \end{center}
\end{frame}
\end{document}
